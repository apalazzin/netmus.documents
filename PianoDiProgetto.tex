
\newcommand{\nomedoc}{Piano di Progetto}
\newcommand{\versione}{0.1}
\newcommand{\versioneglossario}{0.2}
\newcommand{\versionenormeprogetto}{0.6}
\newcommand{\nomefile}{PianoDiProgetto-\versione.pdf}
\newcommand{\datacreazione}{11 Dicembre 2010}
\newcommand{\datamodifica}{11 Dicembre 2010}
\newcommand{\stato}{formale}
\newcommand{\uso}{esterno}
\newcommand{\redazione}{Mandolo Andrea}
\newcommand{\verifica}{---}
\newcommand{\approvazione}{---}
\newcommand{\distribuzione}{
VT.G \\
& Prof. Vardanega Tullio\\
& Prof. Cardin Riccardo }

% FUNZIONI TIPOGRAFICHE
\newcommand{\co}{\texttt} % courier
\newcommand{\bo}{\textbf} % bold
\newcommand{\pr}{\par\medskip} % paragrafo spaziato
\newcommand{\sca}{\textsc} % small caps

\documentclass[a4paper,12pt]{report}
% 10pt,11pt,12pt
% titlepage, notitlepage -> per dare inizio o no ad una nuova pagina dopo titolo
% twoside -> per dire se fronte-retro
\usepackage[latin1]{inputenc}
% per caratteri accentati
\usepackage[italian]{babel}
% per regole sintattiche italiane
\usepackage[bookmarks=true, pdfborder={0 0 0 0}]{hyperref}
% per collegamenti ipertestuali
\usepackage{graphicx}
% per inserimento immagini

% \usepackage{enumerate}
% per personalizzare elenchi puntati

\usepackage[hmargin=2cm]{geometry} %margine 2 cm
%\geometry{options varie}

% comandi per gestire meglio header e footer
\usepackage{fancyhdr}  % header e footer
\usepackage{totpages}
\pagestyle{fancy}
\renewcommand{\headrulewidth}{0.4pt}
\renewcommand{\footrulewidth}{0.4pt}

\setlength{\headheight}{1.2cm} % NON TOCCARE
\setlength{\voffset}{-1.5cm} % NON TOCCARE
\setlength{\textheight}{700pt} % NON TOCCARE
\setlength{\parindent}{0pt} % INDENTAZIONE

\lhead{\nomedoc\  (ver. \versione)}
\chead{}
\rhead{\includegraphics[height=1cm]{img/netmus.png}}
\lfoot{\includegraphics[height=0.8cm]{img/logo.png}}
\cfoot{}
\rfoot{\thepage}

% \usepackage{listings}   per codice sorgente

\author{VT.G - Valter Texas Group}

\begin{document}

\pagenumbering{Roman} % INIZIO NUMERAZIONE ARABA

\vspace*{1cm}
\begin{center}

\includegraphics[width=9cm]{img/logo.png}\\
\vspace{0.5cm}
\begin{LARGE} \sca{VT.G - Valter Texas Group} \end{LARGE}\\
\vspace{0.5cm}
\begin{Large}
\emph{valtertexasgroup@googlegroups.com} \end{Large}\\
\vspace*{1cm} \includegraphics[width=5cm]{img/netmus.png}\\
\vspace{0.5cm}
\begin{Large} \sca{\nomedoc} \end{Large}\\
\vspace{1cm}
\begin{Large} \emph{Ingegneria del Software A.A. 2010-2011} \end{Large}\\
\end{center}
\vspace{1cm}

% INFORMAZIONI DOCUMENTO
\begin{center}
\begin{tabular}{r|l}
\hline & \\
\bo{Nome} & \nomefile \\
\bo{Versione attuale} & \versione \\
\bo{Data creazione} & \datacreazione \\
\bo{Data ultima modifica} & \datamodifica \\
\bo{Stato} & \stato \\
\bo{Uso} & \uso \\
\bo{Redazione} & \redazione \\
\bo{Verifica} & \verifica \\
\bo{Approvazione} & \approvazione \\
\bo{Distribuzione} & \distribuzione \\
& \\\hline
\end{tabular}
\end{center}
\newpage

% REGISTRO MODIFICHE
\section*{Registro delle modifiche}
\begin{tabular}{lll}
% REVISIONI DEL DOCUMENTO
% DALLA PIU' NUOVA ALLA PIU' VECCHIA

\bo{Data:} 11/12/2010 &
\bo{Versione:} 0.1 &
\bo{Autore:} Mandolo Andrea\\
\hline\\
\multicolumn{3}{p{470px}}{ Stesura prima versione del Piano di Progetto.}\\ \\

\end{tabular}

% INDICE
\tableofcontents
\thispagestyle{fancy} % per lo stile di header e footer


\chapter*{Sommario} 
Il presente documento illustra l'organigramma dettagliato del gruppo VT.G, le
assegnazioni previste, la rotazione dei ruoli di progetto, l'impegno previsto
per ogni ruolo e per ogni membro ed il conto economico preventivo.


\thispagestyle{fancy} % serve perche' nelle pagine di inizio Chapter esca header e footer
\pagenumbering{arabic} % INIZIO NUMERAZIONE NORMALE
\rfoot{\thepage\ di \pageref{TotPages}}
\addcontentsline{toc}{chapter}{Sommario}

\chapter{Introduzione}
\thispagestyle{fancy} % serve perche' nelle pagine di inizio Chapter esca header e footer

\section{Scopo del documento}
Il piano di progetto fissa le risorse disponili nel gruppo, la suddivisione
delle attivit\`a di progetto, il calendario delle attivit\`a ed un prospetto
economico preventivo. Deve descrivere l'organizzazione delle attivit\`a di
progetto al fine di produrre risultati utili al responsabile per valutare in
maniera appropriata il progresso del lavoro.


\section{Scopo del prodotto}

\section{Glossario}
Il Glossario \`e definito con un documento a parte (\emph{Glossario.pdf}). Tutti
i termini caratterizzati da \underline{questa sottolineatura} sono ivi
definiti.\\ Verr\`a sottolineata solamente la prima occorrenza di ciascun
termine presente nel Glossario, per non compromettere la leggibilit\`a del documento.

\section{Riferimenti}

\subsection{Normativi} % oppure rif. a Norme di progetto con leggi e tutto
\begin{itemize}
  \item ISO/IEC 12207:1995 - Cicli di vita software
  \item ISO/IEC 9126:2001 - Quality Model
\end{itemize}

\subsection{Informativi}
\begin{itemize}
  \item Capitolato d'appalto CO2-NETMUS del corso di Ingegneria del Software
  A.A. 2010/11 :\\
  \url{http://www.math.unipd.it/~tullio/..}
  \item Slide delle lezioni del corso :\\
  \url{http://www.math.unipd.it/~tullio/.d.}
\end{itemize}



\chapter{Organizzazione del gruppo}
\thispagestyle{fancy}

\section{Membri del gruppo}
Il gruppo VT.G - Valter Texas Group si compone in data 26/11/2010 in previsione
del progetto didattico di Ingegneria del Software 2010/2011 e, dopo aver
valutato e studiato i capitolati d'appalto presentati a lezione il 30/11/2010,
accetta di sviluppare in data 03/12/2010 il progetto presentato nel capitolato
C02 - NetMus proposto dal Prof. Cardin Riccardo.\\

Il gruppo \`e composto da 7 membri:

\begin{center}
\begin{tabular}{lcl}
\hline
\bo{Nominativo} & \bo{Matricola} & \bo{E-mail} \\
\hline
Baron Federico & 599799 & fede.baron.89@gmail.com \\
Caputo Cosimo & 524037 & caputo.cosimo85@gmail.com \\
Daminato Simone & 574545 & skyled@alice.it \\
Lovato Daniele & 578396 & danyleleorti@hotmail.com \\
Mandolo Andrea & 563175 & andrea.mandolo@gmail.com \\
Palazzin Alberto & 522095 & alberto.palazzin@gmail.com \\
Trezzi Giovanni & 487539 & giovytr@trezzi.net \\
\hline
\end{tabular}
\end{center}

\section{Ruoli di progetto}
Durante lo sviluppo del progetto saranno presenti nel gruppo i seguenti ruoli:
\begin{itemize}
  \item Responsabile
  \item Amministratore
  \item Analista
  \item Progettista
  \item Programmatore
  \item Verificatore
\end{itemize}

Tali ruoli verranno assegnati a rotazione tra i membri del gruppo, dando
cos\`\i\ la possibilit\`a a tutti di provare ogni tipo di lavoro.

\section{Attribuzione dei ruoli}
dire come disptribuiamo i ruoli,  piu ruoli senza conflitto interessi\\
dire del periodo iniziale di redazione documentale nel quale tutti analisti
\newpage

\chapter{Ciclo di vita}
\thispagestyle{fancy}


\chapter{Pianificazione}
\thispagestyle{fancy}

\section{Diagramma di Gantt}

\section{Prospetto  Economico}
\subsection{RR - RPP}
\subsection{RPP - RPD}
\subsection{RPD - RQ}
\subsection{RQ - RA}

\section{Costi complessivi}

\chapter{Analisi e gestione dei rischi}
\thispagestyle{fancy}

Questo capitolo ha lo scopo di classificare eventuali problematiche per grado
di pericolosit\`a, di analizzarle e di determinare le azioni da intraprendere per
ridurne i danni. L'attenzione ai rischi sar\`a continua nel corso del
progetto.\\ 
Nonostante il nostro gruppo non abbia ancora una visione completa e dettagliata del 
progetto, i rischi che abbiamo individuato sono riportati nelle seguenti sezioni.

\section{Indisposizione dei membri del gruppo}
Grado di pericolosit\`a: alto\\
Per i componenti del VT.G questo progetto \`e una delle prime esperienze nel campo della progettazione software in dettaglio. 
La disponibilit\`a dei membri \`e perci� uno dei rischi maggiori. 
Assenze causate da problematiche fisiche (malattie, incidenti e imprevisti vari) comporterebbero gravi perdite 
di tempo per tutti gli altri membri e conseguenti ritardi nell'avanzamento del
lavoro.\\
\\
Ogni componente \`e tenuto a confermare in modo tempestivo la propria presenza prima di ogni nuovo incontro.
I componenti di VT.G risiedono in differenti zone della regione, i ritrovi devono quindi essere organizzati 
per tempo e comunicati a tutti tempestivamente. Ritardi e/o assenza causa trasporti sono difficilmente gestibili, ma \`e possibile svolgere delle conferenze da casa tramite skype.

\section{Conoscenza delle Tecnologie Utilizzate}
Grado di pericolosit\`a: medio\\
Per la maggior parte dei membri del gruppo questo progetto \`e il primo approccio con tecnologie come Google App Engine, 
perci� almeno inizialmente questo sistema potrebbe risultare problematico e
creare ritardi nello sviluppo.\\
\\ 
\`E stato concordato che ogni membro potr\`a seguire un corso proposto
  dall'azienda di un altro capitolato, quindi le basi per l'utilizzo di questo sistema saranno assicurate per tutti. Il fattore di rischio perci� \`e dettato principalmente dalla difficolt\`a del sistema in s\`e.
Il linguaggio di programmazione adottato (java) non viene considerato come un rischio notevole, in quanto studiato durante il corso di laurea in modo esauriente. 
L'unico inconveniente pu� essere la scrittura del codice del database che sar\`a implentato in GQL, linguaggio simile a SQL 
(ampiamente trattato nel nostro corso di studio).

\section{Usabilit\`a del Prodotto}
Grado di pericolosit\`a: basso\\
Il sistema Netmus \`e un buon prodotto di catalogazione e visualizzazione delle proprie preferenze musicali, 
ma spesso i lettori mp3 che di per s\`e sono gi\`a dispositivi ad alta portabilit\`a mettono a disposizione la visualizzazione del catalogo multimediale, 
aggiornato e ordinato. Inoltre sono presenti altri software simili a
NetMus distribuiti sul web come ad esempio LastFM, molto famoso e
utilizzato nella rete. Ci� potrebbe ridurre
il numero di utenti disposti ad utilizzare il sistema NetMus.\\ 
\\
Una possibile soluzione \`e quella di decorare il progetto con delle
innovative funzionalit\`a, cos� da riuscire ad attirare un maggior numero di clienti.

\section{Strumenti d'Utilizzo}
Grado di pericolosit\`a: medio\\
Altro fattore di rischio identificato \`e l'utilizzo delle strumentazioni adatte. 
Tutti i membri devono utilizzare le stesse strumentazioni e avere le stesse possibilit\`a nel lavoro. 
Guasti hardware, imossibilit\`a di connettersi alla rete Internet, mancanza di strumentazioni software 
adatte comportano ritardi, incompatibilit\`a e sopratutto confusione. \\
\\
Per questo ogni membro del gruppo, 
rispettando le norme di progetto si \`e procurato la strumentazione software adatta prima dell'inizio del 
lavoro, e in caso di problemi pi� complessi \`e invitato a comunicare il fatto al responsabile che cercher\`a una soluzione idonea.

\section{Informazioni Aggiuntive sui Brani Musicali}
Grado di pericolosit\`a: medio\\
I file mp3 da cui si reperiscono i brani spesso non possiedono informazioni
complete, anzi una buona parte di questi file non contiene affatto informazioni, perci� la libreria virtuale 
di un utente risulterebbe composta da molti brani senza titolo,autore e altre
informazioni essenziali.\\ 
\\
Per rendere la liberia virtuale completa di tutte le informazioni si \`e ben pensato di implementare una 
funzione di ricerca automatica nel server dai record di tutti i brani registrati, cos� da completare 
quelli incompleti. Nei casi estremi inoltre sar\`a resa possibile una funzione di inserimento delle informazioni manualmente.

% \chapter{Consuntivo}            da RR in poi
% \thispagestyle{fancy}

\end{document}
