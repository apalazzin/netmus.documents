
\newcommand{\nomedoc}{Piano Di Progetto}
\newcommand{\versione}{1.2}
\newcommand{\versioneglossario}{1.0}
\newcommand{\versionenormeprogetto}{1.0}
\newcommand{\versioneAR}{1.0}
\newcommand{\nomefile}{PianoDiProgetto-\versione.pdf}
\newcommand{\datacreazione}{11 Dicembre 2010}
\newcommand{\datamodifica}{12 Gennaio 2011}
\newcommand{\stato}{formale}
\newcommand{\uso}{esterno} 
\newcommand{\redazione}{Lovato Daniele\\& Mandolo Andrea}
\newcommand{\verifica}{Trezzi Giovanni}
\newcommand{\approvazione}{Palazzin Alberto}
\newcommand{\distribuzione}{
VT.G \\
& Prof. Vardanega Tullio\\
& Prof. Cardin Riccardo}

% FUNZIONI TIPOGRAFICHE
\newcommand{\co}{\texttt} % courier
\newcommand{\bo}{\textbf} % bold
\newcommand{\pr}{\par\medskip} % paragrafo spaziato
\newcommand{\sca}{\textsc} % small caps

\documentclass[a4paper,12pt]{report}
% 10pt,11pt,12pt
% titlepage, notitlepage -> per dare inizio o no ad una nuova pagina dopo titolo
% twoside -> per dire se fronte-retro
\usepackage[latin1]{inputenc}
% per caratteri accentati
\usepackage[italian]{babel}
% per regole sintattiche italiane
\usepackage[bookmarks=true, pdfborder={0 0 0 0}]{hyperref}
% per collegamenti ipertestuali
\usepackage{graphicx}
% per inserimento immagini

% \usepackage{enumerate}
% per personalizzare elenchi puntati

\usepackage[hmargin=2cm]{geometry} %margine 2 cm
%\geometry{options varie}

% comandi per gestire meglio header e footer
\usepackage{fancyhdr}  % header e footer
\usepackage{totpages}
\pagestyle{fancy}
\renewcommand{\headrulewidth}{0.4pt}
\renewcommand{\footrulewidth}{0.4pt}

\setlength{\headheight}{1.2cm} % NON TOCCARE
\setlength{\voffset}{-1.5cm} % NON TOCCARE
\setlength{\textheight}{700pt} % NON TOCCARE
\setlength{\parindent}{0pt} % INDENTAZIONE

\lhead{\nomedoc\  (ver. \versione)}
\chead{}
\rhead{\includegraphics[height=1cm]{img/netmus.png}}
\lfoot{\includegraphics[height=0.8cm]{img/logo.png}}
\cfoot{}
\rfoot{\thepage}

% \usepackage{listings}   per codice sorgente

\author{VT.G - Valter Texas Group}

\begin{document}

\pagenumbering{Roman} % INIZIO NUMERAZIONE ARABA

\vspace*{1cm}
\begin{center}

\includegraphics[width=9cm]{img/logo.png}\\
\vspace{0.5cm}
\begin{LARGE} \sca{VT.G - Valter Texas Group} \end{LARGE}\\
\vspace{0.5cm}
\begin{Large}
\emph{valtertexasgroup@googlegroups.com} \end{Large}\\
\vspace*{1cm} \includegraphics[width=5cm]{img/netmus.png}\\
\vspace{0.5cm}
\begin{Large} \sca{\nomedoc} \end{Large}\\
\vspace{1cm}
\begin{Large} \emph{Ingegneria del Software A.A. 2010-2011} \end{Large}\\
\end{center}
\vspace{1cm}

% INFORMAZIONI DOCUMENTO
\begin{center}
\begin{tabular}{r|l}
\hline & \\
\bo{Nome} & \nomefile \\
\bo{Versione attuale} & \versione \\
\bo{Data creazione} & \datacreazione \\
\bo{Data ultima modifica} & \datamodifica \\
\bo{Stato} & \stato \\
\bo{Uso} & \uso \\
\bo{Redazione} & \redazione \\
\bo{Verifica} & \verifica \\
\bo{Approvazione} & \approvazione \\
\bo{Distribuzione} & \distribuzione \\
& \\\hline
\end{tabular}
\end{center}
\newpage

% REGISTRO MODIFICHE
\section*{Registro delle modifiche}

\begin{longtable}{|p{0.13\textwidth}|c|p{0.2\textwidth}|p{0.46\textwidth}|}
\hline
\rowcolor{orange} \bo{Data} & \bo{Versione} & \bo{Autore} & \bo{Descrizione} \\
\hline
\endhead
\hline
\endfoot
19/01/2011 & 1.2 & Lovato Daniele & Perfezionata Analisi dei Rischi e aggiunta
sezione dei rischi rilevati nella fase RR-RP.\\
\hline
12/01/2011 & 1.1 & Mandolo Andrea & Modificato layout Registro delle
modifiche.\\
\hline
19/12/2010 & 1.0 & Palazzin Alberto & Validazione per consegna RR.\\
\hline
18/12/2010 & 0.6 & Trezzi Giovanni & Verificato l'intero documento.\\
\hline
16/12/2010 & 0.5 & Mandolo Andrea & Corretti errori ortografici e sintattici.\\
\hline
13/12/2010 & 0.4 & Lovato Daniele & Corretti errori grammaticali e di
battitura\\
&&& aggiunti commenti ai grafici.\\
\hline
13/12/2010 & 0.3 & Mandolo Andrea & Completati contenuti dei capitoli 2-3-4.\\
\hline
12/12/2010 & 0.2 & Lovato Daniele & Aggiunta sezione analisi dei rischi.\\
\hline
11/12/2010 & 0.1 & Mandolo Andrea & Stesura prima versione del documento.\\
\end{longtable}


% INDICE
\tableofcontents
\thispagestyle{fancy} % per lo stile di header e footer


\chapter*{Sommario}
Il presente documento descrive l'organizzazione del gruppo VT.G, le
assegnazioni previste, la rotazione dei ruoli di progetto, l'impegno previsto
per ogni ruolo e per ogni membro ed il conto economico preventivo.\\
Viene inoltre presentata un'analisi dei rischi presenti durante il progetto, con relativa strategia preventiva su di essi.


\thispagestyle{fancy} % serve perche' nelle pagine di inizio Chapter esca header e footer
\pagenumbering{arabic} % INIZIO NUMERAZIONE NORMALE
\rfoot{\thepage\ di \pageref{TotPages}}
\addcontentsline{toc}{chapter}{Sommario}

\chapter{Introduzione}
\thispagestyle{fancy} % serve perche' nelle pagine di inizio Chapter esca header e footer

\section{Scopo del documento}
Il piano di progetto fissa le risorse disponili nel gruppo, la suddivisione
delle attivit\`a di progetto, il calendario delle attivit\`a ed un prospetto
economico preventivo. Deve descrivere l'organizzazione delle attivit\`a di
progetto al fine di produrre risultati utili al responsabile per valutare in
maniera appropriata il progresso del lavoro.


\section{Scopo del prodotto}

\section{Glossario}
Il Glossario \`e definito con un documento a parte (\emph{Glossario.pdf}). Tutti
i termini caratterizzati da \underline{questa sottolineatura} sono ivi
definiti.\\ Verr\`a sottolineata solamente la prima occorrenza di ciascun
termine presente nel Glossario, per non compromettere la leggibilit\`a del documento.

\section{Riferimenti}

\subsection{Normativi} % oppure rif. a Norme di progetto con leggi e tutto
\begin{itemize}
  \item ISO/IEC 12207:1995 - Cicli di vita software
  \item ISO/IEC 9126:2001 - Quality Model
\end{itemize}

\subsection{Informativi}
\begin{itemize}
  \item Capitolato d'appalto CO2-NETMUS del corso di Ingegneria del Software
  A.A. 2010/11 :\\
  \url{http://www.math.unipd.it/~tullio/..}
  \item Slide delle lezioni del corso :\\
  \url{http://www.math.unipd.it/~tullio/.d.}
\end{itemize}



\chapter{Organizzazione del gruppo}
\thispagestyle{fancy}

\section{Membri del gruppo}
Il gruppo VT.G - Valter Texas Group si compone in data 26/11/2010 in previsione
del progetto didattico di Ingegneria del Software 2010/2011, inserito
all'interno del corso di laurea in Informatica dell'Universit\`a di Padova,
e dopo aver valutato e studiato i capitolati d'appalto presentati a lezione il
30/11/2010, accetta di sviluppare in data 03/12/2010 il progetto presentato nel capitolato C02 - NetMus proposto dal Prof. Cardin Riccardo.\\

Il gruppo \`e composto da 7 membri:

\begin{center}
\begin{tabular}{lcl}
\hline
\bo{Nominativo} & \bo{Matricola} & \bo{E-mail} \\
\hline
Baron Federico & 599799 & fede.baron.89@gmail.com \\
Caputo Cosimo & 524037 & caputo.cosimo85@gmail.com \\
Daminato Simone & 574545 & skyled@alice.it \\
Lovato Daniele & 578396 & danyleleorti@hotmail.com \\
Mandolo Andrea & 563175 & andrea.mandolo@gmail.com \\
Palazzin Alberto & 522095 & alberto.palazzin@gmail.com \\
Trezzi Giovanni & 487539 & giovytr@trezzi.net \\
\hline
\end{tabular}
\end{center}

\section{Ruoli di progetto}
Durante lo sviluppo del progetto saranno presenti nel gruppo i seguenti ruoli:
\begin{itemize}
  \item Responsabile
  \item Amministratore
  \item Analista
  \item Progettista
  \item Programmatore
  \item Verificatore
\end{itemize}

Tali ruoli verranno assegnati a rotazione tra i membri del gruppo, dando
cos\`\i\ la possibilit\`a a tutti di provare ogni tipo di lavoro.

\section{Attribuzione dei ruoli}
Nel capitolo 4 di Pianificazione verr\`a illustrata nel dettaglio la distribuzione dei ruoli
all'interno del gruppo, nei periodi compresi tra le varie revisioni.
In un certo periodo, garantendo assenza di conflitto d'interessi, un membro
coprir\`a pi\`u ruoli, ma sempre assicurando un'adeguata ripartizione delle ore
di lavoro individuale.\\

Per il periodo della stesura iniziale dei documenti, ciascun
componente ricopre il ruolo di analista, in modo da poter garantire una pi\`u
accurata definizione e studio dei requisiti iniziali.

\chapter{Ciclo di vita}
\thispagestyle{fancy}
Lo sviluppo del progetto NetMus seguir\`a un modello di ciclo di vita
di tipo incrementale. Una volta fissate analisi dei
requisiti e progettazione architetturale si proceder\`a con due iterazioni sulle fasi di
progettazione di dettaglio e realizzazione. In questo modo si avr\`a la
possibilit\`a di correggere potenziali mancanze e migliorare il
software.\\
La prima iterazione di progettazione logica e realizzazione sar\`a la pi\`u
longeva: in essa verranno soddisfatti i requisiti obbligatori descritti nel
documento \emph{AnalisiDeiRequisiti-\versioneAR.pdf}.\\
Nella seconda iterazione invece, verranno realizzate ed integrate le parti
software che soddisfano i requisiti desiderabili ed opzionali che si \`e deciso
di implementare.\\

Questo modello, data la nostra inesperienza su progetti di tale dimensione e
sulle nuove tecnologie usate (\underline{GAE}, \underline{GWT} e
\underline{JavaFX}), ci assicura un rischio minore di fallimento: dopo la prima iterazione completa il prodotto dovrebbe
gi\`a possedere i requisiti obbligatori concordati con il committente; nella
seconda si andr\`a solamente ad aggiungere valore senza dover rivoluzionare il
lavoro gi\`a fatto.\\

Le revisioni che il gruppo VT.G dovr\`a sostenere sono:
\begin{itemize}
  \item Revisione dei requisiti (\underline{RR}): esterna;
  \item Revisione di progettazione (\underline{RP}): interna (funzione minima);
  \item Revisione di qualifica (\underline{RQ}): interna;
  \item Revisione di accettazione (\underline{RA}): esterna.
\end{itemize} \vspace{0.5cm}

`Funzione minima' alla revisione di progettazione indica che porteremo come
prodotti in ingresso la Specifica Tecnica, Piano Di Progetto v2, Piano Di
Qualifica v2 e lo stato di uscita sar\`a il prodotto specificato.\\
\\
Questa scelta \`e stata presa poich\`e riteniamo pi\`u importante avere una
valutazione esterna correttiva sulla progettazione architetturale, invece che
sulla progettazione di dettaglio.\\
\\
La Definizione Del Prodotto verr\`a sviluppata internamente durante gli
incrementi di sviluppo, cos\`\i\ come il Piano Di Progetto v3 e Piano di
Qualifica v3.


\chapter{Pianificazione}
\thispagestyle{fancy}

\section{Diagramma di Gantt}
Per facilitare la gestione oraria e il rispetto delle scadenze preposte,
forniamo un \underline{diagramma di} \underline{Gantt} che illustra in dettaglio
l'andamento delle attivit\`a del gruppo durante l'intero sviluppo del progetto.
\vspace{0.8cm}
\begin{figure}[htbp]
  \centering
  \includegraphics[width=17.2cm, angle=0]{img/PP/gantt1.png}
\caption{Diagramma di Gantt}
\end{figure}
\newpage

Come si pu\`o notare dal grafico, le revisioni sono indicate come punti di 
verifica con dei piccoli rombi neri, per semplificare la visione.\\
La fase di analisi proseguir\`a anche dopo la RR, per adattare i nostri
requisiti ad eventuali variazioni correttive emerse durante tale revisione.\\

Si pu\`o chiaramente vedere che la fase di \underline{verifica} sar\`a presente
costantemente a partire dalla stesura dei documenti, durante la progettazione
la codifica fino ad arrivare ai test di qualifica.\\

La stesura dei documenti ricoprir\`a tutta la durata del progetto, per portare
in ingresso alle revisioni i nuovi documenti richiesti o le versioni aggiornate
di quelli precedentemente presentati.

\section{Prospetto  Economico}
Come specificato nelle richieste del progetto didattico, ciascun ruolo avr\`a un
determinato costo orario (vedi tabella sottostante) che contribuir\`a al calcolo
del costo complessivo del prodotto, che ricordiamo non dovr\`a essere
inferiore a 13.000 Euro.

\vspace{1cm}
\begin{table}[h]
\begin{center}
\begin{tabular}{|l|c|}
\hline
\rowcolor{orange}
\bo{Ruolo}  & \bo{Costo(\euro)} \\
\hline Responsabile & 30 \\ \hline
Amministratore & 20 \\ \hline
Analista & 25 \\ \hline
Progettista & 22 \\ \hline
Programmatore & 15 \\ \hline
Verificatore & 15 \\
\hline
\end{tabular}
\caption{Ruoli e costi orari}
\end{center}
\end{table}


\vspace{0.5cm}
A partire dalla consegna dei documenti per la RR, verranno tracciate le ore di
lavoro dei membri del gruppo. L'impegno totale di ciascun componente dovr\`a
essere compreso tra 85 e 105 ore, dal quale risulteranno proporzionali
ripartizioni tra carico di lavoro e responsabilit\`a.\\
\\
Di seguito riportiamo una stima dei costi per ogni revisione e il carico di
lavoro previsto per ogni membro del gruppo.
\newpage

\subsection{RR - RPP}

\vspace{0.5cm}
\bo{Responsabile:} Daminato Simone\\

\bo{Amministratore:} Baron Federico

\vspace{1cm}
\begin{table}[h]
\begin{center}
\begin{tabular}{|l|c|c|c|c|c|c|c|}
\hline
& \bo{Resp.}\cellcolor{orange} & \bo{Amm.}\cellcolor{orange} &
\bo{Anl.}\cellcolor{orange} & \bo{Proget.}\cellcolor{orange} &
\bo{Program.}\cellcolor{orange} & \bo{Verif.}\cellcolor{orange} & \bo{Ore
Totali}\cellcolor{orange} \\ \hline

\bo{Baron}\cellcolor{orange}    &   & 15 &  5 &  6 & &   & 26 \\ \hline
\bo{Caputo}\cellcolor{orange}   &   &    &  5 & 12 & & 5 & 22 \\ \hline
\bo{Daminato}\cellcolor{orange} & 12&    &    &  7 & & 6 & 25 \\ \hline
\bo{Lovato}\cellcolor{orange}   &   &    & 10 & 10 & &   & 20 \\ \hline
\bo{Mandolo}\cellcolor{orange}  &   &    &  8 & 12 & &   & 20 \\ \hline
\bo{Palazzin}\cellcolor{orange} &   &    &  6 &  9 & & 5 & 20 \\ \hline
\bo{Trezzi}\cellcolor{orange}   &   &    &  7 & 10 & & 6 & 23 \\  \hline

\end{tabular}
\caption{Pianificazione oraria RR-RPP}
\end{center}
\end{table}
\vspace{0.5cm}

\bo{Ore totali:} 156.\\

\bo{Costo previsto per il periodo RR-RPP:} \euro\ 3467.

\vspace{0.8cm}
\begin{figure}[htbp]
  \centering
  \includegraphics[width=17.2cm, angle=0]{img/PP/RR-RPP.png}
\caption{Distribuzione ore RR-RPP}
\end{figure}
\newpage


\subsection{RPP - RPD}

\vspace{0.5cm}
\bo{Responsabile:} Palazzin Alberto\\

\bo{Amministratore:} Trezzi Giovanni

\vspace{1cm}
\begin{table}[h]
\begin{center}
\begin{tabular}{|l|c|c|c|c|c|c|c|}
\hline
& \bo{Resp.}\cellcolor{orange} & \bo{Amm.}\cellcolor{orange} &
\bo{Anl.}\cellcolor{orange} & \bo{Proget.}\cellcolor{orange} &
\bo{Program.}\cellcolor{orange} & \bo{Verif.}\cellcolor{orange} & \bo{Ore
Totali}\cellcolor{orange} \\ \hline

\bo{Baron}\cellcolor{orange}    &    &    & 2 & 10 & 7 & 5 & 24 \\ \hline
\bo{Caputo}\cellcolor{orange}   &    &    & 4 & 11 &   & 8 & 23 \\ \hline
\bo{Daminato}\cellcolor{orange} &    &    & 7 & 12 &   & 5 & 24 \\ \hline
\bo{Lovato}\cellcolor{orange}   &    &    &   & 12 & 3 & 7 & 22 \\ \hline
\bo{Mandolo}\cellcolor{orange}  &    &    & 2 &  7 & 8 & 7 & 24 \\ \hline
\bo{Palazzin}\cellcolor{orange} & 12 &    & 4 & 10 &   &   & 26\\ \hline
\bo{Trezzi}\cellcolor{orange}   &    & 13 &   &  9 & 2 &   & 24 \\  \hline

\end{tabular}
\caption{Pianificazione oraria RPP-RPD}
\end{center}
\end{table}
\vspace{0.5cm}

\bo{Ore totali:} 167.\\

\bo{Costo previsto per il periodo RPP-RPD:} \euro\ 3437.

\vspace{0.8cm}
\begin{figure}[htbp]
  \centering
  \includegraphics[width=17.2cm, angle=0]{img/PP/RPP-RPD.png}
\caption{Distribuzione ore RPP-RPD}
\end{figure}
\newpage



\subsection{RPD - RQ}

\vspace{0.5cm}
\bo{Responsabile:} Trezzi Giovanni\\

\bo{Amministratore:} Caputo Cosimo

\vspace{1cm}
\begin{table}[h]
\begin{center}
\begin{tabular}{|l|c|c|c|c|c|c|c|}
\hline
& \bo{Resp.}\cellcolor{orange} & \bo{Amm.}\cellcolor{orange} &
\bo{Anl.}\cellcolor{orange} & \bo{Proget.}\cellcolor{orange} &
\bo{Program.}\cellcolor{orange} & \bo{Verif.}\cellcolor{orange} & \bo{Ore
Totali}\cellcolor{orange} \\ \hline

\bo{Baron}\cellcolor{orange}    &    &    &    &  8 & 15 &  5 & 28 \\ \hline
\bo{Caputo}\cellcolor{orange}   &    & 14 &    &    & 16 &  3 & 33 \\ \hline
\bo{Daminato}\cellcolor{orange} &    &    &    &  8 & 20 &    & 28 \\ \hline
\bo{Lovato}\cellcolor{orange}   &    &    &    &  4 & 21 &  8 & 33 \\ \hline
\bo{Mandolo}\cellcolor{orange}  &    &    &    &  8 & 18 &  4 & 30 \\ \hline
\bo{Palazzin}\cellcolor{orange} &    &    &    &  4 & 18 &  8 & 30 \\ \hline
\bo{Trezzi}\cellcolor{orange}   &  9 &    &    &  3 & 16 &  2 & 30 \\ \hline

\end{tabular}
\caption{Pianificazione oraria RPD-RQ}
\end{center}
\end{table}
\vspace{0.5cm}

\bo{Ore totali:} 212.\\

\bo{Costo previsto per il periodo RPD-RQ:} \euro\ 3630.

\vspace{0.8cm}
\begin{figure}[htbp]
  \centering
  \includegraphics[width=17.2cm, angle=0]{img/PP/RPD-RQ.png}
\caption{Distribuzione ore RPD-RQ}
\end{figure}
\newpage


\subsection{RQ - RA}

\vspace{0.5cm}
\bo{Responsabile:} Mandolo Andrea\\

\bo{Amministratore:} Lovato Daniele

\vspace{1cm}
\begin{table}[h]
\begin{center}
\begin{tabular}{|l|c|c|c|c|c|c|c|}
\hline
& \bo{Resp.}\cellcolor{orange} & \bo{Amm.}\cellcolor{orange} &
\bo{Anl.}\cellcolor{orange} & \bo{Proget.}\cellcolor{orange} &
\bo{Program.}\cellcolor{orange} & \bo{Verif.}\cellcolor{orange} & \bo{Ore
Totali}\cellcolor{orange} \\ \hline

\bo{Baron}\cellcolor{orange}    &    &    &    &    &  3 & 18 & 21 \\ \hline
\bo{Caputo}\cellcolor{orange}   &    &    &    &    &  6 & 15 & 21 \\ \hline
\bo{Daminato}\cellcolor{orange} &    &    &    &    &  4 & 18 & 22 \\ \hline
\bo{Lovato}\cellcolor{orange}   &    & 13 &    &    &    & 11 & 24 \\ \hline
\bo{Mandolo}\cellcolor{orange}  & 10 &    &    &    &    & 15 & 25 \\ \hline
\bo{Palazzin}\cellcolor{orange} &    &    &    &    &  7 & 16 & 23 \\ \hline
\bo{Trezzi}\cellcolor{orange}   &    &    &    &    &  4 & 18 & 22 \\  \hline

\end{tabular}
\caption{Pianificazione oraria RQ-RA}
\end{center}
\end{table}
\vspace{0.5cm}

\bo{Ore totali:} 158.\\

\bo{Costo previsto per il periodo RQ-RA:} \euro\ 2585.

\vspace{0.8cm}
\begin{figure}[htbp]
  \centering
  \includegraphics[width=17.2cm, angle=0]{img/PP/RQ-RA.png}
\caption{Distribuzione ore RQ-RA}
\end{figure}
\newpage


\section{Costi complessivi}

\vspace{0.5cm}
In tutto il periodo di sviluppo del software, verranno complessivamente usate
693 ore per un costo totale di \euro\ 13.119,00.

\vspace{0.3cm}
\begin{table}[h]
\begin{center}
\begin{tabular}{|l|c|c|c|c|c|c|c|}
\hline
& \bo{Resp.}\cellcolor{orange} & \bo{Amm.}\cellcolor{orange} &
\bo{Anl.}\cellcolor{orange} & \bo{Proget.}\cellcolor{orange} &
\bo{Program.}\cellcolor{orange} & \bo{Verif.}\cellcolor{orange} & \bo{Ore
Totali}\cellcolor{orange} \\ \hline

\bo{Baron}\cellcolor{orange}    &  0 & 15 &  7 & 24 & 25 & 28 & 99 \\ \hline
\bo{Caputo}\cellcolor{orange}   &  0 & 14 &  9 & 23 & 22 & 31 & 99 \\ \hline
\bo{Daminato}\cellcolor{orange} & 12 &  0 &  7 & 27 & 24 & 29 & 99 \\ \hline
\bo{Lovato}\cellcolor{orange}   &  0 & 13 & 10 & 26 & 24 & 26 & 99 \\ \hline
\bo{Mandolo}\cellcolor{orange}  & 10 &  0 & 10 & 27 & 26 & 26 & 99 \\ \hline
\bo{Palazzin}\cellcolor{orange} & 12 &  0 & 10 & 23 & 25 & 29 & 99 \\ \hline
\bo{Trezzi}\cellcolor{orange}   &  9 & 13 &  7 & 22 & 22 & 26 & 99 \\ 
\hline

\end{tabular}
\caption{Pianificazione oraria totale}
\end{center}
\end{table}
\vspace{0cm}

Riportiamo qui di seguito un grafico che illustra la ripartizione dei ruoli
sul conto di ore totale.\\



\vspace{0cm}
\begin{figure}[htbp]
  \centering
  \includegraphics[width=17.2cm, angle=0]{img/PP/totale.png}
\caption{Grafico ripartizione ruoli su ore totali}
\end{figure}

\vspace{0.8cm}
 
Vogliamo fare un piccolo appunto riguardo l'esigua fetta di tempo dedicata
all'analisi. La nostra contabilizzazione economica parte dalla consegna dei documenti per 
la RR (20 Dicembre 2010), momento in cui la parte pi\`u importante dell'analisi \`e stata fatta.
L'analisi che viene riportata in questo grafico \`e conseguenza delle
rivisitazioni dovute alle eventuali variazioni che possono presentarsi alla RR.\\ 

\chapter{Analisi e gestione dei rischi}
\thispagestyle{fancy}

Questo capitolo ha lo scopo di classificare eventuali problematiche per grado
di pericolosit\`a, di analizzarle e di determinare le azioni da intraprendere per
ridurne i danni. L'attenzione ai rischi sar\`a continua nel corso del
progetto.\\ 
Nonostante il nostro gruppo non abbia ancora una visione completa e dettagliata del 
progetto, i rischi che abbiamo individuato sono riportati nelle seguenti sezioni.

\section{Elenco dei rischi maggiori}

\subsection{Indisposizione dei membri del gruppo}
\bo{Grado di pericolosit\`a}: alto\\
Per i componenti del VT.G questo progetto \`e una delle prime esperienze nel campo della progettazione software in dettaglio. 
La disponibilit\`a dei membri \`e perci\`o uno dei rischi maggiori. 
Assenze causate da problematiche fisiche (malattie, incidenti e imprevisti vari) comporterebbero gravi perdite 
di tempo per tutti gli altri membri e conseguenti ritardi nell'avanzamento del
lavoro.\\
La Proabilit� di occorrenza di questo rischio � direttamente proporzionale alla
frequenza con cui questi incontri si svolgono. 
Si pu� affermare che la minor frequenza di riunioni � una conseguenza dell'indisposizione di alcuni membri.
Ad ogni modo la probabilit\`a stimata \`e del 30\%.\\
\\
Ogni componente \`e tenuto a confermare in modo tempestivo la propria presenza
prima di ogni nuovo incontro. I componenti di VT.G risiedono in differenti zone
della regione, i ritrovi devono quindi essere organizzati per tempo e comunicati
a tutti tempestivamente. Ritardi e/o assenza causa trasporti sono difficilmente
gestibili, ma \`e possibile svolgere delle conferenze da casa tramite il
software di videoconferenza Skype.

\subsection{Conoscenza delle Tecnologie Utilizzate}
\bo{Grado di pericolosit\`a}: medio\\
Per la maggior parte dei membri del gruppo questo progetto \`e il primo approccio con tecnologie come GAP, 
perci\`o almeno inizialmente questo sistema potrebbe risultare problematico e
creare ritardi nello sviluppo.\\
La Probabilit\`a di occorrenza \`e stata stimata del 20\%.\\
\\ 
\`E stato concordato che ogni membro potr\`a seguire un corso proposto
  dall'azienda di un altro capitolato, quindi le basi per l'utilizzo di questo
  sistema saranno assicurate per tutti. Il fattore di rischio perci\`o \`e
  dettato principalmente dalla difficolt\`a del sistema in s\`e. Il linguaggio
  di programmazione adottato (\underline{Java}) non viene considerato come un
  rischio notevole, in quanto studiato durante il corso di laurea in modo
  esauriente. L'unico inconveniente pu\`o essere la scrittura del codice del
  database che sar\`a implentato in \underline{GQL}, linguaggio simile a
  \underline{SQL} (trattato sufficientemente nel nostro corso di studio), usato
  da \underline{Google Datastore}.

\subsection{Usabilit\`a del Prodotto}
\bo{Grado di pericolosit\`a}: basso\\
Il sistema NetMus \`e un buon prodotto di catalogazione e visualizzazione delle
proprie preferenze musicali, ma spesso i lettori Mp3 che di per s\`e sono gi\`a
dispositivi ad alta portabilit\`a mettono a disposizione la visualizzazione del
catalogo multimediale, aggiornato e ordinato. Inoltre sono presenti altri
software simili a NetMus distribuiti sul web come ad esempio LastFM, molto
famoso e utilizzato nella rete. Ci potrebbe ridurre il numero di utenti disposti
ad utilizzare il sistema NetMus.\\
La Probabilita' di occorrenza del rischio e molto variabile, dipendente
sopratutto da come viene svolto il lavoro nelle varie fasi. Ad ogni modo si
stima una probabilita' di occorrenza pari al 33\%.\\
\\
Una possibile soluzione \`e quella di decorare il progetto con delle
innovative funzionalit\`a, cos\`\i\ da riuscire ad attirare un maggior numero di clienti.

\subsection{Strumenti d'Utilizzo}
\bo{Grado di pericolosit\`a}: medio\\
Altro fattore di rischio identificato \`e l'utilizzo delle strumentazioni adatte. 
Tutti i membri devono utilizzare le stesse strumentazioni e avere le stesse possibilit\`a nel lavoro. 
Guasti hardware, impossibilit\`a di connettersi alla rete \underline{Internet},
mancanza di strumentazioni software adatte comportano ritardi, incompatibilit\`a e sopratutto confusione. \\
La Probabilit\`a di occorrenza \`e del 40\%.\\
\\
Ogni membro del gruppo, 
rispettando quanto scritto nel documento
\emph{NormeDiProgetto-\versionenormeprogetto.pdf}, si \`e procurato la
strumentazione software adatta prima dell'inizio del lavoro. In caso di problemi
pi\`u complessi ogni membro \`e invitato a comunicare il fatto
all'amministratore che cercher\`a una soluzione idonea.

\subsection{Informazioni Aggiuntive sui Brani Musicali}
\bo{Grado di pericolosit\`a}: medio\\
I file Mp3 da cui si reperiscono i brani spesso non possiedono informazioni
complete, anzi una buona parte di questi file non contiene affatto informazioni, perci\`o la libreria virtuale 
di un utente risulterebbe composta da molti brani senza titolo, autore e altre
informazioni essenziali.\\
La Probabilit\`a di occorrenza \`e del 60\%.\\
\\
Per rendere la libreria virtuale completa di tutte le informazioni si \`e ben pensato di implementare una 
funzione di ricerca automatica nel server dai record di tutti i brani registrati, cos\`\i\ da completare 
quelli incompleti. Nei casi estremi inoltre sar\`a resa possibile una funzione
di inserimento delle informazioni manualmente.

\section{Rischi nella fase RR - RP}
Il rischio maggiore che il gruppo ha dovuto affrontare in questa fase \`e stato il
dover correggere e rivedere alcuni documenti per intero. Ci\`o non ha gravato
sul monte ore totali, infatti come si pu� vedere dal Piano di Progetto era gi�
stato previsto un ritorno sui documenti dopo l'ingresso in Revisione dei
Requisiti. Perci\`o il rischio non \`e stato rilevato.\\
\\
Il grado di pericolosit\`a del rischio esposto al paragrafo 5.1.2 si \`e
ridotto. Ci\`o \`e dettato dal fatto che i membri hanno potuto avere una prima
esperienza personale con i tutorial di App Engine e con un Corso tenuto da un'esterno. 
La probabilit� di occorrenza del rischio \`e comunque alta.\\
\\
% \chapter{Consuntivo}            da RR in poi
% \thispagestyle{fancy}

\listoftables
\addcontentsline{toc}{chapter}{Indice Tabelle}
\listoffigures
\addcontentsline{toc}{chapter}{Indice Figure}
\end{document}
