
\newcommand{\nomedoc}{Piano di qualifica}
\newcommand{\versione}{0.1}
\newcommand{\nomefile}{PianoQualifica\versione.pdf}
\newcommand{\datacreazione}{9 Dicembre 2010}
\newcommand{\datamodifica}{9 Dicembre 2010}
\newcommand{\stato}{formale}
\newcommand{\uso}{interno}
\newcommand{\redazione}{Giovanni Trezzi}
\newcommand{\verifica}{Cosimo Caputo}
\newcommand{\approvazione}{Valter}
\newcommand{\distribuzione}{
VT.G \\
& Prof. Vardanega Tullio }

% FUNZIONI TIPOGRAFICHE
\newcommand{\co}{\texttt} % courier
\newcommand{\bo}{\textbf} % bold
\newcommand{\pr}{\par\medskip} % paragrafo spaziato
\newcommand{\sca}{\textsc} % small caps

\documentclass[a4paper,12pt]{report}
% 10pt,11pt,12pt
% titlepage, notitlepage -> per dare inizio o no ad una nuova pagina dopo titolo
% twoside -> per dire se fronte-retro
\usepackage[latin1]{inputenc}
% per caratteri accentati
\usepackage[italian]{babel}
% per regole sintattiche italiane
\usepackage[bookmarks=true, pdfborder={0 0 0 0}]{hyperref}
% per collegamenti ipertestuali
\usepackage{graphicx}
% per inserimento immagini

% \usepackage{enumerate}
% per personalizzare elenchi puntati

\usepackage[hmargin=2cm]{geometry} %margine 2 cm
%\geometry{options varie}

% comandi per gestire meglio header e footer
\usepackage{fancyhdr}  % header e footer
\usepackage{totpages}
\pagestyle{fancy}
\renewcommand{\headrulewidth}{0.4pt}
\renewcommand{\footrulewidth}{0.4pt}

\setlength{\headheight}{1.2cm} % NON TOCCARE
\setlength{\voffset}{-1.5cm} % NON TOCCARE
\setlength{\textheight}{700pt} % NON TOCCARE
\setlength{\parindent}{0pt} % INDENTAZIONE

\lhead{\nomedoc\  (ver. \versione)}
\chead{}
\rhead{\includegraphics[height=1cm]{img/netmus.png}}
\lfoot{\includegraphics[height=0.8cm]{img/logo.png}}
\cfoot{}
\rfoot{\thepage}

% \usepackage{listings}   per codice sorgente

\author{VT.G - Valter Texas Group}

\begin{document}

\pagenumbering{Roman} % INIZIO NUMERAZIONE ARABA

\vspace*{1cm}
\begin{center}

\includegraphics[width=9cm]{img/logo.png}\\
\vspace{0.5cm}
\begin{LARGE} \sca{VT.G - Valter Texas Group} \end{LARGE}\\
\vspace{0.5cm}
\begin{Large}
\emph{valtertexasgroup@googlegroups.com} \end{Large}\\
\vspace*{1cm} \includegraphics[width=5cm]{img/netmus.png}\\
\vspace{0.5cm}
\begin{Large} \sca{\nomedoc} \end{Large}\\
\vspace{1cm}
\begin{Large} \emph{Ingegneria del Software A.A. 2010-2011} \end{Large}\\
\end{center}
\vspace{1cm}

% INFORMAZIONI DOCUMENTO
\begin{center}
\begin{tabular}{r|l}
\hline & \\
\bo{Nome} & \nomefile \\
\bo{Versione attuale} & \versione \\
\bo{Data creazione} & \datacreazione \\
\bo{Data ultima modifica} & \datamodifica \\
\bo{Stato} & \stato \\
\bo{Uso} & \uso \\
\bo{Redazione} & \redazione \\
\bo{Verifica} & \verifica \\
\bo{Approvazione} & \approvazione \\
\bo{Distribuzione} & \distribuzione \\
& \\\hline
\end{tabular}
\end{center}
\newpage

% REGISTRO MODIFICHE
\section*{Registro delle modifiche}
\begin{tabular}{lll}
% REVISIONI DEL DOCUMENTO
% DALLA PIU' NUOVA ALLA PIU' VECCHIA

% TEMPLATE PER REVISIONE
\bo{Data:} 03/12/2010 &
\bo{Versione:} 0.1 &
\bo{Autore:} Giovanni Trezzi\\
\hline\\
\multicolumn{3}{p{470px}}{ Stesura prima versione di modello.}\\ \\

\end{tabular}

% INDICE
\tableofcontents
\thispagestyle{fancy} % per lo stile di header e footer


\chapter*{Sommario}


\thispagestyle{fancy} % serve perche' nelle pagine di inizio Chapter esca header e footer
\pagenumbering{arabic} % INIZIO NUMERAZIONE NORMALE
\rfoot{\thepage\ di \pageref{TotPages}}
\addcontentsline{toc}{chapter}{Sommario}

\chapter{Introduzione}
\thispagestyle{fancy} % serve perche' nelle pagine di inizio Chapter esca header e footer

\section{Scopo del documento}



\section{Scopo del prodotto}

\section{Glossario}
Il Glossario \`e definito con un documento a parte (\emph{Glossario.pdf}). Tutti
i termini caratterizzati da \underline{questa sottolineatura} sono ivi
definiti.\\ Verr\`a sottolineata solamente la prima occorrenza di ciascun
termine presente nel Glossario, per non compromettere la leggibilit\`a del documento.

\section{Riferimenti}

\subsection{Normativi} % oppure rif. a Norme di progetto con leggi e tutto
\begin{itemize}
  \item ISO/IEC 12207:1995 - Cicli di vita software
  \item ISO/IEC 9126:2001 - Quality Model
\end{itemize}

\subsection{Informativi}
\begin{itemize}
  \item Capitolato d'appalto CO2-NETMUS del corso di Ingegneria del Software
  A.A. 2010/11 :\\
  \url{http://www.math.unipd.it/~tullio/..}
  \item Slide delle lezioni del corso :\\
  \url{http://www.math.unipd.it/~tullio/.d.}
\end{itemize}


% INIZIO CAPITOLO 2

\chapter{Visione generale della strategia di verifica}
\thispagestyle{fancy} % serve perche' nelle pagine di inizio Chapter esca header e footer

\section{Organizzazione, pianificazione strategica e temporale, responsabilit\`a}

Al fine di garantire un'elevata qualit\`a del prodotto finale l'impegni che ci si
assume \`e quello di pianificare ed effettuare una costante attivit\`a di
verifica in tutte le fasi del progetto. Ognis ingola modifica o l'insieme di
pi\`u modifiche comporteranno una conseguente attivit\`a di verifica. Al fine di
ottimizzare tempo e risorse, le attivit\`a verranno effettuate solamente quando
una singola o pi\`u modifiche avranno comportato differenze di tipo rilevante
all'interno di ogni fase del progetto. Le attivit\`a di verifica saranno
tracciate per mezzo di strumenti specifici. Sar\`a dovere di chi effettua le
modifiche segnalare i cambiamenti ai verificatori i quali provvederanno ad
un'ulteriore attivit\`a di verifica. Sar\`a dovere del responsabile di qualifica
accertarsi che tutte le attivit\`a si svolgano nelle modalit\`a descritte nel presente documento.

\subsection{Verifica della documentazione}

Per ogni documento dovranno essere verificati i seguenti aspetti:
\begin{itemize}
\item Sintassi e grammatica.
\item Chiarezza espositiva.
\item Correttezza e completezza dei concetti espressi.
\item Formattazione del documento secondo quando specificato dall'amministratore
nelle Norme di Progetto.
\end{itemize}

Nel caso sia necessario intervenire per correggere uno o pi\`u di questi aspetti
sar\`a dovere del verificatore contattare il redattore del documento affinch\`e
questo gestisca le relative attivit\`a di modifica. Una volta modificato, il
documento dovr\`a essere nuovamente sottoposto all'attenzione del verificatore
per una nuova attivit\`a di verifica.

\subsubsection*{VERIFICA DELLA FASE PROGETTUALE}

La fase di progettazione dovr\`a essere verificata secondo i seguenti aspetti:
Soddisfacimento di tutti i requisiti individuati nel documento Analisi dei Requisiti
Individuazione di elementi progettuali superflui, non aderenti a nessun requisito richiesto.
Corretta corrispondeza e tracciabilit\`a tra i requisiti e le relative parti del progetto.
Correttezza progettuale nel soddisfacimento dei requisiti.
Correttezza rispetto a quanto stabilito nel documento Norme di Progetto
Possibilit\`a di manutenere ed estendere i vari elementi cos\`i come previsti nella progettazione del prodotto.


Nel caso si necessario intervenire in maniera correttiva rispetto ad uno o pi\`u di questi aspetti sar\`a dovere del verificatore contattare il progettista responsabile affinch\`e provveda alle relative modifiche riguardanti la progettazione.

\subsubsection*{VERIFICA DELLA FASE DI REALIZZAZIONE/PROGRAMMAZIONE}

La fase di realizzazione/programmazione dovr\`a essere verificata secondo i seguenti aspetti:
\begin{itemize}
\item Analisi statica del codice.
\item Analisi dinamica del codice.
\item Corretto funzionamento dell'applicazione, o delle sue parti, rispetto a
quanto previsto nella fase di progettazione.
\item Affidabilit\`a
\item Efficenza
\item Portabilit\`a
\item Manutenibilit\`a
\end{itemize}

Nel caso sia necessario intervenire sul codice in base ad uno di questi aspetti
sar\`a dovere del verificatore contattare il programmatore responsabile della
relativa porzione di codice.

Inoltre, per favorire l'attivit\`a di verifica, ogni programmatore sar\`a tenuto
ad eseguire attivit\`a di debugging sul codice per cercare di diminuire il pi\`u
possibile le possibilit\`a d'errore.

\subsubsection*{VALIDAZIONE}

L'impegno che ci si assume \`e quello di fornire un prodotto completamente e
correttamente funzionante secondo i requisiti concordati. Qualora venissero
riscontrati difetti o non conformit\`a, durante la verifica in fase di collaudo,
ci si impegna ad effettuare ogni modifica o intervento che risultasse necessario
al fine di risolvere ogni eventuale problema.

\section{Risorse necessarie e disponibili}

Per il processo di qualifica sono necessarie risorse di tipo umano le quali
effettueranno attivit\`a di verifica e validazione cos\`i come definito nel
presente documento. Sono necessarie inoltre risorse di tipo logistico quali
programmi per le attivit\`a di tracciamento tra i requisiti e i moduli software,
un framework per i test specifici di unit\`a del codice, un applicativo software
per la gestione e la risoluzione delle anomalie, un'ambiente di sviluppo
adeguato con possibilit\`a di effettuare debugging da parte dei programmatori.
	
\section{Strumenti tecniche e metodi}

Per le attivit\`a di verifica ci avvarremo dei seguenti strumenti:
(provvisori, da discutere tutti assieme)

\begin{itemize}

\item Junit: verr\`a utilizzato come unit test framework per il linguaggio di
programmazione Java.

\item Emma: verr\`a utilizzato per il calcolo della copertura del codice.

\item Eclipse + GWT: useremo le funzionalit\`a di debug di eclipse su codice java
mentre l'applicazione da testare verr\`a eseguita in hosted mode tramite GWT.

\item Google Project Hosting: verr\`a utilizzata la funzionalit\`a issue tracker per
la gestione e creazione di ticket cosi come descritto in x.x

\item FindBugs: verr\`a utilizzato per effettuare analisi statiche del codice
sorgente Java al fine di trovare errori comuni.

\item Metrics: verr\`a utilizzato per calcolare la misura di varie metriche del
codice sorgente durante la fase di compilazione in modo da tenere continuamente
sotto controllo lo stato del codice stesso.

\item Cobertura: verr\`a utilizzato per misurare la carenza di test di unit\`a
all'interno di un progetto: misurare la percentuale di linee di codice che
vengono controllate dai test di unit\`a , la percentuale di diramazioni del
codice che vengono controllate dai test di unit\`a, la complessit\`a ciclomatica
di ogni classe e il numero di volte che una linea di codice \`e stata eseguita
dai testi di unit\`a.

\item CheckStyle: verr\`a utilizzato come aiuto per garantire che il codice Java
aderisca alle pi\`u diffuse norme di codifica.
\end{itemize}


% INIZIO CAPITOLO 3

\chapter{Gestione amministrativa della revisione}
\thispagestyle{fancy} % per lo stile di header e footer

\section{Comunicazione e risoluzione di anomalie}

Possiamo prevedere che la maggioranza delle anomalie verr\`a individuata in fase
di verifica. Ci\`o nonostante non possiamo escludere che vengano riscontrate
anche durante una qualsiasi fase di ciclo di vita, quindi non vogliamo limitare
il compito della comunicazione ai soli verificatori ma intendiamo estendere il
caso a tutti i membri del gruppo di lavoro.

\subsection{Comunicazione}

Per notificare anomalie dovr\`a essere usato lo strumento issues-tracker messo a
disposizione nella pagina http://code.google.com/p/netmus/ . Per ogni anomalia
individuata si dovr\`a creare un ticket. Per la creazione e la struttura dei
singoli ticket fare riferimento al documento Norme di Progetto paragrafo omonimo.

\end{document}
