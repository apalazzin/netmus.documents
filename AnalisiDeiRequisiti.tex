
\newcommand{\nomedoc}{Analisi dei requisiti}
\newcommand{\versione}{0.3}
\newcommand{\versioneglossario}{0.1}
\newcommand{\versionenormeprogetto}{0.3}
\newcommand{\nomefile}{AnalisiDeiRequisiti-\versione.pdf}
\newcommand{\datacreazione}{2 Dicembre 2010}
\newcommand{\datamodifica}{13 Dicembre 2010}
\newcommand{\stato}{formale}
\newcommand{\uso}{interno}
\newcommand{\redazione}{Mandolo Andrea}
\newcommand{\verifica}{Baffo}
\newcommand{\approvazione}{Valter}
\newcommand{\distribuzione}{
VT.G \\
& Prof. Vardanega Tullio\\
& Prof. Cardin Riccardo }

% FUNZIONI TIPOGRAFICHE
\newcommand{\co}{\texttt} % courier
\newcommand{\bo}{\textbf} % bold
\newcommand{\pr}{\par\medskip} % paragrafo spaziato
\newcommand{\sca}{\textsc} % small caps

\documentclass[a4paper,12pt]{report}
% 10pt,11pt,12pt
% titlepage, notitlepage -> per dare inizio o no ad una nuova pagina dopo titolo
% twoside -> per dire se fronte-retro
\usepackage[latin1]{inputenc}
% per caratteri accentati
\usepackage[italian]{babel}
% per regole sintattiche italiane
\usepackage[bookmarks=true, pdfborder={0 0 0 0}]{hyperref}
% per collegamenti ipertestuali
\usepackage{graphicx}
% per inserimento immagini

% \usepackage{enumerate}
% per personalizzare elenchi puntati

\usepackage[hmargin=2cm]{geometry} %margine 2 cm
%\geometry{options varie}

% comandi per gestire meglio header e footer
\usepackage{fancyhdr}  % header e footer
\usepackage{totpages}
\pagestyle{fancy}
\renewcommand{\headrulewidth}{0.4pt}
\renewcommand{\footrulewidth}{0.4pt}

\setlength{\headheight}{1.2cm} % NON TOCCARE
\setlength{\voffset}{-1.5cm} % NON TOCCARE
\setlength{\textheight}{700pt} % NON TOCCARE
\setlength{\parindent}{0pt} % INDENTAZIONE

\lhead{\nomedoc\  (ver. \versione)}
\chead{}
\rhead{\includegraphics[height=1cm]{img/netmus.png}}
\lfoot{\includegraphics[height=0.8cm]{img/logo.png}}
\cfoot{}
\rfoot{\thepage}

% \usepackage{listings}   per codice sorgente

\author{VT.G - Valter Texas Group}

\begin{document}

\pagenumbering{Roman} % INIZIO NUMERAZIONE ARABA

\vspace*{1cm}
\begin{center}

\includegraphics[width=9cm]{img/logo.png}\\
\vspace{0.5cm}
\begin{LARGE} \sca{VT.G - Valter Texas Group} \end{LARGE}\\
\vspace{0.5cm}
\begin{Large}
\emph{valtertexasgroup@googlegroups.com} \end{Large}\\
\vspace*{1cm} \includegraphics[width=5cm]{img/netmus.png}\\
\vspace{0.5cm}
\begin{Large} \sca{\nomedoc} \end{Large}\\
\vspace{1cm}
\begin{Large} \emph{Ingegneria del Software A.A. 2010-2011} \end{Large}\\
\end{center}
\vspace{1cm}

% INFORMAZIONI DOCUMENTO
\begin{center}
\begin{tabular}{r|l}
\hline & \\
\bo{Nome} & \nomefile \\
\bo{Versione attuale} & \versione \\
\bo{Data creazione} & \datacreazione \\
\bo{Data ultima modifica} & \datamodifica \\
\bo{Stato} & \stato \\
\bo{Uso} & \uso \\
\bo{Redazione} & \redazione \\
\bo{Verifica} & \verifica \\
\bo{Approvazione} & \approvazione \\
\bo{Distribuzione} & \distribuzione \\
& \\\hline
\end{tabular}
\end{center}
\newpage

% REGISTRO MODIFICHE
\section*{Registro delle modifiche}
\begin{tabular}{lll}
% REVISIONI DEL DOCUMENTO
% DALLA PIU' NUOVA ALLA PIU' VECCHIA
\bo{Data:} 13/12/2010 &
\bo{Versione:} 0.3 &
\bo{Autore:} Baron Federico\\
\hline\\
\multicolumn{3}{p{470px}}{ Stesura completa dell'elenco dei requisiti della
componente 1.}\\
\\

\bo{Data:} 09/12/2010 &
\bo{Versione:} 0.2 &
\bo{Autore:} Daminato Simone\\
\hline\\
\multicolumn{3}{p{470px}}{ Redazione del capitolo 2.}\\
\\

\bo{Data:} 07/12/2010 &
\bo{Versione:} 0.1 &
\bo{Autore:} Lovato Daniele\\
\hline\\
\multicolumn{3}{p{470px}}{ Redazione del capitolo 1.}\\
\\

\end{tabular}

% INDICE
\tableofcontents
\thispagestyle{fancy} % per lo stile di header e footer


\chapter*{Sommario}


\thispagestyle{fancy} % serve perche' nelle pagine di inizio Chapter esca header e footer
\pagenumbering{arabic} % INIZIO NUMERAZIONE NORMALE
\rfoot{\thepage\ di \pageref{TotPages}}
\addcontentsline{toc}{chapter}{Sommario}

\chapter{Introduzione}
\thispagestyle{fancy} % serve perche' nelle pagine di inizio Chapter esca header e footer

\section{Scopo del documento}
Il presente documento ha lo scopo di presentare al committente l'Analisi dei
Requisiti e le potenzialit\`a software del nostro prodotto per soddisfare le
richieste del capitolato d'appalto 02 NetMus.


\section{Scopo del prodotto}

\section{Glossario}
Il Glossario \`e definito con un documento a parte (\emph{Glossario.pdf}). Tutti
i termini caratterizzati da \underline{questa sottolineatura} sono ivi
definiti.\\ Verr\`a sottolineata solamente la prima occorrenza di ciascun
termine presente nel Glossario, per non compromettere la leggibilit\`a del documento.

\section{Riferimenti}

\subsection{Normativi} % oppure rif. a Norme di progetto con leggi e tutto
\begin{itemize}
  \item ISO/IEC 12207:1995 - Cicli di vita software
  \item ISO/IEC 9126:2001 - Quality Model
\end{itemize}

\subsection{Informativi}
\begin{itemize}
  \item Capitolato d'appalto CO2-NETMUS del corso di Ingegneria del Software
  A.A. 2010/11 :\\
  \url{http://www.math.unipd.it/~tullio/..}
  \item Slide delle lezioni del corso :\\
  \url{http://www.math.unipd.it/~tullio/.d.}
\end{itemize}


% INIZIO CAPITOLO 2
\chapter{Descrizione generale}
\thispagestyle{fancy}
\section{Contesto d'uso del prodotto}
Il progetto Netmus che abbiamo intenzione di sviluppare \`e un software
destinato ad una vasta gamma d'utenza, di et\`a variabile, stimata a 12 anni o
superiore, in particolare adolescenti e adulti dai 15 ai 35 anni.
\subsection{Piattaforma d'esecuzione e interfacciamento con l'ambiente di
installazione e uso}
Il prodotto finale utilizzer\`a diverse teconologie (in particolare: Java, GAE e
GWT) che renderanno possibile l'utilizzo del sistema su pi\`u piattaforme
d'esecuzione con sistemi operativi diversi, i cui unici due requisiti
fondamentali saranno di possedere una connessione ad internet e di aver
installata la Java Runtime Machine. In particolare, il prodotto finale sar\`a
testato sui principali sistemi operativi: Windows (xp e 7), Linux (principali e
pi\`u diffuse distribuzioni), Mac OS X (Snow Leopard).
\section{Funzioni del prodotto}
Il prodotto finale sar\`a composto da due componenti. Una componente di
persistenza e visualizzazione della libreria virtuale dell'utente. Un'altra
componente di recupero delle informazioni dei brani musicali contenuti nei
sistemi di riproduzione personale dell'utente. La prima componente avr\`a il
compito di gestire la memorizzazione delle informazioni dei brani nel database,
e la sua visualizzazione e gestione con un'interfaccia semplice e facilmente
utilizzabile. La seconda componente svolger\`a il ruolo di estrazione delle
informazioni dai brani musicali, e le invier\`a al server senza interferire con
l'utente che intanto potr\`a continuare a svolgere il proprio lavoro indisturbato.
L'estrazione avverr\`a non appena verr\`a rilevato un nuovo dispositivo di
archiviazione di massa (storage USB o lettore mp3) che non cripta i propri file,
e in modo del tutto autonomo verranno ricavate le informazioni, memorizzate in
locale in caso di assenza di connessione, e poi successivamente inviate al server.
\section{Caratteristiche degli utenti}
L'utente tipo che utilizza il sistema non \`e in possesso di particolari
conoscenze informatiche, ma \`e in grado di navigare sul web e di eseguire alcune
semplici operazioni correlate (come per esempio effettuare un login).
\section{Vincoli generali}
Il software nel suo funzionamento e sviluppo:
\begin{itemize}
  \item sar\`a completo della documentazione necessaria e del manuale d'uso,
  \item sar\`a portabile (verr\`a testato e garantito il suo funzionamento sui
  principali sistemi operativi),
  \item sar\`a sicuro per gli utenti: garantir\`a il controllo del proprio account e
  dei propri dati, e della propria privacy,
  \item per la raccolta dei dati dei brani localmente non sar\`a necessario
  l'accesso ad internet,
  \item per l'aggiornamento e la visualizzazione dei propri dati online sar\`a
  necessaria la connessione ad internet,
  \item il manuale utente sar\`a fornito sia in italiano che in inglese.
\end{itemize}
\section{Assunzioni e dipendenze}
Si assume che sui personal computer utilizzati dagli utenti sia gi\`a installata
una JVM.
\section{Glossario}
Esister\`a un unico glossario comune a tutti i documenti, contenente la
spiegazione di tutti i termini e gli acronimi che verranno utilizzati
all'interno della documentazione. Questo documento sar\`a organizzato in ordine
alfabetico in modo da permettere una rapida ricerca dei termini e sar\`a suddiviso
due categorie: termini e acronimi.

\chapter{Lista dei requisiti}
\thispagestyle{fancy}
Ad ogni requisito � stato assegnato un codice di identificazione per facilitarne
il tracciamento, per maggiori informazioni per quanto riguarda la notazione
utilizzata per catalogare i requisiti si veda il documento
\emph{NormeDiProgetto-1.0.pdf}.

\section{Componente di persistenza e visualizzazione (C1)}
\subsection{Requisiti funzionali}

\bo{WEB Application NetMus} \\
\bo{Id:} C1FN-1 \\
\bo{Tipo:} funzionale \\
\bo{Richiesta:} obbligatorio \\
\bo{Fonte:} capitolato\\
La componente 1 del software NetMus deve permettere la gestione di un catalogo
multimediale virtuale ad ogni utente che ha effettuato la registrazione.
Deve essere accessibile all'utente come applicazione WEB.\\ \\

\bo{Grafica simile ad iTunes} \\
\bo{Id:} C1FN-1.1 \\
\bo{Tipo:} funzionale \\
\bo{Richiesta:} obbligatorio \\
\bo{Fonte:} capitolato \\
L'interfaccia grafica deve dare un'esperienza di visualizzazione simile alla
libreria musicale fornita dal software iTunes (\emph{http://www.apple.com/it/itunes/}).\\
\\

\bo{Elenco brani} \\
\bo{Id:} C1FN-1.1.1 \\
\bo{Tipo:} funzionale \\
\bo{Richiesta:} obbligatorio \\
\bo{Fonte:} capitolato \\
La parte principale della visualizzazione della libreria \`e costituita
dall'elenco dei brani dell'utente opportunamente raggruppati e catalogati con la
possibilit\`a di ordinamento in base ad una o pi\`u delle categorie.
\\
\\

\bo{Menu laterali} \\
\bo{Id:} C1FN-1.1.2 \\
\bo{Tipo:} funzionale \\
\bo{Richiesta:} obbligatorio \\
\bo{Fonte:} capitolato \\
Come avviene in iTunes la navigazione tra le varie pagine deve essere favorita
da un comodo men\`u laterale alla sinistra della finestra dei brani.
\\
\\

\bo{Informazioni dettagliate brano} \\
\bo{Id:} C1FN-1.1.3 \\
\bo{Tipo:} funzionale \\
\bo{Richiesta:} obbligatorio \\
\bo{Fonte:} capitolato \\
Per ognuno dei propri brani catalogati sar\`a possibile accedere alle informazioni
dettagliate, come ad esempio la copertina dell'album, in una nuova finestra
\\
\\

\bo{Player YouTube} \\
\bo{Id:} C1FD-1.1.4 \\
\bo{Tipo:} funzionale \\
\bo{Richiesta:} desiderabile \\
\bo{Fonte:} capitolato \\
Insieme alle informazioni di un brano dovrebbe comparire anche un player fornito
da Youtube con cui ascoltarlo.
\\
\\

\bo{Registrazione} \\
\bo{Id:} C1FN-1.2 \\
\bo{Tipo:} funzionale \\
\bo{Richiesta:} obbligatorio \\
\bo{Fonte:} capitolato \\
Un individuo pu\`o diventare utente di Netmus ed avere la possibilit\`a di gestire
il proprio catalogo multimediale solamente dopo aver effettuato la
registrazione, che prevede l'inserimento di un username (unico), una
password ed un indirizzo email (attraverso cui fare la conferma) pi\`u alcune
informazioni personali.
\\
\\

\bo{Pagina di login} \\
\bo{Id:} C1FN-1.2.1 \\
\bo{Tipo:} funzionale \\
\bo{Richiesta:} obbligatorio \\
\bo{Fonte:} verbale1 \\
L'applicazione avr\`a un sistema di autenticazione proprio oltre a fornire
l'autenticazione di Google acquisita da Google AppEngine.
\\
\\

\bo{Personalizzazione catalogo} \\
\bo{Id:} C1FN-1.3 \\
\bo{Tipo:} funzionale \\
\bo{Richiesta:} obbligatorio \\
\bo{Fonte:} interna \\
La gestione del catalogo prevede alcuni metodi di personalizzazione basilari che
permettono all'utente di visualizzare la propria libreria nel modo preferito ma
che non vanno a modificare le informazioni a livello di database.
\\
\\

\bo{Cancellazione brano} \\
\bo{Id:} C1FN-1.3.1 \\
\bo{Tipo:} funzionale \\
\bo{Richiesta:} obbligatorio \\
\bo{Fonte:} verbale1 \\
Un utente potr\`a rimuovere dalla propria libreria qualche brano, i dati
rimarranno nel database.
\\
\\

\bo{Modifica informazioni brano} \\
\bo{Id:} C1FD-1.3.2 \\
\bo{Tipo:} funzionale \\
\bo{Richiesta:} desiderabile \\
\bo{Fonte:} interna \\
Tutte le informazioni di un brano potranno essere modificate dall'utente che le
ha caricate, il database non risentir\`a di questi cambiamenti quindi gli altri
utenti non risentiranno in alcun modo di questi cambiamenti.
\\
\\

\bo{Creazione playlist} \\
\bo{Id:} C1FO-1.3.3 \\
\bo{Tipo:} funzionale \\
\bo{Richiesta:} opzionale \\
\bo{Fonte:} interna \\
Viene data all'utente la possibilit\`a di creare delle liste lunghe a piacere dei
propri brani allo scopo di ordinare la propria libreria oppure di ascoltarle in
sequenza attraverso i player.
\\
\\

\bo{Ranking brani} \\
\bo{Id:} C1FO-1.3.4 \\
\bo{Tipo:} funzionale \\
\bo{Richiesta:} opzionale \\
\bo{Fonte:} interna \\
L'utente pu\`o assegnare un punteggio ad ogni brano in base al suo gradimento
personale, questo pu\`o tornare utile come criterio di ordinamento.
\\
\\

\bo{Gestione profilo personale} \\
\bo{Id:} C1FN-1.4 \\
\bo{Tipo:} funzionale \\
\bo{Richiesta:} obbligatorio \\
\bo{Fonte:} interna \\
In ogni momento l'utente di NetMus pi\`u modificare i suoi dati di registrazione,
eccetto l'username. Il profilo, se desiderato, pu\`o diventare pubblico e visibile
dagli altri utenti fornendo la possibilit\`a di confrontare il proprio catalogo
musicale e le le altre informazioni.
\\
\\

\bo{Modifica informazioni personali} \\
\bo{Id:} C1FN-1.4.1 \\
\bo{Tipo:} funzionale \\
\bo{Richiesta:} obbligatorio \\
\bo{Fonte:} interna \\
Una volta effettuato il login sar\`a possibile accedere ad una pagina apposita per
aggiornare ed aggiungere informazioni al proprio profilo.
\\
\\

\bo{Cambio password} \\
\bo{Id:} C1FN-1.4.2 \\
\bo{Tipo:} funzionale \\
\bo{Richiesta:} obbligatorio \\
\bo{Fonte:} interna \\
Quando l'utente lo desidera pu\`o cambiare la password del proprio account. \\ \\

\bo{Cancellazione account} \\
\bo{Id:} C1FN-1.4.3 \\
\bo{Tipo:} funzionale \\
\bo{Richiesta:} obbligatorio \\
\bo{Fonte:} interna \\
Deve essere possibile per ogni utente chiudere il proprio profilo in maniera
permanente eliminando dal database ogni sua informazione.
\\
\\

\bo{Pubblicazione profilo} \\
\bo{Id:} C1FD-1.4.4 \\
\bo{Tipo:} funzionale \\
\bo{Richiesta:} desiderabile \\
\bo{Fonte:} interna \\
Per interagire con gli altri sar\`a necessario pubblicare il proprio profilo e
di conseguenza la propria libreria di canzoni rendendo visibile a tutti ci\`o che
contengono.
\\
\\

\bo{Riproduzione tracce in streaming} \\
\bo{Id:} C1FD-1.5 \\
\bo{Tipo:} funzionale \\
\bo{Richiesta:} desiderabile \\
\bo{Fonte:} capitolato \\
NetMus prevede un sistema di riproduzione delle canzoni salvate nel database.
Non contenendo alcun file multimediale l'ascolto sar\`a reso possibile dalla
riproduzione in streaming fornita da Youtube. La ricerca dei brani
corrispondenti alle informazioni nel database sar\`a a totale carico
dell'applicazione NetMus.
\\
\\

\bo{Interazione con altri utenti} \\
\bo{Id:} C1FD-1.7 \\
\bo{Tipo:} funzionale \\
\bo{Richiesta:} desiderabile \\
\bo{Fonte:} interna \\
La comunicazione con altri utenti sar\`a favorita da alcune semplici funzionalit\`a
che rendono la propria libreria oggetto di confronto dando pi\`u spazio
allo scambio di idee e conoscenze tra gli utilizzatori di NetMus.
\\
\\

\bo{Visualizzazione altri profili} \\
\bo{Id:} C1FD-1.7.1 \\
\bo{Tipo:} funzionale \\
\bo{Richiesta:} desiderabile \\
\bo{Fonte:} interna \\
Sar\`a possibile visualizzare l'intera libreria musicale e le altre informazioni
di tutti gli utenti che hanno permesso la condivisione del proprio profilo.
\\
\\

\bo{Lasciare commenti} \\
\bo{Id:} C1FO-1.7.2 \\
\bo{Tipo:} funzionale \\
\bo{Richiesta:} opzionale \\
\bo{Fonte:} interna \\
Uno strumento semplice ma molto coinvolgente sar\`a la possibilit\`a di scrivere
commenti alle informazioni condivise da altri utenti come brani, autori e
playlist.
\\
\\

\bo{Elaborazione dati utente} \\
\bo{Id:} C1FD-1.8 \\
\bo{Tipo:} funzionale \\
\bo{Richiesta:} desiderabile \\
\bo{Fonte:} interna \\
Il sistema NetMus si occuper\`a anche di elaborare in modo intelligente i dati
raccolti dai vari utenti permettendo di avere nel proprio profilo dei dati
riassuntivi riguardanti la propria libreria ma anche di avere delle informazioni
di confronto con le librerie degli altri utenti.
\\
\\

\bo{Esportazione pdf} \\
\bo{Id:} C1FO-1.8.1 \\
\bo{Tipo:} funzionale \\
\bo{Richiesta:} opzionale \\
\bo{Fonte:} interna \\
Sar\`a possibile esportare il proprio catalogo opportunamente indicizzato e con
informazioni aggiuntive procurate da NetMus in formato pdf predisposto per la stampa.
\\
\\

\bo{Ricezione ed elaborazione brani} \\
\bo{Id:} C1FN-1.9 \\
\bo{Tipo:} funzionale \\
\bo{Richiesta:} obbligatorio \\
\bo{Fonte:} capitolato \\
Netmus deve occuparsi della rizeione dei dati in arrivo da tutte le componenti
di invio che si connettono ad esso, dovr\`a quindi gestire queste connessioni in
modo concorrente nel miglior modo possibile. Inoltre per tutte le informzioni
ricevute con successo si occuper\`a di verificarne la validit\`a e di arricchirne i
contenuti ove possibile.
\\
\\

\bo{Controllo validit\`a dati} \\
\bo{Id:} C1FN-1.9.1 \\
\bo{Tipo:} funzionale \\
\bo{Richiesta:} obbligatorio \\
\bo{Fonte:} capitolato \\
Tutte le informazioni inviate da C2 saranno controllate in maniera
rapida ma che garantisca che non vi siano dati maligni per NetMus o che non
riguardino brani musicali.
\\
\\

\bo{Completamento informazioni da database interno} \\
\bo{Id:} C1FN-1.9.2 \\
\bo{Tipo:} funzionale \\
\bo{Richiesta:} obbligatorio \\
\bo{Fonte:} interna \\
Sar\`a sfruttato il database interno per completare ed aggiungere informazioni
mancanti ai brani in entrata. Questo sar\`a fatto basandosi su opportuni algoritmi
di ricerca che garantiscano un buona probabilit\`a che i dati suggeriti per
l'aggiornamento siano quelli corretti.
\\
\\

\bo{Inserimento nel database} \\
\bo{Id:} C1FN-1.9.5 \\
\bo{Tipo:} funzionale \\
\bo{Richiesta:} obbligatorio \\
\bo{Fonte:} capitolato \\
Una volta controllate ed aggiornate le informazioni ricevute verranno salvate
nel database entrando a far parte a tutti gli effetti della libreria musicale
dell'utente da cui sono state ricevute.
\\
\\

\bo{Completamento informazioni da servizio esterno} \\
\bo{Id:} C1FO-1.9.3 \\
\bo{Tipo:} funzionale \\
\bo{Richiesta:} opzionale \\
\bo{Fonte:} capitolato \\
Per reperire informazioni sicure che possano servire per correggere eventuali
dati mancanti riguardo i brani inviati a NetMus, se il database interno non le
possiede, si sfrutter\`a uno o pi\`u servizi gratuiti esterni.
\\
\\

\bo{Invio nuove informazioni a C2} \\
\bo{Id:} C1FD-1.10 \\
\bo{Tipo:} funzionale \\
\bo{Richiesta:} desiderabile \\
\bo{Fonte:} verbale1 \\
Se saranno trovate informazioni aggiuntive, da database interno o esterno, ai
dati inviati da qualche utente sar\`a possibile anche, se desiderato, inviarle
a C2 per andare ad aggiornare i file multimediali dell'utente.
\\
\\

\bo{Gestione database} \\
\bo{Id:} C1FN-1.13 \\
\bo{Tipo:} funzionale \\
\bo{Richiesta:} obbligatorio \\
\bo{Fonte:} capitolato \\
Il fulcro della persistenza del sistema NetMus sar\`a un database nel quale
verranno salvati tutti i dati gestiti. In particolare il database sar\`a quello
fornito da Google AppEngine, Google DataStore che trae grandissimi benefici dal
cloud computing.
\\
\\



\subsection{Requisiti di qualit\`a}

\bo{Ottimizzazione della ricerca su YoutTube} \\
\bo{Id:} C1QD-1.5.1 \\
\bo{Tipo:} qualit\`a \\
\bo{Richiesta:} desiderabile \\
\bo{Fonte:} interna \\
Youtube \`e uno strumento estremamente utile poich\`e garantisce una vasta gamma di
dati sulla quale fare ricerche ma proprio per questo sar\`a importante che NetMus
implementi degli algoritmi opportuni che eseguano queste ricerche in modo molto
efficente.
\\
\\

\bo{Identificazione dati ridondanti} \\
\bo{Id:} C1QN-1.9.4 \\
\bo{Tipo:} qualit\`a \\
\bo{Richiesta:} obbligatorio \\
\bo{Fonte:} interna \\
NetMus sar\`a in grado di associare le informazioni simili e
riguardanti lo stesso brano in modo da inserirle una sola vola all'interno del
database. Questo verr\`a fatto solo se si ha la sicurezza che si tratti dello
stesso brano, altrimenti verrano mantenute le diverse versioni fino a quando si
avranno informazioni sufficenti per saperlo.
\\
\\

\bo{Gestione concorrenza} \\
\bo{Id:} C1QN-1.9.6 \\
\bo{Tipo:} qualit\`a \\
\bo{Richiesta:} obbligatorio \\
\bo{Fonte:} interna \\
La componente di persistenza sar\`a in grado di gestire la concorrenza tra le
richieste di invio dati simultanee da vari clients in modo da garantire una
corretta ricezione di tutte le informazioni.
\\
\\

\bo{Scalabilit\`a} \\
\bo{Id:} C1QN-1.6 \\
\bo{Tipo:} qualit\`a \\
\bo{Richiesta:} obbligatorio \\
\bo{Fonte:} capitolato \\
NetMus garantur\`a la possibilit\`a di crescere di dimensione, in tutti i vari
aspetti, senza dover effettuare alcun ritocco o modifica all'architettura del
sistema.
\\
\\

\bo{Scalabilit\`a interfaccia grafica} \\
\bo{Id:} C1QN-1.6.1 \\
\bo{Tipo:} qualit\`a \\
\bo{Richiesta:} obbligatorio \\
\bo{Fonte:} capitolato \\
La libreria personale dell'utente sar\`a potenzialmente infinita, di conseguenza
le pagine che ne permetteranno la visualizzazione saranno in grado di adattarsi
senza imporre vincoli di alcun tipo.
\\
\\

\bo{Scalabilit\`a massa di utenza} \\
\bo{Id:} C1QN-1.6.2 \\
\bo{Tipo:} qualit\`a \\
\bo{Richiesta:} obbligatorio \\
\bo{Fonte:} capitolato \\
Poich\`e il bacino d'utenza potenziale \`e estremamente elevato NetMus sar\`a molto
efficente per quanto riguarda gli accessi simultanei di molti utenti
all'applicazione Web ed al database.
\\
\\

\bo{Cloud computing} \\
\bo{Id:} C1QN-1.11 \\
\bo{Tipo:} qualit\`a \\
\bo{Richiesta:} obbligatorio \\
\bo{Fonte:} capitolato \\
Sar\`a utilizzato il supporto al cloud computing fornito da Google AppEngine,
anche per quanto riguarda il database grazie a Google DataStore.
\\
\\

\bo{Utilizzo} \\
\bo{Id:} C1QN-3 \\
\bo{Tipo:} qualit\`a \\
\bo{Richiesta:} obbligatorio \\
\bo{Fonte:} interna \\
Un occhio di riguardo sar\`a dato alla semplicit\`a di utilizzo ed alla possibilit\`a
per tutti i potenziali utenti di imparare ad utulizzare NetMus. L'utente
deve poter essere subito in grado di visualizzare e gestire il proprio catalogo
musicale dopo essersi registrato ed aver messo in funzione la C2.
\\
\\

\bo{Accessibilit\`a} \\
\bo{Id:} C1QO-3.1 \\
\bo{Tipo:} qualit\`a \\
\bo{Richiesta:} opzionale \\
\bo{Fonte:} interna \\
Il software Web presenter\`a alcune caratteristiche per favorire l'accesso a
qualsiasi tipoligia di utenza come ad esempio il codice sorgente validato
secondo paramentri W3C.
\\
\\

\bo{Portabilit\`a} \\
\bo{Id:} C1QN-3.3 \\
\bo{Tipo:} qualit\`a \\
\bo{Richiesta:} obbligatorio \\
\bo{Fonte:} interna \\
Deve essere compatibile col maggior numero di configurazioni software e
hardware. \\ \\

\bo{Supporto multi-lingua} \\
\bo{Id:} C1QD-3.4 \\
\bo{Tipo:} qualit\`a \\
\bo{Richiesta:} desiderabile \\
\bo{Fonte:} capitolato \\
Saranno accessibili due versioni dell'applicazione, una in italiano ed una in
lingua inglese, con gli stessi contenuti e sar\`a possibile spostarsi da una
all'altra in qualsiasi momento.
\\
\\

\bo{Manutenibilit\`a} \\
\bo{Id:} C1QN-3.6 \\
\bo{Tipo:} qualit\`a \\
\bo{Richiesta:} obbligatorio \\
\bo{Fonte:} interna \\
Saranno seguiti dei criteri di scrittura del codice e stesura dei documenti che
favoriranno il pi\`u possibile il processo di manutenzione.
\\
\\

\bo{Gestione errori} \\
\bo{Id:} C1QN-3.7 \\
\bo{Tipo:} qualit\`a \\
\bo{Richiesta:} obbligatorio \\
\bo{Fonte:} interna \\
Il sistema sar\`a robusto relativamente agli errori causati all'utilizzo della
rete internet, risaputamente inaffidabile.
\\
\\

\bo{Manuale utente} \\
\bo{Id:} C1QN-4.1 \\
\bo{Tipo:} qualit\`a \\
\bo{Richiesta:} obbligatorio \\
\bo{Fonte:} capitolato \\
Sar\`a redatto un manuale utente che conterr\`a tutte e sole le informazioni
necessarie all'utilizzo di NetMus in modo da essere facilmente comprensibile da
tutta l'enorme fascia di utenza a cui si rivolge.
\\
\\

\bo{Manuale utente inglese} \\
\bo{Id:} C1QD-4.1.1 \\
\bo{Tipo:} qualit\`a \\
\bo{Richiesta:} desiderabile \\
\bo{Fonte:} capitolato \\
Sar\`a fornita anche una traduzione del manuale utente in lingua inglese. \\ \\

\subsection{Requisiti di vincolo}

\subsubsection{Interfacciamento con gli ambienti di installazione e d'uso }

\bo{Tecnologie GAE e GWT} \\
\bo{Id:} C1VN-1.12 \\
\bo{Tipo:} vincolo \\
\bo{Richiesta:} obbligatorio \\
\bo{Fonte:} capitolato \\
Per lo sviluppo dell'applicazione Web NetMus sar\`a utlizzata la piattaforma
Google AppEngine (GAE) in concomitanza con l'insieme di strumenti forniti da
Google Web Toolkit (GWT). Entrambi sono gratuiti.
\\
\\

\bo{Google DataStore} \\
\bo{Id:} C1VN-1.13.1 \\
\bo{Tipo:} vincolo \\
\bo{Richiesta:} obbligatorio \\
\bo{Fonte:} capitolato \\
Il database verr\`a sviluppato utilizzando Google DataStore. \\ \\


\subsubsection{Norme vigenti nel dominio applicativo}

\bo{Quote YouTube} \\
\bo{Id:} C1VD-1.5.2 \\
\bo{Tipo:} vincolo \\
\bo{Richiesta:} desiderabile \\
\bo{Fonte:} interna \\
NetMus rispetter\`a le qute di traffico imposte da Youtube per quanto riguarda il
traffico in entrata ed in uscita dai loro server.
\\
\\

\bo{YouTube Terms of Service} \\
\bo{Id:} C1VD-1.5.3 \\
\bo{Tipo:} vincolo \\
\bo{Richiesta:} desiderabile \\
\bo{Fonte:} interna \\
NetMus rispetter\`a le regole stabilite da YouTube per quanto riguarda i\\
propri contenuti (\emph{http://www.youtube.com/t/terms}) e per quanto
riguarda l'utilizzo delle YouTube API
(\emph{http://code.google.com/apis/youtube/terms.html}).
\\
\\

\bo{Open source} \\
\bo{Id:} C1VN-3.2 \\
\bo{Tipo:} vincolo \\
\bo{Richiesta:} obbligatorio \\
\bo{Fonte:} capitolato \\
La licenza d'uso dell'applicazione sar\`a di tipo Open source. \\ \\

\subsubsection{Caratteristiche dell'utente}

\bo{Semplicit\`a di utilizzo} \\
\bo{Id:} C1VN-3.5 \\
\bo{Tipo:} vincolo \\
\bo{Richiesta:} obbligatorio \\
\bo{Fonte:} interna \\
Tutte le componenti dell'applicazione Web NetMus saranno mirate alla semplicit\`a
di utilizzo da parte dell'utente. A partire dall'interfaccia grafica sar\`a tutto
molto intuitivo.
\\
\\

\section{Componente di recupero delle informazioni (C2)}
\subsection{Requisiti funzionali}
C2FN-1-Componente per il recupero automatico delle informazioni (capitolato).

C2FN-1.1-Recupero delle informazioni dei file musicali.

C2FN-1.1.1-Recupero automatico all'inserimento di un nuovo dispositivo.

C2FN-1.1.2-Recupero manuale (a comando dell'utente).

C2FN-1.1.3-Raccolta delle informazioni anche con connessione non disponibile.

C2FN-1.1.4-Raccolta delle informazioni anche dall'hard disk (cartelle indicate
dall'utente).

C2FN-1.1.5-Vanno ignorati i file musicali con assenza di informazioni.

C2FO-1.1.5.1-I file ignorati vengono indicati all'utente.

C2FD-1.2-Aggiornamento e completamento delle informazioni dei file musicali
dell'utente.

C2FD-1.2.1-Sotto esplicita richiesta dell'utente, aggiornamento e completamento
delle informazioni dei file musicali con recupero dei dati da C1.

C2FN-1.3-Comunicazione con C1.

C2FN-1.3.1-Invio delle informazioni dei brani a C1.

\subsection{Requisiti di qualit\`a}
C2FN-1.1.3.1-Utilizzo di poco spazio per la memorizzazione temporanea
delle informazioni.

C2FN-1.3.2-Comunicazioni solo quando strettamente necessarie.

C2QN-1.4-Utilizzo.

C2QD-1.4.1-Portabilit\`a.

C2QN-1.4.2-Semplicit\`a di utilizzo.

C2QD-1.4.3-Supporto multi-lingua.

\subsection{Requisiti di vincolo}
\subsubsection{Interfacciamento con l'utente}
C2QN-1.4.4-Arreca meno disturbo possibile all'utente.

C2QN-1.4.5-Richiede l'autorizzazione per la raccolta delle informazioni.

\end{document}
