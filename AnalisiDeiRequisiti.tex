
\newcommand{\nomedoc}{Analisi dei requisiti}
\newcommand{\versione}{0.2}
\newcommand{\versioneglossario}{0.1}
\newcommand{\versionenormeprogetto}{0.3}
\newcommand{\nomefile}{AnalisiDeiRequisiti-\versione.pdf}
\newcommand{\datacreazione}{2 Dicembre 2010}
\newcommand{\datamodifica}{9 Dicembre 2010}
\newcommand{\stato}{formale}
\newcommand{\uso}{interno}
\newcommand{\redazione}{Mandolo Andrea}
\newcommand{\verifica}{Baffo}
\newcommand{\approvazione}{Valter}
\newcommand{\distribuzione}{
VT.G \\
& Prof. Vardanega Tullio\\
& Prof. Cardin Riccardo }

% FUNZIONI TIPOGRAFICHE
\newcommand{\co}{\texttt} % courier
\newcommand{\bo}{\textbf} % bold
\newcommand{\pr}{\par\medskip} % paragrafo spaziato
\newcommand{\sca}{\textsc} % small caps

\documentclass[a4paper,12pt]{report}
% 10pt,11pt,12pt
% titlepage, notitlepage -> per dare inizio o no ad una nuova pagina dopo titolo
% twoside -> per dire se fronte-retro
\usepackage[latin1]{inputenc}
% per caratteri accentati
\usepackage[italian]{babel}
% per regole sintattiche italiane
\usepackage[bookmarks=true, pdfborder={0 0 0 0}]{hyperref}
% per collegamenti ipertestuali
\usepackage{graphicx}
% per inserimento immagini

% \usepackage{enumerate}
% per personalizzare elenchi puntati

\usepackage[hmargin=2cm]{geometry} %margine 2 cm
%\geometry{options varie}

% comandi per gestire meglio header e footer
\usepackage{fancyhdr}  % header e footer
\usepackage{totpages}
\pagestyle{fancy}
\renewcommand{\headrulewidth}{0.4pt}
\renewcommand{\footrulewidth}{0.4pt}

\setlength{\headheight}{1.2cm} % NON TOCCARE
\setlength{\voffset}{-1.5cm} % NON TOCCARE
\setlength{\textheight}{700pt} % NON TOCCARE
\setlength{\parindent}{0pt} % INDENTAZIONE

\lhead{\nomedoc\  (ver. \versione)}
\chead{}
\rhead{\includegraphics[height=1cm]{img/netmus.png}}
\lfoot{\includegraphics[height=0.8cm]{img/logo.png}}
\cfoot{}
\rfoot{\thepage}

% \usepackage{listings}   per codice sorgente

\author{VT.G - Valter Texas Group}

\begin{document}

\pagenumbering{Roman} % INIZIO NUMERAZIONE ARABA

\vspace*{1cm}
\begin{center}

\includegraphics[width=9cm]{img/logo.png}\\
\vspace{0.5cm}
\begin{LARGE} \sca{VT.G - Valter Texas Group} \end{LARGE}\\
\vspace{0.5cm}
\begin{Large}
\emph{valtertexasgroup@googlegroups.com} \end{Large}\\
\vspace*{1cm} \includegraphics[width=5cm]{img/netmus.png}\\
\vspace{0.5cm}
\begin{Large} \sca{\nomedoc} \end{Large}\\
\vspace{1cm}
\begin{Large} \emph{Ingegneria del Software A.A. 2010-2011} \end{Large}\\
\end{center}
\vspace{1cm}

% INFORMAZIONI DOCUMENTO
\begin{center}
\begin{tabular}{r|l}
\hline & \\
\bo{Nome} & \nomefile \\
\bo{Versione attuale} & \versione \\
\bo{Data creazione} & \datacreazione \\
\bo{Data ultima modifica} & \datamodifica \\
\bo{Stato} & \stato \\
\bo{Uso} & \uso \\
\bo{Redazione} & \redazione \\
\bo{Verifica} & \verifica \\
\bo{Approvazione} & \approvazione \\
\bo{Distribuzione} & \distribuzione \\
& \\\hline
\end{tabular}
\end{center}
\newpage

% REGISTRO MODIFICHE
\section*{Registro delle modifiche}
\begin{tabular}{lll}
% REVISIONI DEL DOCUMENTO
% DALLA PIU' NUOVA ALLA PIU' VECCHIA
\bo{Data:} 09/12/2010 &
\bo{Versione:} 0.2 &
\bo{Autore:} Daminato Simone\\
\hline\\
\multicolumn{3}{p{470px}}{ Redazione del capitolo 2.}\\
\\

\bo{Data:} 07/12/2010 &
\bo{Versione:} 0.1 &
\bo{Autore:} Lovato Daniele\\
\hline\\
\multicolumn{3}{p{470px}}{ Redazione del capitolo 1.}\\
\\

\end{tabular}

% INDICE
\tableofcontents
\thispagestyle{fancy} % per lo stile di header e footer


\chapter*{Sommario}


\thispagestyle{fancy} % serve perche' nelle pagine di inizio Chapter esca header e footer
\pagenumbering{arabic} % INIZIO NUMERAZIONE NORMALE
\rfoot{\thepage\ di \pageref{TotPages}}
\addcontentsline{toc}{chapter}{Sommario}

\chapter{Introduzione}
\thispagestyle{fancy} % serve perche' nelle pagine di inizio Chapter esca header e footer

\section{Scopo del documento}
Il presente documento ha lo scopo di presentare al committente l'Analisi dei
Requisiti e le potenzialit\`a software del nostro prodotto per soddisfare le
richieste del capitolato d'appalto 02 NetMus.


\section{Scopo del prodotto}

\section{Glossario}
Il Glossario \`e definito con un documento a parte (\emph{Glossario.pdf}). Tutti
i termini caratterizzati da \underline{questa sottolineatura} sono ivi
definiti.\\ Verr\`a sottolineata solamente la prima occorrenza di ciascun
termine presente nel Glossario, per non compromettere la leggibilit\`a del documento.

\section{Riferimenti}

\subsection{Normativi} % oppure rif. a Norme di progetto con leggi e tutto
\begin{itemize}
  \item ISO/IEC 12207:1995 - Cicli di vita software
  \item ISO/IEC 9126:2001 - Quality Model
\end{itemize}

\subsection{Informativi}
\begin{itemize}
  \item Capitolato d'appalto CO2-NETMUS del corso di Ingegneria del Software
  A.A. 2010/11 :\\
  \url{http://www.math.unipd.it/~tullio/..}
  \item Slide delle lezioni del corso :\\
  \url{http://www.math.unipd.it/~tullio/.d.}
\end{itemize}


% INIZIO CAPITOLO 2
\chapter{Descrizione generale}
\thispagestyle{fancy}
\section{Contesto d'uso del prodotto}
Il progetto Netmus che abbiamo intenzione di sviluppare \`e un software
destinato ad una vasta gamma d'utenza, compresa nella fascia d'et\`a dai 12 ai
100 anni, che interesser\`a per\`o soprattutto adolescenti e adulti dai 15 ai 35 anni.
\subsection{Piattaforma d'esecuzione e interfacciamento con l'ambiente di
installazione e uso}
Il prodotto finale utilizzer\`a diverse teconologie (in particolare: Java, GAE e
GWT) che renderanno possibile l'utilizzo del sistema su pi\`u piattaforme
d'esecuzione con sistemi operativi diversi, i cui unici due requisiti
fondamentali saranno di possedere una connessione ad internet e di aver
installata la Java Runtime Machine. In particolare, il prodotto finale sar\`a
testato sui principali sistemi operativi: Windows (xp e 7), Linux (principali e
pi\`u diffuse distribuzioni), Mac OS X (Snow Leopard).
\section{Funzioni del prodotto}
Il prodotto finale sar\`a composto da due componenti. Una componente di
persistenza e visualizzazione della libreria virtuale dell'utente. Un'altra
componente di recupero delle informazioni dei brani musicali contenuti nei
sistemi di riproduzione personale dell'utente. La prima componente avr\`a il
compito di gestire la memorizzazione delle informazioni dei brani nel database,
e la sua visualizzazione e gestione con un'interfaccia semplice e facilmente
utilizzabile. La seconda componente svolger\`a il ruolo di estrazione delle
informazioni dai brani musicali, e le invier\`a al server senza interferire con
l'utente che intanto potr\`a continuare a svolgere il proprio lavoro indisturbato.
L'estrazione avverr\`a non appena verr\`a rilevato un nuovo dispositivo di
archiviazione di massa (storage USB o lettore mp3) che non cripta i propri file,
e in modo del tutto autonomo verranno ricavate le informazioni, memorizzate in
locale in caso di assenza di connessione, e poi successivamente inviate al server.
\section{Caratteristiche degli utenti}
L'utente tipo che utilizza il sistema non \`e in possesso di particolari
conoscenze informatiche, ma \`e in grado di navigare sul web e di eseguire alcune
semplici operazioni correlate (come per esempio effettuare un login).
\section{Vincoli generali}
Il software nel suo funzionamento e sviluppo:
\begin{itemize}
  \item sar\`a completo della documentazione necessaria e del manuale d'uso,
  \item sar\`a portabile (verr\`a testato e garantito il suo funzionamento sui
  principali sistemi operativi),
  \item sar\`a sicuro per gli utenti: garantir\`a il controllo del proprio account e
  dei propri dati, e della propria privacy,
  \item per la raccolta dei dati dei brani localmente non sar\`a necessario
  l'accesso ad internet,
  \item per l'aggiornamento e la visualizzazione dei propri dati online sar\`a
  necessaria la connessione ad internet,
  \item il manuale utente sar\`a fornito sia in italiano che in inglese.
\end{itemize}
\section{Assunzioni e dipendenze}
Si assume che sui personal computer utilizzati dagli utenti sia gi\`a installata
una JVM.
\section{Glossario}
Esister\`a un unico glossario comune a tutti i documenti, contenente la
spiegazione di tutti i termini e gli acronimi che verranno utilizzati
all'interno della documentazione. Questo documento sar\`a organizzato in ordine
alfabetico in modo da permettere una rapida ricerca dei termini e sar\`a suddiviso
due categorie: termini e acronimi.

\chapter{Lista dei requisiti}
\thispagestyle{fancy}
Ad ogni requisito � stato assegnato un codice di identificazione per facilitarne
il tracciamento, per maggiori informazioni per quanto riguarda la notazione
utilizzata per catalogare i requisiti si veda il documento
\emph{NormeDiProgetto-1.0.pdf}.

\section{Componente di persistenza e visualizzazione (Componente 1)}
\subsection{Requisiti funzionali}
C1FN-1-WEB Application NetMus (capitolato)

C1FN-1.1-Grafica simile ad iTunes (capitolato)

C1FN-1.1.1-Elenco brani raggruppati opportunamente

C1FN-1.1.2-Menu laterali

C1FN-1.1.3-Visualizzare informazioni dettagliate di un brano

C1FD-1.1.4-Visualizza player YouTube

C1FO-1.1.5-Visualizzazione copertina disco

C1FN-1.2-Creazione profilo utente tramite registrazione (capitolato)

C1FO-1.2.1-Pagina di login indipendente da google login

C1FD-1.3-Personalizzazione del proprio catalogo (interna)

C1FD-1.3.1-Cancellazione brano

C1FD-1.3.2-Modifica info brano

C1FO-1.3.3-Creazione playlist

C1FO-1.3.4-Ranking brani

C1FN-1.4-Gestione profilo personale (interna)

C1FN-1.4.1-Modifica informazioni personali

C1FN-1.4.2-Cambio password

C1FN-1.4.3-Cancellazione del proprio account

C1FD-1.4.4-Aggiungi infromazioni personali

C1FD-1.4.5-Pubblicazione profilo

C1FD-1.5-Riproduzione tracce in streaming (capitolato)

C1FD-1.7-Interazione con altri utenti (interna)

C1FD-1.7.1-Visualizzazione altri profili

C1FO-1.7.2-Lasciare commenti su altri profili

C1FD-1.8-Elaborazione dati utente (statistiche, brani in comune ecc ecc)

C1FN-1.9-Ricezione ed elaborazione brani

C1FN-1.9.1-Controllo di validit� dei dati

C1FN-1.9.2-Completamento info da database interno

C1FN-1.9.4-Identificazione dati ridondanti

C1FN-1.9.5-Inserimento nel database

C1FO-1.9.3-Completamento info da servizio esterno

C1FD-1.10-Invio nuove informazioni a componente 2

C1FN-1.13-Gestione database

\subsection{Requisiti di qualit\`a}
C1QD-1.5.1-Ottimizzazione della ricerca su YoutTube

C1QN-1.6-Scalabilit� (capitolato)

C1QN-1.6.1-Scalabilit� interfaccia grafica

C1QN-1.6.2-Scalabilit� massa di utenza

C1QN-1.11-Deve utilizzare il cloud computing

C1QN-3-Utilizzo

C1QN-3.1-Accessibilit�

C1QN-3.3-Portabilit�

C1QD-3.4-Supporto multi-lingua

C1QN-3.5-Semplicit� di utilizzo

C1QN-3.6-Manutenibilit�

C1QN-3.7-Gestione delle eccezioni

C1QN-4.1-Manuale utente

\subsection{Requisiti di vincolo}

\subsubsection{Interfacciamento con gli ambienti di installazione e d'uso }
C1VN-1.12-Deve utilizzare le tecnologie GAE e GWT

C1VN-1.13.1-Deve utilizzare Google DataStore


\subsubsection{Norme vigenti nel dominio applicativo}
C1VD-1.5.2-Quote YouTube

C1VD-1.5.3-YouTube Terms of Service

C1VN-3.2-Open source


\subsubsection{Caratteristiche dell'utente}


\section{Componente di recupero delle informazioni (Componente 2)}
\subsection{Requisiti funzionali}
c2FN-1-Componente per il recupero automatico delle informazioni (capitolato).

C2FN-1.1-
\begin{itemize}
  \item Dovrebbe raccogliere i dati anche se non � disponibile una connessione.
  \item Deve richiedere l'autorizzazione dell'utente per la raccolta delle
  informazioni.
  \item Deve arrecare meno disturbo possibile all'utente.
  \item Dovrebbe essere portabile.
  \item Potrebbe permettere all'utente di aggiornare e/o completare le
  informazioni relative alle proprie canzoni attingendo alle informazioni della
  componente persistente.
  \item Deve essere semplice da utilizzare.
  \item Deve reperire automaticamente tutte le informazioni delle canzoni
  all'inserimento di un dispositivo.
  \item Dovrebbe permettere di reperire le informazioni anche dalle cartelle
  dell'hard disk indicate dell'utente.
  \item Deve permettere di effettuare anche una scansione manuale.
  \item Potrebbe essere tradotto in pi� lingue (preferibilmente italiano e
  inglese).
  \item Deve impedire che vengano caricati brani di cui non si riescono a
  reperire le informazioni.
  \item Dovrebbe avvisare l'utente di eventuali brani i cui dati non vengono
  caricati.
  \item Deve evitare di usare troppa memoria per memorizzare temporaneamente le
  informazioni.
  \item Deve evitare comunicazioni inutili col server.
\end{itemize}
\end{document}
