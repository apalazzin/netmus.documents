
\newcommand{\nomedoc}{Studio Di Fattibilit\`a}
\newcommand{\versione}{1.2}
\newcommand{\versioneglossario}{1.0}
\newcommand{\versionenormeprogetto}{1.0}
\newcommand{\nomefile}{StudioDiFattibilita-\versione.pdf}
\newcommand{\datacreazione}{7 Dicembre 2010}
\newcommand{\datamodifica}{21 Gennaio 2011}
\newcommand{\stato}{formale}
\newcommand{\uso}{interno}
\newcommand{\redazione}{Palazzin Alberto}
\newcommand{\verifica}{Mandolo Andrea}
\newcommand{\approvazione}{Lovato Daniele}
\newcommand{\distribuzione}{
VT.G \\
& Prof. Vardanega Tullio\\
& Prof. Cardin Riccardo }

% FUNZIONI TIPOGRAFICHE
\newcommand{\co}{\texttt} % courier
\newcommand{\bo}{\textbf} % bold
\newcommand{\pr}{\par\medskip} % paragrafo spaziato
\newcommand{\sca}{\textsc} % small caps

\documentclass[a4paper,12pt]{report}
% 10pt,11pt,12pt
% titlepage, notitlepage -> per dare inizio o no ad una nuova pagina dopo titolo
% twoside -> per dire se fronte-retro
\usepackage[latin1]{inputenc}
% per caratteri accentati
\usepackage[italian]{babel}
% per regole sintattiche italiane
\usepackage[bookmarks=true, pdfborder={0 0 0 0}]{hyperref}
% per collegamenti ipertestuali
\usepackage{graphicx}
% per inserimento immagini

% \usepackage{enumerate}
% per personalizzare elenchi puntati

\usepackage[hmargin=2cm]{geometry} %margine 2 cm
%\geometry{options varie}

% comandi per gestire meglio header e footer
\usepackage{fancyhdr}  % header e footer
\usepackage{totpages}
\pagestyle{fancy}
\renewcommand{\headrulewidth}{0.4pt}
\renewcommand{\footrulewidth}{0.4pt}

\setlength{\headheight}{1.2cm} % NON TOCCARE
\setlength{\voffset}{-1.5cm} % NON TOCCARE
\setlength{\textheight}{700pt} % NON TOCCARE
\setlength{\parindent}{0pt} % INDENTAZIONE

\lhead{\nomedoc\  (ver. \versione)}
\chead{}
\rhead{\includegraphics[height=1cm]{img/netmus.png}}
\lfoot{\includegraphics[height=0.8cm]{img/logo.png}}
\cfoot{}
\rfoot{\thepage}

% \usepackage{listings}   per codice sorgente

\author{VT.G - Valter Texas Group}

\begin{document}

\pagenumbering{Roman} % INIZIO NUMERAZIONE ARABA

\vspace*{1cm}
\begin{center}

\includegraphics[width=9cm]{img/logo.png}\\
\vspace{0.5cm}
\begin{LARGE} \sca{VT.G - Valter Texas Group} \end{LARGE}\\
\vspace{0.5cm}
\begin{Large}
\emph{valtertexasgroup@googlegroups.com} \end{Large}\\
\vspace*{1cm} \includegraphics[width=5cm]{img/netmus.png}\\
\vspace{0.5cm}
\begin{Large} \sca{\nomedoc} \end{Large}\\
\vspace{1cm}
\begin{Large} \emph{Ingegneria del Software A.A. 2010-2011} \end{Large}\\
\end{center}
\vspace{1cm}

% INFORMAZIONI DOCUMENTO
\begin{center}
\begin{tabular}{r|l}
\hline & \\
\bo{Nome} & \nomefile \\
\bo{Versione attuale} & \versione \\
\bo{Data creazione} & \datacreazione \\
\bo{Data ultima modifica} & \datamodifica \\
\bo{Stato} & \stato \\
\bo{Uso} & \uso \\
\bo{Redazione} & \redazione \\
\bo{Verifica} & \verifica \\
\bo{Approvazione} & \approvazione \\
\bo{Distribuzione} & \distribuzione \\
& \\\hline
\end{tabular}
\end{center}
\newpage

\chapter*{Sommario}
\thispagestyle{fancy} % serve perche' nelle pagine di inizio Chapter esca headere footer
Dopo aver letto, discusso ed analizzato i vari capitolati d'appalto, \`e
stato stilato il documento dello Studio Di Fattibilit\`a per riassumere e
documentare le valutazioni effettuate dal gruppo. \`E stato infine selezionato
il capitolato C02 NetMus come futuro progetto da sviluppare, ritenendolo
stimolante e formativo dal punto di vista tecnologico.

\newpage
% REGISTRO MODIFICHE
\section*{Registro delle modifiche}

\begin{longtable}{|p{0.13\textwidth}|c|p{0.2\textwidth}|p{0.46\textwidth}|}
\hline
\rowcolor{orange} \bo{Data} & \bo{Versione} & \bo{Autore} & \bo{Descrizione} \\
\hline
\endhead
\hline
\endfoot
21/01/2011 & 1.2 & Mandolo Andrea & Corretti Riferimenti.\\
\hline
12/01/2011 & 1.1 & Mandolo Andrea & Modificato layout Registro delle
modifiche.\\
\hline
19/12/2010 & 1.0 & Lovato Daniele & Validazione per consegna RR.\\
\hline
16/12/2010 & 0.5 & Mandolo Andrea & Verificato l'intero documento.\\
\hline
10/12/2010 & 0.4 & Mandolo Andrea & Correzione errori grammaticali.\\
\hline
09/12/2010 & 0.3 & Palazzin Alberto & Aggiunta Sezione "Componenti del Progetto
e Dettagli Tecnologie".\\
\hline
09/12/2010 & 0.2 & Palazzin Alberto & Aggiunta sottolineature per il
Glossario.\\
\hline
07/12/2010 & 0.1 & Palazzin Alberto & Stesura prima versione del documento.
\end{longtable}

% INDICE
\tableofcontents

\chapter{Introduzione}
\thispagestyle{fancy} % serve perche' nelle pagine di inizio Chapter esca header e footer
\pagenumbering{arabic} % INIZIO NUMERAZIONE NORMALE
\rfoot{\thepage\ di \pageref{TotPages}}

\section{Scopo del documento}
Il presente documento ha lo scopo di valutare la fattibilit\`a di progettazione
e sviluppo di ogni capitolato d'appalto rispetto alle tecnologie e alle
tempistiche richieste. Nello specifico vengono descritte le motivazioni per le
quali il gruppo ha scelto il capitolato C02 piuttosto che altri.


\section{Scopo del prodotto}

\section{Glossario}
Il Glossario \`e definito con un documento a parte (\emph{Glossario.pdf}). Tutti
i termini caratterizzati da \underline{questa sottolineatura} sono ivi
definiti.\\ Verr\`a sottolineata solamente la prima occorrenza di ciascun
termine presente nel Glossario, per non compromettere la leggibilit\`a del documento.

\section{Riferimenti}

\subsection{Normativi} % oppure rif. a Norme di progetto con leggi e tutto
\begin{itemize}
  \item ISO/IEC 12207:1995 - Cicli di vita software
  \item ISO/IEC 9126:2001 - Quality Model
\end{itemize}

\subsection{Informativi}
\begin{itemize}
  \item Capitolato d'appalto CO2-NETMUS del corso di Ingegneria del Software
  A.A. 2010/11 :\\
  \url{http://www.math.unipd.it/~tullio/..}
  \item Slide delle lezioni del corso :\\
  \url{http://www.math.unipd.it/~tullio/.d.}
\end{itemize}


% INIZIO CAPITOLO 2

\chapter{Capitolato C01 - MindSlide}
\thispagestyle{fancy}
Il capitolato C01 MindSlide (Software di presentazione basato su \underline{mind
mapping}) proposto dall'azienda Zucchetti richiede l'utilizzo del linguaggio di
markup \underline{HTML5}.
Il progetto risulta interessante considerando l'utilit\`a del prodotto
risultante, ma dopo un'attenta lettura della specifica ed un confronto con gli
altri capitolati, si ritiene che la tecnologia richiesta non sia abbastanza
stimolante per il gruppo.

\chapter{Capitolato C02 - Netmus}
\thispagestyle{fancy}
Il capitolato C02 NetMus (libreria musicale personale, virtuale e distribuita),
proposto dal Prof. Cardin Riccardo, richiede lo sviluppo di una
\underline{applizazione web} che utilizzi tecnologie all'avanguardia, come il
sistema cloud \underline{Google App Engine} e \underline{JavaFX}, e che vada ad
interagire con sistemi di audio/video streaming tipo \underline{YouTube}.\\

Dopo un'accurata lettura e un'intensa discussione si \`e deciso di accettare
questo progetto.

\section{Studio del dominio} 
Le conoscenze e le capacit\`a richieste per lo sviluppo del prodotto
riguardano i seguenti ambiti:
\begin{itemize}
  \item conoscenze sul funzionamento della rete \underline{internet};
  \item capacit\`a d'astrazione sull'idea di \underline{cloud computing};
  \item buone capacit\`a di gestione della concorrenza e della distribuzione;
  \item buone capacit\`a di programmazione in \underline{Java} per l'utilizzo di
  GAE e \underline{JavaFX};
  \item buone conoscenze sulle basi di dati, per usare il database
  non relazionale \underline{Google DataStore};
  \item conoscenze riguardo vari \underline{sistemi operativi}
  (\underline{Windows}, \underline{Linux}, \underline{MacOsx}) per poter interagire con periferiche
  audio e di archiviazione dati.
\end{itemize}

\section{Fattibilit\`a tecnologica e tempistica}
Considerando la preparazione dei membri del gruppo, data dai corsi universitari
della laurea in Informatica dell'Universit\`a di Padova, si dovrebbe avere la
capacit\`a di utilizzare in maniera adeguata la programmazione orientata agli
oggetti tramite linguaggio Java. Invece per il sistema Google App Engine si
potr\`a usufruire dell'eventuale corso di formazione, forse sostenuto
dall'azienda Miriade S.p.A. committente del capitolato C03.
Per implementare lo strato di presentazione si \`e orientati ad utilizzare
\underline{Google Web Toolkit}. Per interfacciarsi con i dispositivi di
archiviazione si pensava di creare una \underline{RIA} utilizzando JavaFX, per
rendere pi\`u gradevole l'esperienza degli utenti. Formarsi su tali tecnologie con manuali e
documentazioni appropriate, non dovrebbe essere un grosso problema. Dopo aver
analizzato il livello di conoscenza, abilit\`a, capacit\`a di apprendimento
all'interno del gruppo e valutando inoltre eventuali impegni dei singoli membri,
si prevede di riuscire a sviluppare il prodotto nei tempi prestabiliti
rispettando le tempistiche per le varie revisioni.

\section{Componenti del progetto e dettagli sulle tecnologie}
Il problema che viene posto nel capitolato C02 presenta due principali componenti
da considerare per lo studio della fattibilit\`a.\\
Una componente che gestisce la memorizzazione e visualizzazione della libreria
virtuale personale dell'utente.\\
L'altra componente gestisce il recupero delle informazioni dei brani
musicali dal dispositivo di riproduzione personale dell'utente.

\subsection{Componente di memorizzazione e gestione della libreria}
La prima componente dovr\`a essere accessibile all'utente come
applicazione web. Il sistema consigliato per lo sviluppo, ossia Google App Engine, 
\`e una tecnologia cloud computing che permette l'utilizzo di risorse hardware o
 software distribuite in remoto.\\
 User\`a inoltre il database distribuito
 Google DataStore. Questo database utilizza una linguaggio simile
 all'\underline{SQL} chiamato \underline{GQL}.

\subsection{Componente di estrazione delle informazioni}
Per sviluppare la seconda componente si \`e pensato di ricorrere alla creazione
di una Rich Internet Application, ossia un'applicazione web che possiede le
caratteristiche e le funzionalit\`a delle \underline{applicazioni desktop}, senza
per\`o necessitare dell'installazione sul \underline{disco fisso}. Questa scelta
\`e stata dettata dal fatto che le RIA si caratterizzano per la dimensione
interattiva, la multimedialit\`a e la velocit\`a d'esecuzione. La parte
dell'applicazione che elabora i dati \`e trasferita a livello client e fornisce
una pronta risposta all'interfaccia utente, mentre la gran parte dei dati e
dell'applicazione rimane sul server remoto, con notevole alleggerimento per il
computer utente. Queste caratteristiche rendono l'applicazione il meno invasiva
possibile per l'utente.

\section{Rischi}
L'analisi dettagliata dei capitolati, illustrata in questo Studio Di
Fattibilit\`a dovrebbe aver diminuito le variabili di rischio. Tuttavia la
nostra mancanza di esperienza o imprevisti inattesi come la mancanza di
risorse umane potrebbero provocare ritardi o addirittura il non completamento
del prodotto.\\

Verranno studiati in maniera pi\`u approfondita i rischi derivanti da
\underline{propriet\`a emergenti} del sistema e le possibili soluzioni
nell'analisi dei rischi interna al Piano Di Progetto.

\chapter{Capitolato C03 - RescueMe}
\thispagestyle{fancy}
Il capitolato C03 RescueMe (sistema globale per gestione catastrofi naturali)
proposto dall'azienda Miriade S.p.A. richiede l'utilizzo delle tecnologie
Google App Engine, Google Web Toolkit e \underline{Spring MVC}. Dopo un'attenta
analisi si ritiene che, anche se gli strumenti da utilizzare siano di
interesse comune al gruppo, lo sviluppo completo del progetto possa richiedere
pi\`u tempo di quello che abbiamo a disposizione, col rischio di non rispettare
i tempi di consegna prefissati.

\chapter{Capitolato C04 - QWAD}
\thispagestyle{fancy}
Il capitolato C04 QWAD (Quick Web Apps Designer) proposto dalla \underline{Onlus}
Informatici Senza Frontiere richiede l'evoluzione di un lavoro gi\`a
svolto da altri team di sviluppo in un web app designer che
verr\`a utilizzato per creare applicazioni di quel tipo da persone non esperte
di programmazione.\\

Scegliere questo capitolato \`e visto da noi come un rischio elevavo
d'insuccesso, poich\'e si prevede che lo studio del database di considerevoli
dimensioni che sta alla base del programma e l'implementazione dell'app
designer in cloud, richieda tempistiche maggiori di quelle che il
nostro gruppo ha a disposizione. Inoltre abbiamo discusso sul fatto
che potrebbe non stimolare in maniera appropriata i membri del gruppo, in quanto
i risultati diverrebbero visibili in un maggior arco di tempo, rischiando di
compromettere cos\`\i\ la produttivit\`a ed il coinvolgimento generale. Pur
apprezzando il contributo sociale apportato da tale progetto, si \`e comunque
deciso di scartarlo.

\end{document}