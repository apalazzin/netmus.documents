
\newcommand{\nomedoc}{StudioDiFattibilita}
\newcommand{\versione}{0.1}
\newcommand{\nomefile}{StudioDiFattibilita\versione.pdf}
\newcommand{\datacreazione}{7 Dicembre 2010}
\newcommand{\datamodifica}{7 Dicembre 2010}
\newcommand{\stato}{formale}
\newcommand{\uso}{interno}
\newcommand{\redazione}{Palazzin Alberto}
\newcommand{\verifica}{Lovato Daniele}
\newcommand{\approvazione}{Valter}
\newcommand{\distribuzione}{
VT.G \\
& Prof. Vardanega Tullio }

% FUNZIONI TIPOGRAFICHE
\newcommand{\co}{\texttt} % courier
\newcommand{\bo}{\textbf} % bold
\newcommand{\pr}{\par\medskip} % paragrafo spaziato
\newcommand{\sca}{\textsc} % small caps

\documentclass[a4paper,12pt]{report}
% 10pt,11pt,12pt
% titlepage, notitlepage -> per dare inizio o no ad una nuova pagina dopo titolo
% twoside -> per dire se fronte-retro
\usepackage[latin1]{inputenc}
% per caratteri accentati
\usepackage[italian]{babel}
% per regole sintattiche italiane
\usepackage[bookmarks=true, pdfborder={0 0 0 0}]{hyperref}
% per collegamenti ipertestuali
\usepackage{graphicx}
% per inserimento immagini

% \usepackage{enumerate}
% per personalizzare elenchi puntati

\usepackage[hmargin=2cm]{geometry} %margine 2 cm
%\geometry{options varie}

% comandi per gestire meglio header e footer
\usepackage{fancyhdr}  % header e footer
\usepackage{totpages}
\pagestyle{fancy}
\renewcommand{\headrulewidth}{0.4pt}
\renewcommand{\footrulewidth}{0.4pt}

\setlength{\headheight}{1.2cm} % NON TOCCARE
\setlength{\voffset}{-1.5cm} % NON TOCCARE
\setlength{\textheight}{700pt} % NON TOCCARE
\setlength{\parindent}{0pt} % INDENTAZIONE

\lhead{\nomedoc\  (ver. \versione)}
\chead{}
\rhead{\includegraphics[height=1cm]{img/netmus.png}}
\lfoot{\includegraphics[height=0.8cm]{img/logo.png}}
\cfoot{}
\rfoot{\thepage}

% \usepackage{listings}   per codice sorgente

\author{VT.G - Valter Texas Group}

\begin{document}

\pagenumbering{Roman} % INIZIO NUMERAZIONE ARABA

\vspace*{1cm}
\begin{center}

\includegraphics[width=9cm]{img/logo.png}\\
\vspace{0.5cm}
\begin{LARGE} \sca{VT.G - Valter Texas Group} \end{LARGE}\\
\vspace{0.5cm}
\begin{Large}
\emph{valtertexasgroup@googlegroups.com} \end{Large}\\
\vspace*{1cm} \includegraphics[width=5cm]{img/netmus.png}\\
\vspace{0.5cm}
\begin{Large} \sca{\nomedoc} \end{Large}\\
\vspace{1cm}
\begin{Large} \emph{Ingegneria del Software A.A. 2010-2011} \end{Large}\\
\end{center}
\vspace{1cm}

% INFORMAZIONI DOCUMENTO
\begin{center}
\begin{tabular}{r|l}
\hline & \\
\bo{Nome} & \nomefile \\
\bo{Versione attuale} & \versione \\
\bo{Data creazione} & \datacreazione \\
\bo{Data ultima modifica} & \datamodifica \\
\bo{Stato} & \stato \\
\bo{Uso} & \uso \\
\bo{Redazione} & \redazione \\
\bo{Verifica} & \verifica \\
\bo{Approvazione} & \approvazione \\
\bo{Distribuzione} & \distribuzione \\
& \\\hline
\end{tabular}
\end{center}
\newpage

% REGISTRO MODIFICHE
\section*{Registro delle modifiche}
\begin{tabular}{lll}
% REVISIONI DEL DOCUMENTO
% DALLA PIU' NUOVA ALLA PIU' VECCHIA

% TEMPLATE PER REVISIONE

\bo{Data:} 07/12/2010 &
\bo{Versione:} 0.1 &
\bo{Autore:} Palazzin Alberto\\
\hline\\
\multicolumn{3}{p{470px}}{ Stesura prima versione di StudioDiFattibilita.}\\ \\

\end{tabular}

% INDICE
\tableofcontents
\thispagestyle{fancy} % per lo stile di header e footer


\chapter*{Sommario}


\thispagestyle{fancy} % serve perche' nelle pagine di inizio Chapter esca header e footer
\pagenumbering{arabic} % INIZIO NUMERAZIONE NORMALE
\rfoot{\thepage\ di \pageref{TotPages}}
\addcontentsline{toc}{chapter}{Sommario}

\chapter{Introduzione}
\thispagestyle{fancy} % serve perche' nelle pagine di inizio Chapter esca header e footer

\section{Scopo del documento}
Il presente documento ha lo scopo di valutare la fattibilit\`a di progettazione
e sviluppo di ogni capitolato d'appalto rispetto alle tecnologie e alle tempistiche richieste. Nello specifico vengono descritte le motivazioni per le quali il gruppo ha scelto il capitolato C2 piuttosto che altri.


\section{Scopo del prodotto}

\section{Glossario}
Il Glossario \`e definito con un documento a parte (\emph{Glossario.pdf}). Tutti
i termini caratterizzati da \underline{questa sottolineatura} sono ivi
definiti.\\ Verr\`a sottolineata solamente la prima occorrenza di ciascun
termine presente nel Glossario, per non compromettere la leggibilit\`a del documento.

\section{Riferimenti}

\subsection{Normativi} % oppure rif. a Norme di progetto con leggi e tutto
\begin{itemize}
  \item ISO/IEC 12207:1995 - Cicli di vita software
  \item ISO/IEC 9126:2001 - Quality Model
\end{itemize}

\subsection{Informativi}
\begin{itemize}
  \item Capitolato d'appalto CO2-NETMUS del corso di Ingegneria del Software
  A.A. 2010/11 :\\
  \url{http://www.math.unipd.it/~tullio/..}
  \item Slide delle lezioni del corso :\\
  \url{http://www.math.unipd.it/~tullio/.d.}
\end{itemize}


% INIZIO CAPITOLO 2

\chapter{Capitolato C1 MindSlide}
\thispagestyle{fancy}
Il capitolato C1 MindSlide (Software di presentazione basato su mind mapping)
proposto dall'azienda Zucchetti richiede l'utilizzo della tecnologia HTML5. Dopo
un'attenta lettura della specifica, si ritiene che non sia abbastanza stimolante.

\chapter{Capitolato C2 Netmus}
\thispagestyle{fancy}
Il capitolato C2 NetMus (Libreria musicale personale, virtuale e distribuita)
proposto dal prof Riccardo Cardin richiede l'utilizzo delle tecnologie Google
App Engine, Google Web Toolkit e JavaFX. Dopo un'accurata lettura e un'intensa
discussione si \`e deciso di accettare questo progetto.
\section{Studio del dominio} 
Le conoscenze e le capacit\`a per l'implementazione del prodotto si basano su
questi ambiti:
\begin{itemize}
  \item internet e cloud computing
  \item gestione remota, concorrenza e distribuzione
  \item programmazione JAVA e utilizzo tecnologie Google App Engine, Google Web
  Toolkit e RIA
  \item progettazione con UML e Gantt
  \item sistemi operativi vari (Windows, Linux, MacOsx)
\end{itemize}

\section{Fattibilit\`a Tecnologica e Tempistica}
Considerando la preparazione ricevuta ai corsi universitari di Programmazione
Concorrente e Distribuita ed Ingegneria del Software si dovrebbe avere la
capacit\`a per utilizzare i linguaggi JAVA e UML, e lo strumento Gantt. Invece
per le nuove tecnologie Google App Engine e Google Web Toolkit si pu\`o utilizzare
il corso di preparazione messo a disposizione dall'azienda Miriade S.p.A.
committente del capitolato C3. Riguardo la piattaforma JavaFx per la realizzazione delle RIA, non dovrebbe essere un problema recuperare le informazioni per l'applicazione . Infine dopo aver analizzato la conoscenza interna del gruppo, capacit\`a di apprendimento e impegni vari si dovrebbe riuscire a sviluppare il prodotto e rispettare le date di consegna per le varie revisioni.

\section{Componenti del Progetto e Dettagli sulle Tecnologie}
Il problema che viene posto nel capitolato 02 presenta due principali componenti
da considerare per lo studio della fattibilit\`a. Una componente che gestisce il
recupero delle informazioni dei brani musicali dal dispositivo di riproduzione personale dell'utente.
L'altra componente gestisce la memorizzazzione e visualizzazione della libreria virtuale personale 
dell'utente.

\subsection{Componente di estrazione delle informazioni}
Per sviluppare la prima componente si pensava di ricorrere alla creazione di una 
Rich Internet Application, ossia un'applicazione web che possiede le caratteristiche 
e le funzionalit� delle applicazioni desktop, senza per� necessitare dell'installazione 
sul disco fisso. Questa scelta \`e stata dettata dal fatto che le RIA si
caratterizzano per la dimensione interattiva, la multimedialit� e la velocit� d'esecuzione.
La parte dell'applicazione che elabora i dati \`e trasferita a livello client e
fornisce una pronta risposta all'interfaccia utente, mentre la gran parte dei dati e dell'applicazione 
rimane sul server remoto, con notevole alleggerimento per il computer utente. 
Queste caratteristiche rendono l'applicazione il meno invasiva possibile per
l'utente.

\subsection{Componente di memorizzazione e gestione della libreria virtuale}
La seconda componente dovr� poter essere accessibile all'utente come
applicazione web. Il sistema consigliato per lo sviluppo, ossia Google App Engine, 
\`e una tecnologia cloud computing che permette l'utilizzo di risorse hardware o
 software distribuite in remoto. Utilizzer� inoltre il database distribuito Google DataStore. 
Questo Database utilizza una sintassi di linguaggio simile all'SQL chiamato GQL.

\section{Rischi}
L'impegno nell'analisi della fattibilit\`a dovrebbe aver minimizzato le variabili di rischio.
Tuttavia la mancanza di esperienza o imprevisti inattesi come la mancanza di apporto di risorse umane potrebbero provocare ritardi o addiritttura la non riuscita nel completamento del lavoro.

\chapter{Capitolato C3 RescueMe}
\thispagestyle{fancy}
Il capitolato C3 RescueMe (Sistema globale per gestione catastrofi naturali)
proposto dall'azienda Miriade S.p.A. richiede l'utilizzo delle tecnologie
Google App Engine, Google Web Toolkit e Spring MVC. Dopo una profonda analisi,
si ritiene che gli strumenti da utilizzare siano di interesse comune al gruppo.
Per\`o si pensa che l'implementazione del progetto richieda tempi molto lunghi, col rischio di non rispettare i tempi di consegna prefissati.

\chapter{Capitolato C4 QWAD}
\thispagestyle{fancy}
Il capitolato C4 QWAD (Quick Web Apps Designer) proposto dalla Onlus Informatici
Senza Frontiere presuppone la continuazione di un lavoro gi\`a svolto da altri
team di sviluppo. Questo fa ermergere la difficolt\`a di studio e preparazione degli strumenti richiesti. Pur considerando la gratuit\`a sociale di tale progetto, si \`e comunque deciso di non accettarlo.

\end{document}