
\section{Scopo del prodotto}
Il progetto NetMus nasce con lo scopo di realizzare un sistema
software basato su \underline{cloud} \underline{computing}, per memorizzare
informazioni di brani musicali in profili utente online.\\ Tali informazioni vengono estratte da
dispositivi musicali o di archiviazione \underline{USB} al momento della loro connessione.

\section{Glossario}
Il Glossario \`e definito con un documento a parte
(\emph{Glossario-\versioneglossario.pdf}). Tutti i termini caratterizzati da
\underline{questa sottolineatura} sono ivi definiti.\\
Verr\`a sottolineata solamente la prima occorrenza di ciascun
termine presente nel Glossario, per non compromettere la leggibilit\`a del documento.

\newpage
\section{Riferimenti}

\subsection{Normativi} % oppure rif. a Norme di progetto con leggi e tutto
\begin{itemize}
  \item Capitolato d'appalto CO2-NETMUS del corso di Ingegneria del Software
  A.A. 2010/11 :\\
  \url{http://www.math.unipd.it/~tullio/IS-1/2010/Progetto/NetMus.pdf};
  \item Verbale intervista proponente:\\
  \co{allegato Verbale-1.0.pdf};
  \item \emph{NormeDiProgetto-\versionenormeprogetto.pdf} che regola e
  accompagna tutti i documenti ufficiali.
\end{itemize}

\subsection{Informativi}
\begin{itemize}
  \item Slide delle lezioni del corso:\\
  \url{http://www.math.unipd.it/~tullio/IS-1/2010/};
  \item ISO/IEC 12207 - Processi Software;
  \item ISO/IEC 9126 - Qualit\`a software;
  \item Sistema cloud computing Google App Engine:\\
  \url{http://code.google.com/intl/it/appengine/};
  \item Set di strumenti Google Web Toolkit:\\
  \url{http://code.google.com/intl/it/webtoolkit/};
  \item Piattaforma JavaFX:\\
  \url{http://javafx.com/}.
\end{itemize}
