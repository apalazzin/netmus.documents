
\newcommand{\nomedoc}{Modello}
\newcommand{\versione}{0.4}
\newcommand{\nomefile}{modello\versione.pdf}
\newcommand{\datacreazione}{2 Dicembre 2010}
\newcommand{\datamodifica}{9 Dicembre 2010}
\newcommand{\stato}{formale}
\newcommand{\uso}{interno}
\newcommand{\redazione}{Mandolo Andrea}
\newcommand{\verifica}{Baffo}
\newcommand{\approvazione}{Valter}
\newcommand{\distribuzione}{
VT.G \\
& Prof. Vardanega Tullio }

% FUNZIONI TIPOGRAFICHE
\newcommand{\co}{\texttt} % courier
\newcommand{\bo}{\textbf} % bold
\newcommand{\pr}{\par\medskip} % paragrafo spaziato
\newcommand{\sca}{\textsc} % small caps

\documentclass[a4paper,12pt]{report}
% 10pt,11pt,12pt
% titlepage, notitlepage -> per dare inizio o no ad una nuova pagina dopo titolo
% twoside -> per dire se fronte-retro
\usepackage[latin1]{inputenc}
% per caratteri accentati
\usepackage[italian]{babel}
% per regole sintattiche italiane
\usepackage[bookmarks=true, pdfborder={0 0 0 0}]{hyperref}
% per collegamenti ipertestuali
\usepackage{graphicx}
% per inserimento immagini

% \usepackage{enumerate}
% per personalizzare elenchi puntati

\usepackage[hmargin=2cm]{geometry} %margine 2 cm
%\geometry{options varie}

% comandi per gestire meglio header e footer
\usepackage{fancyhdr}  % header e footer
\usepackage{totpages}
\pagestyle{fancy}
\renewcommand{\headrulewidth}{0.4pt}
\renewcommand{\footrulewidth}{0.4pt}

\setlength{\headheight}{1.2cm} % NON TOCCARE
\setlength{\voffset}{-1.5cm} % NON TOCCARE
\setlength{\textheight}{700pt} % NON TOCCARE
\setlength{\parindent}{0pt} % INDENTAZIONE

\lhead{\nomedoc\  (ver. \versione)}
\chead{}
\rhead{\includegraphics[height=1cm]{img/netmus.png}}
\lfoot{\includegraphics[height=0.8cm]{img/logo.png}}
\cfoot{}
\rfoot{\thepage}

% \usepackage{listings}   per codice sorgente

\author{VT.G - Valter Texas Group}

\begin{document}

\pagenumbering{Roman} % INIZIO NUMERAZIONE ARABA

\vspace*{1cm}
\begin{center}

\includegraphics[width=9cm]{img/logo.png}\\
\vspace{0.5cm}
\begin{LARGE} \sca{VT.G - Valter Texas Group} \end{LARGE}\\
\vspace{0.5cm}
\begin{Large}
\emph{valtertexasgroup@googlegroups.com} \end{Large}\\
\vspace*{1cm} \includegraphics[width=5cm]{img/netmus.png}\\
\vspace{0.5cm}
\begin{Large} \sca{\nomedoc} \end{Large}\\
\vspace{1cm}
\begin{Large} \emph{Ingegneria del Software A.A. 2010-2011} \end{Large}\\
\end{center}
\vspace{1cm}

% INFORMAZIONI DOCUMENTO
\begin{center}
\begin{tabular}{r|l}
\hline & \\
\bo{Nome} & \nomefile \\
\bo{Versione attuale} & \versione \\
\bo{Data creazione} & \datacreazione \\
\bo{Data ultima modifica} & \datamodifica \\
\bo{Stato} & \stato \\
\bo{Uso} & \uso \\
\bo{Redazione} & \redazione \\
\bo{Verifica} & \verifica \\
\bo{Approvazione} & \approvazione \\
\bo{Distribuzione} & \distribuzione \\
& \\\hline
\end{tabular}
\end{center}
\newpage

% REGISTRO MODIFICHE
\section*{Registro delle modifiche}
\begin{tabular}{lll}
% REVISIONI DEL DOCUMENTO
% DALLA PIU' NUOVA ALLA PIU' VECCHIA
\bo{Data:} 09/12/2010 &
\bo{Versione:} 0.4 &
\bo{Autore:} Daminato Simone\\
\hline\\
\multicolumn{3}{p{470px}}{ Redazione del capitolo 2.}\\
\\

\bo{Data:} 07/12/2010 &
\bo{Versione:} 0.3 &
\bo{Autore:} Lovato Daniele\\
\hline\\
\multicolumn{3}{p{470px}}{ Redazione del capitolo 1.}\\
\\

% TEMPLATE PER REVISIONE
\bo{Data:} 03/12/2010 &
\bo{Versione:} 0.2 &
\bo{Autore:} Mandolo Andrea\\
\hline\\
\multicolumn{3}{p{470px}}{ Rinominato file, dimensione caratteri 12pt,
aggiunti alcuni contenuti comuni.}\\
\\

\bo{Data:} 03/12/2010 &
\bo{Versione:} 0.1 &
\bo{Autore:} Mandolo Andrea\\
\hline\\
\multicolumn{3}{p{470px}}{ Stesura prima versione di modello.}\\ \\

\end{tabular}

% INDICE
\tableofcontents
\thispagestyle{fancy} % per lo stile di header e footer


\chapter*{Sommario}


\thispagestyle{fancy} % serve perche' nelle pagine di inizio Chapter esca header e footer
\pagenumbering{arabic} % INIZIO NUMERAZIONE NORMALE
\rfoot{\thepage\ di \pageref{TotPages}}
\addcontentsline{toc}{chapter}{Sommario}

\chapter{Introduzione}
\thispagestyle{fancy} % serve perche' nelle pagine di inizio Chapter esca header e footer

\section{Scopo del documento}
Il presente documento ha lo scopo di presentare al committente l'Analisi dei
Requisiti e le potenzialit\`a software del nostro prodotto per soddisfare le
richieste del capitolato d'appalto 02 NetMus.


\section{Scopo del prodotto}

\section{Glossario}
Il Glossario \`e definito con un documento a parte (\emph{Glossario.pdf}). Tutti
i termini caratterizzati da \underline{questa sottolineatura} sono ivi
definiti.\\ Verr\`a sottolineata solamente la prima occorrenza di ciascun
termine presente nel Glossario, per non compromettere la leggibilit\`a del documento.

\section{Riferimenti}

\subsection{Normativi} % oppure rif. a Norme di progetto con leggi e tutto
\begin{itemize}
  \item ISO/IEC 12207:1995 - Cicli di vita software
  \item ISO/IEC 9126:2001 - Quality Model
\end{itemize}

\subsection{Informativi}
\begin{itemize}
  \item Capitolato d'appalto CO2-NETMUS del corso di Ingegneria del Software
  A.A. 2010/11 :\\
  \url{http://www.math.unipd.it/~tullio/..}
  \item Slide delle lezioni del corso :\\
  \url{http://www.math.unipd.it/~tullio/.d.}
\end{itemize}


% INIZIO CAPITOLO 2
\chapter{Descrizione generale}
\section{Contesto d'uso del prodotto}
Il progetto Netmus che abbiamo intenzione di sviluppare \`e un software
destinato ad una vasta gamma d'utenza, compresa nella fascia d'et\`a dai 12 ai
100 anni, che interesser\`a per\`o soprattutto adolescenti e adulti dai 15 ai 35 anni.
\subsection{Piattaforma d'esecuzione e interfacciamento con l'ambiente di
installazione e uso}
Il prodotto finale utilizzer\`a diverse teconologie (in particolare: Java, GAE e
GWT) che renderanno possibile l'utilizzo del sistema su pi\`u piattaforme
d'esecuzione con sistemi operativi diversi, i cui unici due requisiti
fondamentali saranno di possedere una connessione ad internet e di aver
installata la Java Runtime Machine. In particolare, il prodotto finale sar\`a
testato sui principali sistemi operativi: Windows (xp e 7), Linux (principali e
pi\`u diffuse distribuzioni), Mac OS X (Snow Leopard).
\section{Funzioni del prodotto}
Il prodotto finale sar\`a composto da due componenti. Una componente di
persistenza e visualizzazione della libreria virtuale dell'utente. Un'altra
componente di recupero delle informazioni dei brani musicali contenuti nei
sistemi di riproduzione personale dell'utente. La prima componente avr\`a il
compito di gestire la memorizzazione delle informazioni dei brani nel database,
e la sua visualizzazione e gestione con un'interfaccia semplice e facilmente
utilizzabile. La seconda componente svolger\`a il ruolo di estrazione delle
informazioni dai brani musicali, e le invier\`a al server senza interferire con
l'utente che intanto potr\`a continuare a svolgere il proprio lavoro indisturbato.
L'estrazione avverr\`a non appena verr\`a rilevato un nuovo dispositivo di
archiviazione di massa (storage USB o lettore mp3) che non cripta i propri file,
e in modo del tutto autonomo verranno ricavate le informazioni, memorizzate in
locale in caso di assenza di connessione, e poi successivamente inviate al server.
\section{Caratteristiche degli utenti}
L'utente tipo che utilizza il sistema non \`e in possesso di particolari
conoscenze informatiche, ma \`e in grado di navigare sul web e di eseguire alcune
semplici operazioni correlate (come per esempio effettuare un login).
\section{Vincoli generali}
Il software nel suo funzionamento e sviluppo:
\begin{itemize}
  \item sar\`a completo della documentazione necessaria e del manuale d'uso,
  \item sar\`a portabile (verr\`a testato e garantito il suo funzionamento sui
  principali sistemi operativi),
  \item sar\`a sicuro per gli utenti: garantir\`a il controllo del proprio account e
  dei propri dati, e della propria privacy,
  \item per la raccolta dei dati dei brani localmente non sar\`a necessario
  l'accesso ad internet,
  \item per l'aggiornamento e la visualizzazione dei propri dati online sar\`a
  necessaria la connessione ad internet,
  \item il manuale utente sar\`a fornito sia in italiano che in inglese.
\end{itemize}
\section{Assunzioni e dipendenze}
Si assume che sui personal computer utilizzati dagli utenti sia gi\`a installata
una JVM.
\section{Glossario}
Esister\`a un unico glossario comune a tutti i documenti, contenente la
spiegazione di tutti i termini e gli acronimi che verranno utilizzati
all'interno della documentazione. Questo documento sar\`a organizzato in ordine
alfabetico in modo da permettere una rapida ricerca dei termini e sar\`a suddiviso
due categorie: termini e acronimi.
\end{document}
