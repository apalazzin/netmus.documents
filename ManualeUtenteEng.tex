\newcommand{\nomedoc}{User's Guide}
\newcommand{\versione}{1.0}
\newcommand{\versioneglossario}{3.0}
\newcommand{\versionenormeprogetto}{3.0}
\newcommand{\nomefile}{User's Guide-\versione.pdf}
\newcommand{\datacreazione}{24 Febbraio 2011}
\newcommand{\datamodifica}{27 Febbraio 2011}
\newcommand{\stato}{formale}
\newcommand{\uso}{esterno}
\newcommand{\redazione}{}
\newcommand{\verifica}{}
\newcommand{\approvazione}{}
\newcommand{\distribuzione}{
VT.G \\
& Prof. Vardanega Tullio\\
& Prof. Cardin Riccardo }

% FUNZIONI TIPOGRAFICHE
\newcommand{\co}{\texttt} % courier
\newcommand{\bo}{\textbf} % bold
\newcommand{\pr}{\par\medskip} % paragrafo spaziato
\newcommand{\sca}{\textsc} % small caps

\documentclass[a4paper,12pt]{report}
% 10pt,11pt,12pt
% titlepage, notitlepage -> per dare inizio o no ad una nuova pagina dopo titolo
% twoside -> per dire se fronte-retro
\usepackage[latin1]{inputenc}
% per caratteri accentati
\usepackage[italian]{babel}
% per regole sintattiche italiane
\usepackage[bookmarks=true, pdfborder={0 0 0 0}]{hyperref}
% per collegamenti ipertestuali
\usepackage{graphicx}
% per inserimento immagini

% \usepackage{enumerate}
% per personalizzare elenchi puntati

\usepackage[hmargin=2cm]{geometry} %margine 2 cm
%\geometry{options varie}

% comandi per gestire meglio header e footer
\usepackage{fancyhdr}  % header e footer
\usepackage{totpages}
\pagestyle{fancy}
\renewcommand{\headrulewidth}{0.4pt}
\renewcommand{\footrulewidth}{0.4pt}

\setlength{\headheight}{1.2cm} % NON TOCCARE
\setlength{\voffset}{-1.5cm} % NON TOCCARE
\setlength{\textheight}{700pt} % NON TOCCARE
\setlength{\parindent}{0pt} % INDENTAZIONE

\lhead{\nomedoc\  (ver. \versione)}
\chead{}
\rhead{\includegraphics[height=1cm]{img/netmus.png}}
\lfoot{\includegraphics[height=0.8cm]{img/logo.png}}
\cfoot{}
\rfoot{\thepage}

% \usepackage{listings}   per codice sorgente

\author{VT.G - Valter Texas Group}

\begin{document}

\pagenumbering{Roman} % INIZIO NUMERAZIONE ARABA

\vspace*{1cm}
\begin{center}

\includegraphics[width=9cm]{img/logo.png}\\
\vspace{0.5cm}
\begin{LARGE} \sca{VT.G - Valter Texas Group} \end{LARGE}\\
\vspace{0.5cm}
\begin{Large}
\emph{valtertexasgroup@googlegroups.com} \end{Large}\\
\vspace*{1cm} \includegraphics[width=5cm]{img/netmus.png}\\
\vspace{0.5cm}
\begin{Large} \sca{\nomedoc} \end{Large}\\
\vspace{1cm}
\begin{Large} \emph{Ingegneria del Software A.A. 2010-2011} \end{Large}\\
\end{center}
\vspace{1cm}

% INFORMAZIONI DOCUMENTO
\begin{center}
\begin{tabular}{r|l}
\hline & \\
\bo{Name} & \nomefile \\
\bo{Current Version} & \versione \\
\bo{Creation} & \datacreazione \\
\bo{Last Modify} & \datamodifica \\
\bo{State} & \stato \\
\bo{Use} & \uso \\
\bo{Editing} & \redazione \\
\bo{Control} & \verifica \\
\bo{Approbation} & \approvazione \\
\bo{Distribution} & \distribuzione \\
& \\\hline
\end{tabular}
\end{center}
\newpage

\chapter*{Summary}
\thispagestyle{fancy}
This document is a simply and intuitive guide for beginners using NetMus
system.\\
It includes a glossary where you can find more-difficult terms that are used
here and an appendix whith the most common promblems that you can encounter,
possible reasons that caused the problem and possible solutions. 

\newpage
% REGISTRO MODIFICHE
\section*{Change History Log}

\begin{longtable}{|p{0.13\textwidth}|c|p{0.2\textwidth}|p{0.46\textwidth}|}
\hline
\rowcolor{orange} \bo{Data} & \bo{Version} & \bo{Author} & \bo{Description} \\
\endhead
\hline 
08/02/2011 & 0.1 & Lovato Daniele & Creation of the document.\\
\hline

\end{longtable}

% INDEX
\tableofcontents

\chapter{Introduction}
\thispagestyle{fancy} % serve perche' nelle pagine di inizio Chapter esca header e footer
\pagenumbering{arabic} % INIZIO NUMERAZIONE NORMALE
\rfoot{\thepage\ di \pageref{TotPages}}
This guide was created for beginners that want start using \co{NetMus}, and it
includes a description of the product and the instrutions for use it. There are
also two appendix in where you can find most common error messages and a
glossary.\\

All the terms in the glossary are underlined in the document in their first
occurence \underline{like this}.

\section{Description of the Product's User}
\co{NetMus} can be used by everyone: you need to know only the basics like using
a \underline{web browser} and navigating through internet. 


\section{How to Read this Guide}
This Guide presents the product \co{NetMus} and describes its functionality and
also the user's approach using the system. In particular it will show the 
actions that you can perform with the software and how to resolve, if it is
possible, problems that you can encounter using it.

\section{Helpful Documents}
Here we present the documents that we have used for writing this guide, and that
you can use to understand more clearly this words:
\begin{itemize}
  \item Analisi dei requisiti v3.0
  \item Norme di progetto v3.0
  \item Piano di progetto v3.0
  \item Specifica tecnica v2.0
  \item Definizione del prodotto v1.0
  \item Piano di qualifica v3.0
  \item Verbale 1 v1.0
  \item Capitolato C2 - NetMus del corso di Ingegneria del Software, A.A.
2010/11 (you can read it at the link 
\url{http://www.math.unipd.it/~tullio/IS-1/2010/Progetto/NetMus.pdf})
\end{itemize}

\section{How to report problems and not-working situations}
If you found any problems you can report it with the online service provided by
Google at this link:  \url{http://code.google.com/p/netmus/issues/list}.\\
You just have to click on top-left button \emph{New issue} and fill the
fields with a name for the problem and a short description of it.
If you want an help in descripting the problem, you can use the template
\emph{Defect report from user} that contains a track that help you writing all
the necessary informations. 

\chapter{General Description}
\thispagestyle{fancy}
Today, the biggest part of the population listen their music in Mp3 digital
format: it's for that reasons that was introduced \co{NetMus}, an useful
application that allows sharing our preferred music and listening it on online
streaming, and that was created with modern tecnologies like \underline{cloud
computing}.\\

\co{NetMus} allows to any user to have a virtual online library with all their
preferred songs, and to listen this songs everywhere and share them with
everyone.\\

This product is not only an online music library, but it has also a lot of
``\underline{social network}'' funcionalities that allows you to interact with every other user
registered to the system. So it's possible to view other users' music library
and listen their own songs, and also it's possible to make your own friend list
with all the other users registered to the system.\\

Using the product is very simple. Just after you get logged to \co{NetMus} you
only have to connect your PC with your Mp3 player and the system will start
working automatically. Otherwise you can also scan a local directory of your PC
that contains music in the mp3 format: all the songs will be fully extracted,
analized and inserted in your profile on your \co{NetMus} account, ready to be
listened.\\

Between \co{NetMus}' funcionalities we have streaming youtube player for audio
and video broadcasting, possibility to create your own playlists, and many other
utilities.\\

All the system is decorated with a graphical interface  too simple and intuitive
with the complete access to all the necessary funcionalities.\\


\chapter{Using Instructions}
\thispagestyle{fancy}

\section{Functional Description}
In this chapter we show a complete guide for \co{NetMus} use.\\
You will find a section were are listed all the required tools for a well
working approach with \co{NetMus}, then you will encounter a detailed overview
of \co{NetMus} with the description of all the actions you can perform with the
application.

\subsection{Recommended System Requirements for NetMus Use}
\co{NetMus} is a web application that is viewed and used with your web browser.
Although you reached it by internet navigation, you don't need only an internet
connection and a web browser to use it.\\

Now we show a list of the required tools that you need for use \co{NetMus}.

\begin{itemize}
  \item Internet Connection (higher than 56k)
  \item Web Browser (Google Chrome, Safari, Opera, Firefox*, IE**)
  \item A JVM (Java Virtual Machine) correctly installed on your PC
  \item Adobe Flash Player plugin for video broadcasting
  \item Javascript activated on your web browser
\end{itemize}

Without anyone of this tools you will not able to use
correctly all the system functionalities, or even you will unable to use
\co{NetMus} at all.\\
\\
\\

\emph{* If you use Firefox, you won't see the animation effects of the
system, because Firefox dosen't support the animations.}\\ 
\emph{** If you wanna use NetMus con Microsoft Internet Explorer, you have to
download the Google Chrome Frame plugin for IE, because this web browser doesn't
support the advanced funcionalities of NetMus.}\\

\section{Permitted Actions}

Go at the link \url{http://netmusbeta.appspot.com} to start using \co{NetMus}.\\

\begin{figure}[htbp]
  \centering
  \includegraphics[width=15cm]{img/MU/login.png}
\caption{Netmus Login View}
\end{figure}


Here is the login page to enter \co{NetMus}. If you are already registered to
Google, you can enter in \co{NetMus} using your Google account.

You have to select the Google login and click on the button ``Log in
NetMus using your Google Account''. You will be redirected to the Google login
page, here you get logged in and then you will be redirected again to
\co{NetMus}, but this time you will be logged to the system.\\

\begin{figure}[htbp]
  \centering
  \includegraphics[width=15cm]{img/MU/loginGoogle.png}
\caption{Google Login View}
\end{figure}


If you aren't Google users and this si your firts access to \co{NetMus}, it
necessary that you create your own account registering to the system.
\co{NetMus} registration and use are completely free. \\

\begin{figure}[htbp]
  \centering
  \includegraphics[width=15cm]{img/MU/registration.png}
\caption{Registration View}
\end{figure}


Registering yourself to the system you need to insert a correct email address
and a password with more than 5 characters. An email will be sent to your email
address for the confirmation of your account's activation.

\subsection{First Access to NetMus}

At your first access, the view that will be presented to you will be like that.

\begin{figure}[htbp]
  \centering
  \includegraphics[width=15cm]{img/MU/profile_blank.png}
\caption{NetMus Homepage at first access}
\end{figure}

Beginning from the left, we find a menu that presents the \co{NetMus} logo and
your \underline{nickname}, that for now is your email address. It also presents
the number of songs contained in your songs' list, that are zero at your first
access. This is playlist menu, infact here we can find all the playlist you have
created, and it also leads creating new ones by clicking on the red ``+''
symbol.\\
\\
The central section is the most important part of the apllication, infact this
is our music library with youtube player and other buttons for other
funcionalities.\\
\\
At the right side of the page there is a bar called ``DEVICE SCANNER BAR" that
allows the scanning process of the devices. If you hover this bar with your
mouse pointer, the bar will expand revealing many details: a label were is
written  the scanning process status and a button for the manual scanning
process of an arbitrary directory selected.\\

\begin{figure}[htbp]
  \centering
  \includegraphics[width=15cm]{img/MU/applet_bar_open.png}
\caption{NetMus homepage with expanded DEVICE SCANNER BAR}
\end{figure}

\subsection{Start Using NetMus}




\section{Errors and their Possible Reasons}

\appendix % inizio appendice
\chapter{Most Common Error Messages}
\thispagestyle{fancy}

\chapter{Glossary}
\thispagestyle{fancy}

\end{document}