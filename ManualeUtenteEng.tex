\newcommand{\nomedoc}{User's Guide}
\newcommand{\versione}{1.0}
\newcommand{\versioneglossario}{3.0}
\newcommand{\versionenormeprogetto}{3.0}
\newcommand{\nomefile}{User's Guide-\versione.pdf}
\newcommand{\datacreazione}{24 Febbraio 2011}
\newcommand{\datamodifica}{27 Febbraio 2011}
\newcommand{\stato}{formale}
\newcommand{\uso}{esterno}
\newcommand{\redazione}{}
\newcommand{\verifica}{}
\newcommand{\approvazione}{}
\newcommand{\distribuzione}{
VT.G \\
& Prof. Vardanega Tullio\\
& Prof. Cardin Riccardo }

% FUNZIONI TIPOGRAFICHE
\newcommand{\co}{\texttt} % courier
\newcommand{\bo}{\textbf} % bold
\newcommand{\pr}{\par\medskip} % paragrafo spaziato
\newcommand{\sca}{\textsc} % small caps

\documentclass[a4paper,12pt]{report}
% 10pt,11pt,12pt
% titlepage, notitlepage -> per dare inizio o no ad una nuova pagina dopo titolo
% twoside -> per dire se fronte-retro
\usepackage[latin1]{inputenc}
% per caratteri accentati
\usepackage[italian]{babel}
% per regole sintattiche italiane
\usepackage[bookmarks=true, pdfborder={0 0 0 0}]{hyperref}
% per collegamenti ipertestuali
\usepackage{graphicx}
% per inserimento immagini

% \usepackage{enumerate}
% per personalizzare elenchi puntati

\usepackage[hmargin=2cm]{geometry} %margine 2 cm
%\geometry{options varie}

% comandi per gestire meglio header e footer
\usepackage{fancyhdr}  % header e footer
\usepackage{totpages}
\pagestyle{fancy}
\renewcommand{\headrulewidth}{0.4pt}
\renewcommand{\footrulewidth}{0.4pt}

\setlength{\headheight}{1.2cm} % NON TOCCARE
\setlength{\voffset}{-1.5cm} % NON TOCCARE
\setlength{\textheight}{700pt} % NON TOCCARE
\setlength{\parindent}{0pt} % INDENTAZIONE

\lhead{\nomedoc\  (ver. \versione)}
\chead{}
\rhead{\includegraphics[height=1cm]{img/netmus.png}}
\lfoot{\includegraphics[height=0.8cm]{img/logo.png}}
\cfoot{}
\rfoot{\thepage}

% \usepackage{listings}   per codice sorgente

\author{VT.G - Valter Texas Group}

\begin{document}

\pagenumbering{Roman} % INIZIO NUMERAZIONE ARABA

\vspace*{1cm}
\begin{center}

\includegraphics[width=9cm]{img/logo.png}\\
\vspace{0.5cm}
\begin{LARGE} \sca{VT.G - Valter Texas Group} \end{LARGE}\\
\vspace{0.5cm}
\begin{Large}
\emph{valtertexasgroup@googlegroups.com} \end{Large}\\
\vspace*{1cm} \includegraphics[width=5cm]{img/netmus.png}\\
\vspace{0.5cm}
\begin{Large} \sca{\nomedoc} \end{Large}\\
\vspace{1cm}
\begin{Large} \emph{Ingegneria del Software A.A. 2010-2011} \end{Large}\\
\end{center}
\vspace{1cm}

% INFORMAZIONI DOCUMENTO
\begin{center}
\begin{tabular}{r|l}
\hline & \\
\bo{Name} & \nomefile \\
\bo{Current Version} & \versione \\
\bo{Creation} & \datacreazione \\
\bo{Last Modify} & \datamodifica \\
\bo{State} & \stato \\
\bo{Use} & \uso \\
\bo{Editing} & \redazione \\
\bo{Control} & \verifica \\
\bo{Approbation} & \approvazione \\
\bo{Distribuition} & \distribuzione \\
& \\\hline
\end{tabular}
\end{center}
\newpage

\chapter*{Summary}
\thispagestyle{fancy}
This document is a simply and intuitive guide for beginners using NetMus
system.\\
It includes a glossary where you can find more-difficult terms that are used
here and an appendix whith the most common promblems that you can encounter,
possible reasons that caused the problem and possible solutions. 

\newpage
% REGISTRO MODIFICHE
\section*{Change History Log}

\begin{longtable}{|p{0.13\textwidth}|c|p{0.2\textwidth}|p{0.46\textwidth}|}
\hline
\rowcolor{orange} \bo{Data} & \bo{Version} & \bo{Author} & \bo{Description} \\
\endhead
\hline 
08/02/2011 & 0.1 & Lovato Daniele & Creation of the document.\\
\hline

\end{longtable}

% INDEX
\tableofcontents

\chapter{Introduction}
\thispagestyle{fancy} % serve perche' nelle pagine di inizio Chapter esca header e footer
\pagenumbering{arabic} % INIZIO NUMERAZIONE NORMALE
\rfoot{\thepage\ di \pageref{TotPages}}
This guide was created for beginners that want start using \co{NetMus}, and it
includes a description of the product and the instrutions for use it. There are
also two appendix in where you can find most common error messages and a
glossary.\\

All the terms in the glossary are underlined in the document in their first
occurence \underline{like this}.

\section{Description of the Product's User}
\co{NetMus} can be used by everyone: you need to know only the basics like using
a \underline{web browser} and navigating through internet. 


\section{How to Read this Guide}
This Guide presents the product \co{NetMus} and describes its functionality and
also the user's approach using the system. In particular it will show the 
actions that you can perform with the software and how to resolve, if it is
possible, problems that you can encounter using it.

\section{Helpful Documents}
Here we present the documents that we have used for writing this guide, and that
you can use to understand more clearly this words:
\begin{itemize}
  \item Analisi dei requisiti v3.0
  \item Norme di progetto v3.0
  \item Piano di progetto v3.0
  \item Specifica tecnica v2.0
  \item Definizione del prodotto v1.0
  \item Piano di qualifica v3.0
  \item Verbale 1 v1.0
  \item Capitolato C2 - NetMus del corso di Ingegneria del Software, A.A.
2010/11 (you can read it at the link 
\url{http://www.math.unipd.it/~tullio/IS-1/2010/Progetto/NetMus.pdf})
\end{itemize}

\section{How to report problems and not-working situations}
If you found any problems you can report it with the online service provided by
Google at this link:  \url{http://code.google.com/p/netmus/issues/list}.\\
You just have to click on top-left button \emph{New issue} and fill the
fields with a name for the problem and a short description of it.
If you want an help in descripting the problem, you can use the template
\emph{Defect report from user} that contains a track that help you writing all
the necessary informations. 

\chapter{General Description}
\thispagestyle{fancy}
Here is a general description of the product.

\chapter{Using Instructions}
\thispagestyle{fancy}

\section{Functional Description}

\section{Permitted Actions}

\section{Errors and their Possible Reasons}

\appendix % inizio appendice
\chapter{Most Common Error Messages}
\thispagestyle{fancy}

\chapter{Glossary}
\thispagestyle{fancy}

\end{document}