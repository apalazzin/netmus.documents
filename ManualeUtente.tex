
\newcommand{\nomedoc}{Manuale Utente}
\newcommand{\versione}{1.0}
\newcommand{\versioneglossario}{3.0}
\newcommand{\versionenormeprogetto}{3.0}
\newcommand{\nomefile}{ManualeUtente-\versione.pdf}
\newcommand{\datacreazione}{24 Febbraio 2011}
\newcommand{\datamodifica}{27 Febbraio 2011}
\newcommand{\stato}{formale}
\newcommand{\uso}{esterno}
\newcommand{\redazione}{---}
\newcommand{\verifica}{---}
\newcommand{\approvazione}{---}
\newcommand{\distribuzione}{
VT.G \\
& Prof. Vardanega Tullio\\
& Prof. Cardin Riccardo }

% FUNZIONI TIPOGRAFICHE
\newcommand{\co}{\texttt} % courier
\newcommand{\bo}{\textbf} % bold
\newcommand{\pr}{\par\medskip} % paragrafo spaziato
\newcommand{\sca}{\textsc} % small caps

\documentclass[a4paper,12pt]{report}
% 10pt,11pt,12pt
% titlepage, notitlepage -> per dare inizio o no ad una nuova pagina dopo titolo
% twoside -> per dire se fronte-retro
\usepackage[latin1]{inputenc}
% per caratteri accentati
\usepackage[italian]{babel}
% per regole sintattiche italiane
\usepackage[bookmarks=true, pdfborder={0 0 0 0}]{hyperref}
% per collegamenti ipertestuali
\usepackage{graphicx}
% per inserimento immagini

% \usepackage{enumerate}
% per personalizzare elenchi puntati

\usepackage[hmargin=2cm]{geometry} %margine 2 cm
%\geometry{options varie}

% comandi per gestire meglio header e footer
\usepackage{fancyhdr}  % header e footer
\usepackage{totpages}
\pagestyle{fancy}
\renewcommand{\headrulewidth}{0.4pt}
\renewcommand{\footrulewidth}{0.4pt}

\setlength{\headheight}{1.2cm} % NON TOCCARE
\setlength{\voffset}{-1.5cm} % NON TOCCARE
\setlength{\textheight}{700pt} % NON TOCCARE
\setlength{\parindent}{0pt} % INDENTAZIONE

\lhead{\nomedoc\  (ver. \versione)}
\chead{}
\rhead{\includegraphics[height=1cm]{img/netmus.png}}
\lfoot{\includegraphics[height=0.8cm]{img/logo.png}}
\cfoot{}
\rfoot{\thepage}

% \usepackage{listings}   per codice sorgente

\author{VT.G - Valter Texas Group}

\begin{document}

\pagenumbering{Roman} % INIZIO NUMERAZIONE ARABA

\vspace*{1cm}
\begin{center}

\includegraphics[width=9cm]{img/logo.png}\\
\vspace{0.5cm}
\begin{LARGE} \sca{VT.G - Valter Texas Group} \end{LARGE}\\
\vspace{0.5cm}
\begin{Large}
\emph{valtertexasgroup@googlegroups.com} \end{Large}\\
\vspace*{1cm} \includegraphics[width=5cm]{img/netmus.png}\\
\vspace{0.5cm}
\begin{Large} \sca{\nomedoc} \end{Large}\\
\vspace{1cm}
\begin{Large} \emph{Ingegneria del Software A.A. 2010-2011} \end{Large}\\
\end{center}
\vspace{1cm}

% INFORMAZIONI DOCUMENTO
\begin{center}
\begin{tabular}{r|l}
\hline & \\
\bo{Nome} & \nomefile \\
\bo{Versione attuale} & \versione \\
\bo{Data creazione} & \datacreazione \\
\bo{Data ultima modifica} & \datamodifica \\
\bo{Stato} & \stato \\
\bo{Uso} & \uso \\
\bo{Redazione} & \redazione \\
\bo{Verifica} & \verifica \\
\bo{Approvazione} & \approvazione \\
\bo{Distribuzione} & \distribuzione \\
& \\\hline
\end{tabular}
\end{center}
\newpage

\chapter*{Sommario}
\thispagestyle{fancy}
(da scrivere)

\newpage
% REGISTRO MODIFICHE
\section*{Registro delle modifiche}

\begin{longtable}{|p{0.13\textwidth}|c|p{0.2\textwidth}|p{0.46\textwidth}|}
\hline
\rowcolor{orange} \bo{Data} & \bo{Versione} & \bo{Autore} & \bo{Descrizione} \\
\hline
\endhead
\hline
\endfoot
24/02/2011 & 0.1 & Daminato Simone & Stesura preambolo.\\ 
24/02/2011 & 0.1 & Mandolo Andrea & Stesura prima versione del Manuale Utente.\\

\end{longtable}

% INDICE
\tableofcontents

\chapter{Introduzione}
\thispagestyle{fancy} % serve perche' nelle pagine di inizio Chapter esca header e footer
\pagenumbering{arabic} % INIZIO NUMERAZIONE NORMALE
\rfoot{\thepage\ di \pageref{TotPages}}
Questo manuale \`e destinato agli utilizzatori di \co{NetMus}, e comprende una
descrizione del prodotto e le istruzioni per l'uso, oltre a due appendici
riportanti gli eventuali messaggi di errore e il glossario.

\section{Definizione dell'utente del prodotto}
\co{NetMus} pu\`o essere utilizzato da chiunque: sono richieste solo poche
nozioni di base, ovvero come navigare in internet utilizzando un browser web.

\section{Come leggere il manuale}
Questo manuale presenta il prodotto \co{NetMus} e descrive le sue funzionalit\`a
e l'approccio dell'utente al suo utilizzo. In particolare, verranno illustrate
le azioni che l'utente pu\`o effettuare con questo software e come risolvere, se
possibile, gli eventuali problemi riscontrabili durante il suo l'utilizzo.

\section{Documenti utili}
I documenti a cui ci si � riferiti per stendere il presente documento, e che
possono tornare utili per la lettura dello stesso, sono:
\begin{itemize}
  \item Analisi dei requisiti v3.0
  \item Norme di progetto v3.0
  \item Piano di progetto v3.0
  \item Specifica tecnica v2.0
  \item Definizione del prodotto v1.0
  \item Piano di qualifica v3.0
  \item Verbale 1 v1.0
  \item Capitolato C2 - NetMus del corso di Ingegneria del Software, A.A.
2010/11 (disponibile all'indirizzo web
\url{http://www.math.unipd.it/~tullio/IS-1/2010/Progetto/NetMus.pdf})
\end{itemize}

\section{Come riportare problemi e malfunzionamenti}
Per segnalare problemi e malfunzionamenti \`e disponibile un servizio online a
questo indirizzo: \url{http://code.google.com/p/netmus/issues/list}.\\
\`E sufficiente cliccare in alto a sinistra su \emph{New issue}, e nel modulo
compilare i campi col titolo e la descrizione del problema.
Come aiuto per descrivere il problema, consigliamo di selezionare il template
\emph{Defect report from user}, che contiene una traccia che aiuta ad inserire
tutte le informazioni necessarie.

\chapter{Descrizione generale}
\thispagestyle{fancy}
Oggigiorno, la maggior parte della musica che ascoltiamo \`e in formato mp3: \`e
per questo che nasce \co{NetMus}, uno strumento per gestire e condividere le
nostre canzoni preferite, che sfrutta le nuove tecnologie, come il cloud
computing.\\

\co{NetMus} permette agli utenti di avere una
libreria virtuale online con tutte le canzoni che si ascoltano abitualmente, in modo da
poterle ascoltare ovunque ci si trovi, e condividerle con altri utenti.\\

\`E molto facile da utilizzare, basta collegare il proprio lettore mp3 e
lasciarlo lavorare in automatico, oppure indicargli in che cartella si tengono
le canzoni: tutte le infomazioni verranno estratte, analizzate, completate e
inserite nel nostro profilo, pronte per essere utilizzate.\\

Tra le possibilit\`a offerte, troviamo sia lo streaming audio che quello
video, il completamento automatico delle informazioni dei brani, le playlist, e
molto altro.
\chapter{Istruzioni per l'uso}
\thispagestyle{fancy}

\section{Descrizione funzionale}

\section{Azioni richieste/permesse}

\section{Errori probabili e cause possibili}


\appendix % inizio appendice
\chapter{Messaggi di errore e loro significato}
\thispagestyle{fancy}

\chapter{Glossario}
\thispagestyle{fancy}

\end{document}
