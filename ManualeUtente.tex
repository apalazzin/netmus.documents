
\newcommand{\nomedoc}{Manuale Utente}
\newcommand{\versione}{1.0}
\newcommand{\versioneglossario}{3.0}
\newcommand{\versionenormeprogetto}{3.0}
\newcommand{\nomefile}{ManualeUtente-\versione.pdf}
\newcommand{\datacreazione}{24 Febbraio 2011}
\newcommand{\datamodifica}{27 Febbraio 2011}
\newcommand{\stato}{formale}
\newcommand{\uso}{esterno}
\newcommand{\redazione}{---}
\newcommand{\verifica}{---}
\newcommand{\approvazione}{---}
\newcommand{\distribuzione}{
VT.G \\
& Prof. Vardanega Tullio\\
& Prof. Cardin Riccardo }

% FUNZIONI TIPOGRAFICHE
\newcommand{\co}{\texttt} % courier
\newcommand{\bo}{\textbf} % bold
\newcommand{\pr}{\par\medskip} % paragrafo spaziato
\newcommand{\sca}{\textsc} % small caps

\documentclass[a4paper,12pt]{report}
% 10pt,11pt,12pt
% titlepage, notitlepage -> per dare inizio o no ad una nuova pagina dopo titolo
% twoside -> per dire se fronte-retro
\usepackage[latin1]{inputenc}
% per caratteri accentati
\usepackage[italian]{babel}
% per regole sintattiche italiane
\usepackage[bookmarks=true, pdfborder={0 0 0 0}]{hyperref}
% per collegamenti ipertestuali
\usepackage{graphicx}
% per inserimento immagini

% \usepackage{enumerate}
% per personalizzare elenchi puntati

\usepackage[hmargin=2cm]{geometry} %margine 2 cm
%\geometry{options varie}

% comandi per gestire meglio header e footer
\usepackage{fancyhdr}  % header e footer
\usepackage{totpages}
\pagestyle{fancy}
\renewcommand{\headrulewidth}{0.4pt}
\renewcommand{\footrulewidth}{0.4pt}

\setlength{\headheight}{1.2cm} % NON TOCCARE
\setlength{\voffset}{-1.5cm} % NON TOCCARE
\setlength{\textheight}{700pt} % NON TOCCARE
\setlength{\parindent}{0pt} % INDENTAZIONE

\lhead{\nomedoc\  (ver. \versione)}
\chead{}
\rhead{\includegraphics[height=1cm]{img/netmus.png}}
\lfoot{\includegraphics[height=0.8cm]{img/logo.png}}
\cfoot{}
\rfoot{\thepage}

% \usepackage{listings}   per codice sorgente

\author{VT.G - Valter Texas Group}

\begin{document}

\pagenumbering{Roman} % INIZIO NUMERAZIONE ARABA

\vspace*{1cm}
\begin{center}

\includegraphics[width=9cm]{img/logo.png}\\
\vspace{0.5cm}
\begin{LARGE} \sca{VT.G - Valter Texas Group} \end{LARGE}\\
\vspace{0.5cm}
\begin{Large}
\emph{valtertexasgroup@googlegroups.com} \end{Large}\\
\vspace*{1cm} \includegraphics[width=5cm]{img/netmus.png}\\
\vspace{0.5cm}
\begin{Large} \sca{\nomedoc} \end{Large}\\
\vspace{1cm}
\begin{Large} \emph{Ingegneria del Software A.A. 2010-2011} \end{Large}\\
\end{center}
\vspace{1cm}

% INFORMAZIONI DOCUMENTO
\begin{center}
\begin{tabular}{r|l}
\hline & \\
\bo{Nome} & \nomefile \\
\bo{Versione attuale} & \versione \\
\bo{Data creazione} & \datacreazione \\
\bo{Data ultima modifica} & \datamodifica \\
\bo{Stato} & \stato \\
\bo{Uso} & \uso \\
\bo{Redazione} & \redazione \\
\bo{Verifica} & \verifica \\
\bo{Approvazione} & \approvazione \\
\bo{Distribuzione} & \distribuzione \\
& \\\hline
\end{tabular}
\end{center}
\newpage

\chapter*{Sommario}
\thispagestyle{fancy}
(da scrivere)

\newpage
% REGISTRO MODIFICHE
\section*{Registro delle modifiche}

\begin{longtable}{|p{0.13\textwidth}|c|p{0.2\textwidth}|p{0.46\textwidth}|}
\hline
\rowcolor{orange} \bo{Data} & \bo{Versione} & \bo{Autore} & \bo{Descrizione} \\
\hline
\endhead
\hline
\endfoot
24/02/2011 & 0.1 & Daminato Simone & Stesura preambolo.\\ 
24/02/2011 & 0.1 & Mandolo Andrea & Stesura prima versione del Manuale Utente.\\

\end{longtable}

% INDICE
\tableofcontents

\chapter{Introduzione}
\thispagestyle{fancy} % serve perche' nelle pagine di inizio Chapter esca header e footer
\pagenumbering{arabic} % INIZIO NUMERAZIONE NORMALE
\rfoot{\thepage\ di \pageref{TotPages}}
Questo manuale \`e destinato agli utilizzatori di \co{NetMus}, e comprende una
descrizione del prodotto e le istruzioni per l'uso, oltre a due appendici
riportanti gli eventuali messaggi di errore e il glossario.

\section{Definizione dell'utente del prodotto}
\co{NetMus} pu\`o essere utilizzato da chiunque: sono richieste solo poche
nozioni di base, ovvero come navigare in internet utilizzando un browser web.

\section{Come leggere il manuale}
Questo manuale presenta il prodotto \co{NetMus} e descrive le sue funzionalit\`a
e l'approccio dell'utente al suo utilizzo. In particolare, verranno illustrate
le azioni che l'utente pu\`o effettuare con questo software e come risolvere, se
possibile, gli eventuali problemi riscontrabili durante il suo l'utilizzo.

\section{Documenti utili}
I documenti a cui ci si � riferiti per stendere il presente documento, e che
possono tornare utili per la lettura dello stesso, sono:
\begin{itemize}
  \item Analisi dei requisiti v3.0
  \item Norme di progetto v3.0
  \item Piano di progetto v3.0
  \item Specifica tecnica v2.0
  \item Definizione del prodotto v1.0
  \item Piano di qualifica v3.0
  \item Verbale 1 v1.0
  \item Capitolato C2 - NetMus del corso di Ingegneria del Software, A.A.
2010/11 (disponibile all'indirizzo web
\url{http://www.math.unipd.it/~tullio/IS-1/2010/Progetto/NetMus.pdf})
\end{itemize}

\section{Come riportare problemi e malfunzionamenti}
Per segnalare problemi e malfunzionamenti \`e disponibile un servizio online a
questo indirizzo: \url{http://code.google.com/p/netmus/issues/list}.\\
\`E sufficiente cliccare in alto a sinistra su \emph{New issue}, e nel modulo
compilare i campi col titolo e la descrizione del problema.
Come aiuto per descrivere il problema, consigliamo di selezionare il template
\emph{Defect report from user}, che contiene una traccia che aiuta ad inserire
tutte le informazioni necessarie.

\chapter{Descrizione generale}
\thispagestyle{fancy}
Oggigiorno, la maggior parte della musica che ascoltiamo \`e in formato mp3: \`e
per questo che nasce \co{NetMus}, uno strumento per gestire e condividere le
nostre canzoni preferite, che sfrutta le nuove tecnologie, come il cloud
computing.\\

\co{NetMus} permette agli utenti di avere una
libreria virtuale online con tutte le canzoni che si ascoltano abitualmente, in modo da
poterle ascoltare ovunque ci si trovi, e condividerle con altri utenti.\\

Oltre alla comodit\`a di poter accedere alla propria libreria musicale, il
sistema \co{NetMus} svolge il ruolo anche da semplice ``social network'' che
permette di interagire con gli altri utenti iscritti. \`E quindi possibile
visualizzare e riprodurre i cataloghi multimediali degli altri utenti, oltre che
creare una lista di amici con qualunque utente si voglia.\\

Il prodotto \`e molto facile da utilizzare, una volta effettuato l'accesso
a \co{NetMus} basta collegare il proprio lettore mp3 e lasciarlo lavorare in
automatico, oppure indicargli in che cartella si tengono le canzoni: tutte le 
infomazioni verranno estratte, analizzate, completate e
inserite nel nostro profilo, pronte per essere utilizzate.\\

Tra le possibilit\`a offerte, troviamo sia lo streaming audio che quello
video, il completamento automatico delle informazioni dei brani, le playlist, e
molto altro.\\

Il tutto \`e implementato con un'interfaccia grafica molto semplice che fornisce
tutte le funzionalit\`a necessarie, ma nonostante la sua semplicit\`a ha
un'impatto visivo notevole.
\chapter{Istruzioni per l'uso}
\thispagestyle{fancy}

\section{Descrizione funzionale}

\subsection{Registrazione a NetMus}

Per iniziare ad utilizzare NetMus collegarsi all'indirizzo
http://netmus2.appspot.com . \\
Qui vi verr\`a presentata la pagina di login per
accedere al sistema. Se possedete gi\`a un account Google, potete loggarvi
direttamente con i dati dell'account di Google, dato che il sistema permette
l'accesso anche con questo account. Se invece non siete utenti Google e questo
\`e il primo accesso, \`e necessario che vi registriate al sistema. La
registrazione cos\`i come l'utlizzo del sistema \co{NetMus} sono completamente
gratuiti.\\ 
La registrazione richiede un' email valida e una password di lunghezza maggiore
di 5 caratteri. Vi verr\`a inviata una mail all'indirizzo indicato che richiede
conferma per l'attivazione dell'account.\\
Una volta attivato il vostro account, avrete libero accesso al sistema
\co{NetMus}.

\subsection{Primo accesso a NetMus}

Al vostro primo accesso al sistema, l'interfaccia che vi verr\`a presentata
sar\`a strutturata in questo modo.\\
Un men\`u a sinistra presenta il logo del sistema, la mail con cui siete
registrati che viene utilizzato come vostro nickname, e il numero di brani
presente nel vostro catalogo, che sar\`a impostato a zero. Questo \`e il men\`u
delle playlist, infatti qui verranno visualizzate tutte le vostre playlist
create e permette di crearne di nuove cliccando il tasto ``+'' presente.\\
\\
La sezione centrale dell'interfaccia \`e la pi\`u importante, \`e infatti il
nostro catalogo multimediale con player annesso e tasti per svariate azioni.\\
\\
All'estrema sinistra della pagina \`e presenta la barra per la scannerizzazione
dei dispositivi di archiviazione di massa, chiamata pi\`u semplicemente ``DEVICE
SCANNER BAR''. Tale barra \`e visibile nella sua completezza solo al passaggio
sopra di essa del mouse. La barra permette la scansione manuale e la
visualizzare l'avanzamento dell'analisi ed estrazione automatica o manuale dei
file mp3 da un dispositivo o una diretory selezionata.

\subsection{Iniziare ad utilizzare NetMus}

Per poter inziare da subito a sfruttare la capacit\`a principale di \co{NetMus},
ossia quella di libreria musicale, \`e necessario inserire un dispositivo di
archiviazione di massa nella porta USB del proprio PC.\\
In alternativa \`e possibile scannerizzare qualunque directory del proprio PC
selezionandola cliccando il tasto ``Scan Folder'' per la scannerizzazione
manuale.\\
Quando parte l'analisi, nella DEVICE SCANNER BAR viene visualizzato
l'avanzamento del processo, che mostra il numero di file analizzati fino al
messaggio ``Sending Done'' che indica che i file sono stati inviati alla
libreria musicale.\\
A questo punto una volta inviati, i file dovrebbero comparire sul catalogo
multimediale.\\
A questo punto \`e possibile inziare a sbizzarrirsi con le funzionalit\`a di
netmus. Le funzionalit\`a per comodit\`a di impostazione visiva e sopratutto di
ricerca, vengono elencate in sottoparagrafi.

\subparagraph{Ascoltare una canzone in streaming}
Come prima funzionalit\`a inseriamo la descrizione di quella pi\`u interessante
del nostro prodotto, ossia la riproduzione delle tracce in streaming.\\
Per riprodurre una qualsiasi canzone del proprio catalogo, basta selezionarla
nel catalogo, quindi cliccare sul tasto play del player in alto, o del player
YouTube in basso.\\ 
Di fatto l'operazione svolta dai due player \`e la stessa, ma
si differenziano per il fatto che il player superiore permette riproduzione,
pausa, passaggio al brano precedente e successivo del catalogo, mentre il player
inferiore si trasforma in un riproduttore di video in streaming, in cui viene
riprodotto il video musicale della canzone trovato da youtube.\\
Inoltre tale player evidenzia il link di YouTube a cui \`e stato trovato il
brano, e con un click il browser ci reindirizza a tale link.

\subparagraph{Creare Playlist}
La creazione di una playlist \`e un'operazione molto semplice e molto comoda per
poter avere una sequenza delle canzoni preferite.\\
Per creare una playlist basta cliccare sul tasto ``+'' nel men\`u a sinistra di
fianco alla voce ``PLAYLIST''. Subito sotto comparir\`a una line edit in cui
inserire il titolo della playlist. Una volta inserito il titolo e premuto invio,
la playlist sar\`a stat creata.\\
Ora non resta che inserire i brani all'interno della propria playlist per
ottenere la sequenza di canzoni desiderata. Per fare ci\`o basta cliccare sul
nome della playlist, si aprir\`a una finestrella (vuota per il momento) in cui
saranno presenti tutti i brani della playlst. Per inserire un brano basta
selezionarlo e successivamente cliccare nella finestrella della playlist il
simbolo ``+" verde.\\
La rimozione di un brano avviene in modo analogo; basta selezionare
nell'elenco dei brani della playlist il brano da rimuovere e successivamente
cliccare il simbolo ``-'' rosso.\\
Da notare che all'interno della finestrella con l'elenco dei brani della
playlist, una volta selezionato un brano dal catalogo multimediale o dalla
playlist le azioni di aggiunta di un brano sono scritte in verde, le azioni di
rimozione di un brano in rosso.\\
\\
Infine per elimare una playlist basta selezionarla e cliccare la ``X'' rossa che
si trova nella finestrella della playlist in alto a destra.

\subparagraph{Visualizzare le informazioni riguardanti un brano}
Per visualizzare ulteriori informazioni riguardanti un brano basta fare un
doppio click sopra lo stesso. Si aprir\`a una finestrella con l'elenco delle
informazioni riguardanti un brano.\\
Tra queste informazioni elenchiamo titolo, autore, album, genere, anno,
compositore, traccia, valutazione e copertina.\\
Per quanto riguarda la valutazione verr\`a spiegata pi\`u a fondo in seguito.\\
Le copertine invece vengono recuperate durante la memorizzazione del brano sul
catalogo.

\subparagraph{Eliminare un brano dal catalogo}
Per eliminare un brano dal proprio catalogo multimediale basta cliccarci sopra
due volte e fare apparire la finestrella con le informazioni dettagliate. A
questo punto cliccare la ``X'' rossa come per le playlist, e il brano verr\`a
immediatamente eliminato dall'elenco.

\subparagraph{Ricercare un brano sul catalogo}

Per la ricerca di un brano all'interno del catalogo \`e stata implementata una
form apposta. La form, visualizzata nella pagina in alto a destra, ci permette
di ricercare titolo, album o artista di qualunque brano del catalogo.\\
La ricerca \`e immediata durante la digitazione, e i risultati vengono mostrati
direttamente sull'interfaccia principale del catalogo multimediale.

\subparagraph{Modificare le informazioni riguardanti un brano}

\subparagraph{Modificare le informazioni de profilo (anche cambio password)}

Nel sistema \co{NetMus} \`e possibile compilare ulteriori campi riguardanti il
proprio profilo utente.\\
Tale operazione \`e possbile cliccando sul tasto verde ``Account''. Si aprir\`a
un pop-up con l'elenco dei campi che \`e possibile compilare. Le informazioni
dovranno essere inserite in tali form.
Le informazioni che \`e possibile aggiungere sono nickname, nome, cognome,
nazionalit\`a, sesso e una breve descrizione di s\`e. Da questa interfaccia \`e
anche possibile cambiare la propria password.\\
Una volta inserite le nuove informazioni, cliccare su salva modifiche per
salvare quanto editato, mentre cliccare sulla ``X'' rossa per annullare.

\subparagraph{Dare un voto ad un brano}

Il sistema \co{NetMus} offre anche la possibilit\`a di valutare un brano su una
scala da 0 a 5 stelline.\\
Da notare che il voto espresso per il brano, verr\`a poi visualizzato nelle
informazioni del brano ma sottoforma di media di tutti i voti ricevuti dai
possessori all'interno del sistema \co{NetMus} di quella stessa canzone.\\
Per valutare un brano basta selezionarlo dal catalogo con un click, e
successivamente cliccare sul numero di stelline desiderato in basso. Il voto
verr\`a automaticamente registrato tra le informazioni del brano in questione.

\subparagraph{Esportare il proprio catalogo multimediale in PDF}

Questa funzione originale \`e stata inserita per permettere l'esportazione del
catalogo multimediale in un documento PDF, pi\`u adeguato alla stampa.\\
Per procedere con l'esportazione a PDF \`e necessario cliccare sul tasto in
basso a destra ``Esporta Lista''. A pressione avvenuta, il sistema
auto-generer\`a un PDF con l'elenco dei brani del catalogo, e aprir\`a una
finestra per far selezionare all'utente la directory su cui salvare il file PDF
creato.

\subparagraph{Visualizzare le statistiche del sistema e del proprio account}

Per visualizzare le statistiche del proprio sistema basta cliccare sul tasto
``Statistiche'' in basso a destra. Verranno dunque visualizzate su un pop-up le
statistiche riguardanti il sistema netmus(canzone pi\`u gettonata, numero di
utenti, ecc..) e le statistiche riguardanti il proprio account (canzone pi\`u
ascoltata, genere preferito, autore preferito, numero di amici ecc..)


\section{Azioni richieste/permesse}

\section{Errori probabili e cause possibili}

\subparagraph{Il dispositivo USB non viene analizzato}
\begin{itemize}
  \item controllare che la DEVICE SCANNER BAR non sia disabilitata. La DEVICE
  SCANNER BAR \`e disabilitata quando nell'angolo in basso vi \`e la scritta
  verde ``Enable''.
\end{itemize}

\subparagraph{Alcuni brani analizzati non sono stati caricati nel catalogo}
\begin{itemize}
  \item controllare che non fossero gi\`a stati analizzati e inseriti nel
  catalogo in una analisi precedente. Per essere sicuri, controllare il log
  presente all'interno del dispositivo in cui \co{NetMus} ha fatto la scansione.
  \item i brani non hanno i tag necessari completi. Se un brano non ha i
  tag artista titolo e album completi il brano non viene catalogato perch\`e non
  c'\`e modo di riconoscere quale brano sia.
  \item i brani contengono tag con codifica diversa dalla ISO-8859-1. Con i tag
  artista, titolo e album con caratteri diversi dalla codifica ISO-8859-1 i file
  vengono ignorati poich\`e \`e impossibile catalogarli dato che \`e come se
  avessero i tag principali vuoti.
\end{itemize}

\subparagraph{La canzone in streaming da youtube non \`e quella corretta}
\begin{itemize}
  \item purtroppo la ricerca su youtube non va sempre a buon fine. E possibile
  cliccare sul tasto ``Wrong Song'' per passare al vide successivo dato dai
  risultati della ricerca su youtube di tale canzone.
  \item i tag di tale canzone sono sbagliati e quindi la ricerca di tale canzone
  produce risulati errati. Modificare le informazioni di tale canzone per
  risolvere il problema
\end{itemize}


\subparagraph{Password dimenticata}
\begin{itemize}
  \item Cazzi tui, ebete.
\end{itemize}


\appendix % inizio appendice
\chapter{Messaggi di errore e loro significato}
\thispagestyle{fancy}

\chapter{Glossario}
\thispagestyle{fancy}

\end{document}
