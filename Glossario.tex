
\newcommand{\nomedoc}{Glossario}
\newcommand{\versione}{1.1}
\newcommand{\nomefile}{Glossario-\versione.pdf}
\newcommand{\datacreazione}{7 Dicembre 2010}
\newcommand{\datamodifica}{12 Gennaio 2011}
\newcommand{\stato}{formale}
\newcommand{\uso}{esterno}
\newcommand{\redazione}{Caputo Cosimo}
\newcommand{\verifica}{Mandolo Andrea}
\newcommand{\approvazione}{Lovato Daniele}
\newcommand{\distribuzione}{
VT.G \\
& Prof. Vardanega Tullio\\
& Prof. Cardin Riccardo}

% FUNZIONI TIPOGRAFICHE
\newcommand{\co}{\texttt} % courier
\newcommand{\bo}{\textbf} % bold
\newcommand{\pr}{\par\medskip} % paragrafo spaziato
\newcommand{\sca}{\textsc} % small caps

\documentclass[a4paper,12pt]{report}
% 10pt,11pt,12pt
% titlepage, notitlepage -> per dare inizio o no ad una nuova pagina dopo titolo
% twoside -> per dire se fronte-retro
\usepackage[latin1]{inputenc}
% per caratteri accentati
\usepackage[italian]{babel}
% per regole sintattiche italiane
\usepackage[bookmarks=true, pdfborder={0 0 0 0}]{hyperref}
% per collegamenti ipertestuali
\usepackage{graphicx}
% per inserimento immagini

% \usepackage{enumerate}
% per personalizzare elenchi puntati

\usepackage[hmargin=2cm]{geometry} %margine 2 cm
%\geometry{options varie}

% comandi per gestire meglio header e footer
\usepackage{fancyhdr}  % header e footer
\usepackage{totpages}
\pagestyle{fancy}
\renewcommand{\headrulewidth}{0.4pt}
\renewcommand{\footrulewidth}{0.4pt}

\setlength{\headheight}{1.2cm} % NON TOCCARE
\setlength{\voffset}{-1.5cm} % NON TOCCARE
\setlength{\textheight}{700pt} % NON TOCCARE
\setlength{\parindent}{0pt} % INDENTAZIONE

\lhead{\nomedoc\  (ver. \versione)}
\chead{}
\rhead{\includegraphics[height=1cm]{img/netmus.png}}
\lfoot{\includegraphics[height=0.8cm]{img/logo.png}}
\cfoot{}
\rfoot{\thepage}

% \usepackage{listings}   per codice sorgente

\author{VT.G - Valter Texas Group}

\begin{document}

\pagenumbering{Roman} % INIZIO NUMERAZIONE ARABA

\vspace*{1cm}
\begin{center}

\includegraphics[width=9cm]{img/logo.png}\\
\vspace{0.5cm}
\begin{LARGE} \sca{VT.G - Valter Texas Group} \end{LARGE}\\
\vspace{0.5cm}
\begin{Large}
\emph{valtertexasgroup@googlegroups.com} \end{Large}\\
\vspace*{1cm} \includegraphics[width=5cm]{img/netmus.png}\\
\vspace{0.5cm}
\begin{Large} \sca{\nomedoc} \end{Large}\\
\vspace{1cm}
\begin{Large} \emph{Ingegneria del Software A.A. 2010-2011} \end{Large}\\
\end{center}
\vspace{1cm}

% INFORMAZIONI DOCUMENTO
\begin{center}
\begin{tabular}{r|l}
\hline & \\
\bo{Nome} & \nomefile \\
\bo{Versione attuale} & \versione \\
\bo{Data creazione} & \datacreazione \\
\bo{Data ultima modifica} & \datamodifica \\
\bo{Stato} & \stato \\
\bo{Uso} & \uso \\
\bo{Redazione} & \redazione \\
\bo{Verifica} & \verifica \\
\bo{Approvazione} & \approvazione \\
\bo{Distribuzione} & \distribuzione \\
& \\\hline
\end{tabular}
\end{center}
\newpage

% REGISTRO MODIFICHE
\section*{Registro delle modifiche}

\begin{longtable}{|p{0.13\textwidth}|c|p{0.2\textwidth}|p{0.46\textwidth}|}
\hline
\rowcolor{orange} \bo{Data} & \bo{Versione} & \bo{Autore} & \bo{Descrizione} \\
\hline
\endhead
\hline
\endfoot
12/01/2011 & 1.1 & Mandolo Andrea & Modificato layout Registro delle
modifiche.\\
\hline
19/12/2010 & 1.0 & Lovato Daniele & Validazione per consegna RR.\\
\hline
8/12/2010 & 0.4 & Mandolo Andrea & Correzione ortografica e sintattica.\\
&&&Commentate lettere senza termini.\\
&&&Aggiunto termine ``Propriet\`a emergenti''.\\

&&&Ordinamento termini in ordine alfabetico.\\
\hline
13/12/2010 & 0.3 & Caputo Cosimo & Aggiunti nuovi termini,\\
&&&apportate correzioni.\\
\hline
10/12/2010 & 0.2 & Caputo Cosimo & Aggiunti nuovi termini.\\
\hline
07/12/2010 & 0.1 & Caputo Cosimo & Stesura prima versione del  documento.\\
\end{longtable}

\newpage
\pagenumbering{arabic}
\section*{Lista termini}

\subsection*{\huge{A}}

\subsubsection*{Analisi dinamica:}
Tipo di analisi basata sulla compilazione del codice, effettuata su singoli
componenti o sull'intero codice.

\subsubsection*{Analisi statica:}
Tipo di analisi che non comporta alcuna esecuzione, si applica sui
documenti attraverso la lettura mirata (inspection) oppure integrale
(walkthrough), oppure sul codice sorgente analizzando riga per riga (desk-check).

\subsubsection*{Applicazione desktop:}
Applicazione software che non necessita di una connessione internet per operare.

\subsubsection*{Applicazione web:}
Applicazione software che non necessita di essere installata sul dispositivo
hardware dell'utente (in genere \`e sita su uno o pi\`u server remoti) ed \`e accessibile tramite browser.

\subsection*{\huge{B}}

\subsubsection*{Branch:} 
Flusso logico che permette a pi\`u persone di lavorare su un unico file o
progetto in maniera indipendente.

\subsection*{\huge{C}}

\subsubsection*{Cloud computing:} Insieme di tecnologie informatiche che
permettono l'utilizzo di risorse hardware (storage, CPU) o software distribuite in remoto.

\subsubsection*{Codice sorgente:} Testo di un programma, scritto in un
linguaggio di programmazione ad alto livello. Il codice sorgente deve essere
opportunamente elaborato per arrivare ad un programma eseguibile.

\subsubsection*{Commit} Comando che permette di creare un nuovo changeset in
un repository e di identificarlo con una descrizione.

\subsection*{\huge{D}}

\subsubsection*{Diagrammi di Gantt:} Strumento per la pianificazione temporale e
gestione delle risorse di un progetto ideato nel 1917 dall'ingegnere
statunitense Henry Laurence Gantt.

\subsubsection*{Disco rigido (HDD):} Dispositivo di memoria di massa che
utilizza uno o pi\`u dischi magnetici per l'archiviazione dei dati.

%\subsection*{\huge{E}}

\subsection*{\huge{F}}
\subsubsection*{Framework:}  Struttura di supporto su cui un software pu\`o
essere organizzato e progettato. Alla base di un framework c'\`e sempre una
serie di librerie di codice utilizzabili con uno o pi\`u linguaggi di
programmazione, spesso corredate da una serie di strumenti di supporto allo
sviluppo del software, come ad esempio un IDE, un debugger, o altri strumenti
ideati per aumentare la velocit\`a� di sviluppo del prodotto finito.

\subsection*{\huge{G}}

\subsubsection*{Goole App Engine (GAE):} Piattaforma Google che consente di
progettare ed eseguire applicazioni, servendosi della scalabilit\`a offerta
dall'infrastruttura web che Google stessa utilizza per le sue applicazioni:
migliaia di server sparsi per il mondo che distribuiscono le copie
dell'applicazione prodotta in modo che qualunque server possa essere in grado di
rispondere ad una richiesta, o perch\'e poco carico in quel momento o perch\'e
si trova geograficamente vicino all'utente, secondo il paradigma del ``cloud computing''.

\subsubsection*{Google Code:} Servizio di repository gratuito di
propriet\`a Google che garantisce velocit\`a e semplicit\`a di gestione.

\subsubsection*{Google DataStore:} Database non relazionale fornito con la
piattaforma GAE, pensato con lo scopo di avere massima scalabilit\`a per
lavorare in ambiente distribuito.

\subsubsection*{Google Groups:} Servizio web che consente di gestire (e
archiviare sotto forma di blog) una mailing list e che fornisce varie funzioni
di comunicazione e collaborazione con i membri iscritti ed autorizzati.

\subsubsection*{Google Web Toolkit (GWT):} Set di tools open source che
permette agli sviluppatori web di creare e manutenere complesse applicazioni front-end
Javascript scritte in Java. Il codice sorgente Java pu\`o essere compilato su
qualsiasi piattaforma con i file Ant inclusi. \`E distribuito sotto licenza
Apache. Punti di forza di GWT sono la riusabilit\`a del codice, la possibilit\`a di
realizzare pagine web dinamiche mediante le chiamate asincrone di AJAX, la
gestione delle modifiche, il bookmarking, l'internazionalizzazione e la
portabilit\`a fra differenti browser.

\subsection*{\huge{H}}
\subsubsection*{HTML5:} Linguaggio di markup per la progettazione delle pagine
web attualmente in fase di definizione (draft) presso il World Wide Web Consortium.

\subsection*{\huge{I}}
\subsubsection*{Internet:}  ``Rete di computer mondiale (contrazione della
locuzione inglese Interconnected Networks, ovvero Reti Interconnesse) ad accesso pubblico
attualmente rappresentante il principale mezzo di comunicazione di massa.
Chiunque infatti disponga di un computer e degli opportuni software,
appoggiandosi a un Internet service provider che gli fornisce un accesso a
Internet attraverso una linea di telecomunicazione dedicata (ADSL, HDSL, VDSL,
GPRS, HSDPA, ecc.) o una linea telefonica della Rete Telefonica Generale (POTS,
ISDN, GSM, UMTS, ecc.), pu\`o accedere a Internet ed utilizzare i suoi servizi.
Ci\`o \`e reso possibile da una suite di protocolli di rete chiamata ``TCP/IP'' dal
nome dei due principali, il TCP e l'IP, la ``lingua'' comune con cui i computer
di Internet si interconnettono e comunicano tra loro indipendentemente dalla loro
architettura hardware e software.'' \\
\emph{Wikipedia, L'enciclopedia libera,
\url{http://it.wikipedia.org/wiki/Internet} (controllata il 19 Dicembre, 2010).}

\subsubsection*{iTunes:} Applicazione sviluppata e distribuita da Apple Inc. per
riprodurre e organizzare file multimediali, permettendo l'acquisto online di canzoni, video e
film attraverso il servizio iTunes Store.

\subsection*{\huge{J}}
\subsubsection*{Java:} ``Linguaggio di programmazione orientato agli oggetti,
derivato dallo Smalltalk (anche se ha una sintassi simile al C++) e creato da
James Gosling e altri ingegneri di Sun Microsystems. La piattaforma di
programmazione Java \`e fondata sul linguaggio stesso, sulla Macchina virtuale
Java (Java Virtual Machine o JVM) e sulle API Java. Java attualmente \`e un
marchio registrato di Oracle.''\\
\emph{Wikipedia, L'enciclopedia libera,
\url{http://it.wikipedia.org/wiki/Java_(linguaggio)} (controllata il 19
Dicembre, 2010).}

\subsubsection*{JavaFX:} Ampliando le potenzialit\`a di Java, consente agli
sviluppatori di utilizzare qualsiasi libreria Java all'interno delle applicazioni JavaFX.
Caratteristiche salienti:
\begin{itemize}
\item Consente agli utenti di visualizzare le applicazioni JavaFX in un browser
o di uscire dagli schemi del browser trascinandole sul desktop 
\item Offre un flusso di lavoro efficiente dalla progettazione allo sviluppo
grazie a Project Nile: i progettisti hanno la possibilit\`a di lavorare con gli strumenti
che prediligono pur collaborando con Web scripter che utilizzano l'ambiente
integrato di sviluppo NetBeans con JavaFX
\item Consente agli sviluppatori di integrare risorse Web di grafica vettoriale,
animazione, audio e video per creare applicazioni interattive e coinvolgenti
\end{itemize}

\subsubsection*{JVM:} Macchina virtuale che esegue i programmi Java in byte-code
(.class), risultato della compilazione di file sorgenti Java (.java).

%\subsection*{\huge{K}}

\subsection*{\huge{L}}
\subsubsection*{LaTex:} Linguaggio di formattazione testi utilizzato
soprattutto in ambiente scientifico, e rimane tutt'ora uno dei pochi programmi
di tipocomposizione ad integrare un formattatore matematico totalmente automatizzato.

\subsubsection{Linux (o GNU/Linux):} ``Sistema operativo open-source di
tipo Unix (o unix-like) costituito dall'integrazione del kernel Linux con
elementi del sistema GNU e di altro software sviluppato e distribuito con
licenza GNU GPL o con altre licenze libere. Linux, in realt\`a, \`e il nome del
kernel sviluppato da Linus Torvalds a partire dal 1991 che, integrato con i
componenti gi\`a realizzati dal progetto GNU (compilatore gcc, libreria Glibc e
altre utility) e da software di altri progetti, \`e stato utilizzato come base per
la realizzazione dei sistemi operativi Open-Source e delle distribuzioni che
vengono normalmente identificate con lo stesso nome. Secondo Richard Stallman,
fondatore del progetto GNU, e secondo la Free Software Foundation, la dicitura
Linux (senza prefisso ``GNU/'') per l'intero sistema operativo sarebbe erronea, in
quanto il nome Linux \`e attribuibile al solo kernel e il sistema, strutturato a
partire dai componenti dell'originale progetto GNU, dovrebbe pi\`u propriamente
chiamarsi GNU/Linux. Secondo altri e secondo l'uso della maggior parte degli
utenti e degli sviluppatori e delle societ\`a coinvolti nello sviluppo del sistema
operativo e del software ad esso collegato, il nome Linux \`e ormai divenuto
sinonimo di sistema ``Linux based'', cio\`e di sistema basato sul kernel
Linux.''\\ \emph{Wikipedia, L'enciclopedia libera,
\url{http://it.wikipedia.org/wiki/Linux} (controllata il 19 Dicembre, 2010).}

% Molto conosciuto nell'uso server, Linux gode del supporto di societ\`a come
%IBM, Sun Microsystems, Hewlett-Packard, Red Hat e Novell ed \`e usato come
%sistema operativo su una gran variet\`a di hardware; dai computer desktop ai
%supercomputer, fino a sistemi embedded come cellulari e palmari, e ai netbook.
%Sebbene non sia insostituibile per questo scopo, \`e anche il sistema operativo
%pi\`u comunemente usato per eseguire Apache, MySQL e PHP, i software alla base
%della maggior parte dei server web di tutto il mondo. Le iniziali di questi tre
%progetti, insieme all'iniziale della parola Linux, hanno dato origine
%all'acronimo LAMP. Con l'evoluzione di ambienti desktop come KDE e GNOME, il
%sistema offre una interfaccia grafica simile a quella di Microsoft Windows o di
%Mac OS X, pi\`u vicina alle esigenze degli utenti meno esperti, rendendo il
%passaggio da un sistema all'altro meno traumatico.

\subsection*{\huge{M}}
\subsubsection*{MacOSX:} Sistema operativo sviluppato da Apple Inc. per i
computer Macintosh, nato nel 2001 per combinare le note caratteristiche
dell'interfaccia utente del Mac OS originale con l'architettura di un sistema
operativo di derivazione Unix (Debian).

\subsubsection*{Mercurial:} Software per il versionamento con il quale ogni
sviluppatore possiede una copia completa del repository ed ogni working copy
contiene l'intero storico delle revisioni.

\subsubsection*{Merge:} Operazione di sincronizzazione utile alla
risoluzione dei conflitti tra diverse revisioni di un file in un repository.
Pu\`o essere effettuato entro certi limiti in maniera automatica (se le
differenze sono semplicemente nuove parti di testo) oppure manualmente.

%\subsection*{\huge{N}}

\subsection*{\huge{O}}

\subsubsection*{Open-Source:} Dicesi di un sofware i cui autori ne permettono e
ne favoriscono il libero studio e l'apporto di modifiche da parte di altri programmatori indipendenti.
\subsection*{\huge{P}}

\subsubsection*{Pdf (Portable Document Format):} Formato di file basato su un
linguaggio di descrizione di pagina sviluppato da Adobe Systems nel 1993 per rappresentare
documenti in modo indipendente dall'hardware e dal software utilizzati per generarli o per visualizzarli.

\subsubsection*{Propriet\`a emergenti:} Si tratta di propriet\`a del sistema
softaware nel suo complesso e non relative a singole componenti. Sono molto
delicate poich\'e conseguenza dell'interazione tra le varie componenti.\\
Sono visibili e misurabili solo dopo l'integrazione del sistema.

\subsubsection*{Pull:} Operazione di sincronizzazione che consiste
nell'aggiornare i contenuti locali di un progetto in base alla versione del repository.

\subsubsection*{Push:} Operazione di sincronizzazione che consiste
nell'aggiornare il contenuto del repository (eventualmente dopo il merge delle modifiche).


\subsection*{\huge{Q}}
\subsubsection*{Query:} Termine utilizzato per indicare l'interrogazione di un
database nel compiere determinate operazioni (inserimento dati, cancellazione
dati, ecc.) da eseguire in uno o pi\`u database.

\subsection*{\huge{R}}

\subsubsection*{Repository:} Ambiente per l'immagazzinamento, la gestione e la
condivisione di cospicue quantit\`a di dati. Esso permette l'aggiunta, modifica
e rimozione dei dati tramite apposite primitive (push, pull, merge) e di
tracciarne le modifiche tramite note (commit).

\subsubsection*{RIA (Rich Internet Application):} Applicazioni web che
possiedono le caratteristiche e le funzionalit\`a delle applicazioni desktop, senza per\`o
necessitare dell'installazione sul disco fisso.
  
\subsection*{\huge{S}}
\subsubsection*{Sistema operativo:} Particolare software che, installato su un
sistema di elaborazione, permetta l'utilizzo di software pi\`u specifici senza che
essi debbano preoccuparsi dell'architettura hardware sottostante.

\subsubsection*{Spring:} Framework Open-Source per lo sviluppo di applicazioni
su piattaforma Java. \`E nato con l'intento di gestire la complessit\`a nello
sviluppo di applicazioni enterprise.

\subsubsection*{SQL (Structured Query Language):} Linguaggio di
interrogazione per database progettato per leggere, modificare e gestire dati
memorizzati in un sistema basato sul modello relazionale, per creare e
modificare schemi di database, per creare e gestire strumenti di controllo ed
accesso ai dati.

\subsection*{\huge{T}}
\subsubsection*{Ticketing:} Meccanismo di gestione di progetti condivisi tramite
tickets. Un ticket corrisponde ad un'attivit\`a, come un bug o un suggerimento
ed \`e possibile assegnargli una priorit\`a, una milestone (ovvero una ipotetica
data per la sua risoluzione), attribuirgli dei dettagli ed un componente del
team che si occuper\`a di gestire l'elemento. Lo status del ticket consente di
verificare se il ticket \`e in lavorazione, in standby o completato.

\subsection*{\huge{U}}
\subsubsection*{UML:} Acronimo di Unified Modeling Language, \`e un linguaggio
di modellazione e specifica basato sul paradigma object-oriented, utilizzato per
descrivere soluzioni analitiche e progettuali in modo sintetico e comprensibile.

\subsubsection*{USB (Universal Serial Bus):} Standard di comunicazione seriale
che consente di collegare diverse periferiche ad un computer. \`e stato progettato
per consentire a pi\`u periferiche di essere connesse usando una sola interfaccia
standardizzata ed un solo tipo di connettore, e per migliorare la funzionalit\`a
plug-and-play consentendo di collegare/scollegare i dispositivi senza dover
riavviare il computer (hot swap).

\subsubsection*{Username:} In informatica definisce il nome con il quale
l'utente viene riconosciuto da un computer, da un programma o da un server.

\subsection*{\huge{V}} 
\subsubsection*{Validazione:}
\`E un processo che serve ad accertare che il prodotto realizzato sia quello
atteso e viene applicato, a lavoro finito, sia ai documenti che alle componenti
software.

\subsubsection*{Verifica:}
\`E un processo che viene eseguito molto spesso durante lo sviluppo dell'intero
progetto ed ha il compito di accertare che il lavoro sia stato svolto nel modo
corretto e quindi non abbia introdotto difetti nel prodotto.

\subsubsection*{Versioning (Versionamento):} Pratica di sviluppo che permette di
tracciare l'evoluzione di un progetto e di coordinare le operazioni sul medesimo
al fine di permettere maggior manutenibilit\`a e riutilizzabilit\`a.

\subsection*{\huge{W}}
\subsubsection*{Windows:} Famiglia di ambienti operativi e sistemi operativi
commerciali della Microsoft Corporation dedicati ai personal computer, alle workstation e ai server.

\subsubsection*{W3C:} Associazione fondata nell'ottobre del 1994 da Tim Berners
Lee, padre del Web, al MIT (Massachusetts Institute of Technology) con lo scopo di migliorare
gli esistenti protocolli e linguaggi per il World Wide Web e di aiutare il web a
sviluppare tutte le sue potenzialit\`a.

%\subsection*{\huge{X}}

\subsection*{\huge{Y}}
\subsubsection*{YouTube:} Sito web attualmente di propriet\`a di Google Inc. che
consente la riproduzione in streaming, la pubblicazione e condivisione di video.

%\subsection*{\huge{Z}}


\end{document}

