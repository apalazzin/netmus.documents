
\newcommand{\nomedoc}{Glossario}
\newcommand{\versione}{0.1}
\newcommand{\nomefile}{Glossario\versione.pdf}
\newcommand{\datacreazione}{7 Dicembre 2010}
\newcommand{\datamodifica}{8 Dicembre 2010}
\newcommand{\stato}{formale}
\newcommand{\uso}{esterno}
\newcommand{\redazione}{Cosimo Caputo}
\newcommand{\verifica}{Federico Baron}
\newcommand{\approvazione}{Valter}
\newcommand{\distribuzione}{
VT.G \\
& Prof. Vardanega Tullio }

% FUNZIONI TIPOGRAFICHE
\newcommand{\co}{\texttt} % courier
\newcommand{\bo}{\textbf} % bold
\newcommand{\pr}{\par\medskip} % paragrafo spaziato
\newcommand{\sca}{\textsc} % small caps

\documentclass[a4paper,12pt]{report}
% 10pt,11pt,12pt
% titlepage, notitlepage -> per dare inizio o no ad una nuova pagina dopo titolo
% twoside -> per dire se fronte-retro
\usepackage[latin1]{inputenc}
% per caratteri accentati
\usepackage[italian]{babel}
% per regole sintattiche italiane
\usepackage[bookmarks=true, pdfborder={0 0 0 0}]{hyperref}
% per collegamenti ipertestuali
\usepackage{graphicx}
% per inserimento immagini

% \usepackage{enumerate}
% per personalizzare elenchi puntati

\usepackage[hmargin=2cm]{geometry} %margine 2 cm
%\geometry{options varie}

% comandi per gestire meglio header e footer
\usepackage{fancyhdr}  % header e footer
\usepackage{totpages}
\pagestyle{fancy}
\renewcommand{\headrulewidth}{0.4pt}
\renewcommand{\footrulewidth}{0.4pt}

\setlength{\headheight}{1.2cm} % NON TOCCARE
\setlength{\voffset}{-1.5cm} % NON TOCCARE
\setlength{\textheight}{700pt} % NON TOCCARE
\setlength{\parindent}{0pt} % INDENTAZIONE

\lhead{\nomedoc\  (ver. \versione)}
\chead{}
\rhead{\includegraphics[height=1cm]{img/netmus.png}}
\lfoot{\includegraphics[height=0.8cm]{img/logo.png}}
\cfoot{}
\rfoot{\thepage}

% \usepackage{listings}   per codice sorgente

\author{VT.G - Valter Texas Group}

\begin{document}

\pagenumbering{Roman} % INIZIO NUMERAZIONE ARABA

\vspace*{1cm}
\begin{center}

\includegraphics[width=9cm]{img/logo.png}\\
\vspace{0.5cm}
\begin{LARGE} \sca{VT.G - Valter Texas Group} \end{LARGE}\\
\vspace{0.5cm}
\begin{Large}
\emph{valtertexasgroup@googlegroups.com} \end{Large}\\
\vspace*{1cm} \includegraphics[width=5cm]{img/netmus.png}\\
\vspace{0.5cm}
\begin{Large} \sca{\nomedoc} \end{Large}\\
\vspace{1cm}
\begin{Large} \emph{Ingegneria del Software A.A. 2010-2011} \end{Large}\\
\end{center}
\vspace{1cm}

% INFORMAZIONI DOCUMENTO
\begin{center}
\begin{tabular}{r|l}
\hline & \\
\bo{Nome} & \nomefile \\
\bo{Versione attuale} & \versione \\
\bo{Data creazione} & \datacreazione \\
\bo{Data ultima modifica} & \datamodifica \\
\bo{Stato} & \stato \\
\bo{Uso} & \uso \\
\bo{Redazione} & \redazione \\
\bo{Verifica} & \verifica \\
\bo{Approvazione} & \approvazione \\
\bo{Distribuzione} & \distribuzione \\
& \\\hline
\end{tabular}
\end{center}
\newpage

% REGISTRO MODIFICHE
\section*{Registro delle modifiche}
\begin{tabular}{lll}
% REVISIONI DEL DOCUMENTO
% DALLA PIU\` NUOVA ALLA PIU\` VECCHIA

\bo{Data:} 07/12/2010 &
\bo{Versione:} 0.1 &
\bo{Autore:} Cosimo Caputo\\
\hline\\
\multicolumn{3}{p{470px}}{ Stesura prima versione di glossario.}\\ \\

\end{tabular}



\section*{Introduzione}

\subsection*{A}
\subsubsection*{Analisi dinamica:} Tipo di analisi basata
sull'esecuzione.
\subsubsection*{Analisi statica:} Tipo di analisi che non
prevede alcuna esecuzione, si applica sulla documentazione attraverso 
la lettura mirata (inspection) oppure integrale (walkthrough), 
oppure sul codice sorgente facendo desk-check (analisi riga-per-riga).
\subsection*{B}
\subsection*{C}
\subsection*{D}
\subsubsection*{Diagrammi di Gantt:} Strumento per la pianificazione e la
verifica della fattibilit\`a (di ordine temporale) di un progetto.
\subsection*{E}
\subsection*{F}
\subsection*{G}
\subsubsection*{Google Groups:} Software che consente di gestire e archiviare
una mailing list e fornisce un metodo per ottenere una vera comunicazione e collaborazione con i membri del gruppo.

\subsubsection*{Goole AppEngine:} Piattaforma di nuova concezione per
la creazione di applicazioni web che permette allo sviluppatore di usufruire dell'infrastruttura
Google nell'implementazione dei propri applicativi.

\subsubsection*{Google Code:} Repository gratuito che garantisce
velocit\`a e semplicit\`a di gestione.

\subsection*{H}
\subsection*{I}
\subsection*{J}
\subsubsection*{JVM:} Macchina virtuale che esegue i programmi Java in byte-code
(.class), risultato della compilazione di file sorgenti Java (.java).
\subsection*{K}
\subsection*{L}
\subsubsection*{LaTex:} Linguaggio di formattazione testi utilizzato
soprattutto in ambiente scientifico, e rimane tutt'ora uno dei pochi programmi
di tipocomposizione ad integrare un formattatore matematico totalmente automatizzato.
\subsection*{M}
\subsubsection*{Mercurial:} Software per il versionamento con il quale
ogni sviluppatore possiede una copia completa del repository ed ogni working copy contiene l'intero storico delle revisioni. \subsection*{N}
\subsection*{O}
\subsubsection*{Open-Source:} Dicesi di un sofware i cui autori ne permettono e
ne favoriscono il libero studio e l'apporto di modifiche da parte di altri programmatori indipendenti.
\subsection*{P}
\subsection*{Q}
\subsection*{R}

\subsubsection*{RR:}  Prima revisione prevista 
\begin{itemize}
\item documenti in ingresso: Capitolato d'appalto.
\item stato del prodotto in uscita: descritto. 
\end{itemize}

\subsubsection*{RPP:} Seconda revisione prevista (pu\`o essere sostituita
da RPD)
\begin{itemize}
\item documenti in ingresso: Specifica tecnica.
\item stato del prodotto in uscita: specificato. 
\end{itemize}

\subsubsection*{RPD:} Terza revisione prevista(pu\`o essere sostenuta come
seconda)
\begin{itemize}
\item documenti in ingresso: definizione di prodotto.
\item stato del prodotto in uscita: definito. 
\end{itemize}

\subsubsection*{RQ:} Quarta revisione prevista
\begin{itemize}
\item documenti in ingresso: Verifiche di qualifica.
\item stato del prodotto in uscita: qualificato. 
\end{itemize}

\subsubsection*{RA:} Quinta revisione prevista
\begin{itemize}
\item documenti in ingresso: Validazione per accettazione.
\item stato del prodotto in uscita: accettato. 
\end{itemize}

\subsubsection*{Repository:} Ambiente per l'immagazinamento, la gestione e la
condivisione di cospicue quantit\`a di dati. Esso permette l'aggiunta,
 modifica e rimozione dei dati tramite apposite primitive (push, pull,merge)
  e di tracciarne le modifiche tramite note (commit).
\subsection*{S}
\subsection*{T}
\subsubsection*{Ticketing:} Meccanismo di gestione di progetti condivisi tramite
tickets. Un ticket corrisponde ad un'attivit\`a, come un bug o un suggerimento
ed \`e possibile assegnargli una priorit\`a, una milestone (ovvero una ipotetica
data per la sua risoluzione), attribuirgli dei dettagli ed un componente del
team che si occuper\`a di gestire l'elemento. Lo status del ticket consente di
verificare se il ticket \`e in lavorazione, in standby o completato.

\subsection*{U}
\subsubsection*{UML:} Acronimo di Unified Modeling Language, \`e un linguaggio
di modellazione e specifica basato sul paradigma object-oriented, utilizzato per descrivere soluzioni analitiche e progettuali in modo sintetico e comprensibile.
\subsection*{V}
\subsubsection*{Versioning (Versionamento):} Pratica di sviluppo che permette di
tracciare l'evoluzione di un progetto e di coordinare le operazioni sul
medesimo al fine di permettere maggior manutenibilit\`a e riutilizzabilit\`a.
\subsection*{W}
\subsection*{X}
\subsection*{Y}
\subsection*{Z}




\end{document}

