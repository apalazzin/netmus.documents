\newcommand{\nomedoc}{Norme di progetto}
\newcommand{\versione}{0.5}
\newcommand{\versioneglossario}{0.1}
\newcommand{\versionenormeprogetto}{0.3}
\newcommand{\nomefile}{NormeDiProgetto-\versione.pdf}
\newcommand{\datacreazione}{7 Dicembre 2010}
\newcommand{\datamodifica}{10 Dicembre 2010}
\newcommand{\stato}{formale}
\newcommand{\uso}{interno}
\newcommand{\redazione}{Baron Federico, Palazzin Alberto}
\newcommand{\verifica}{}
\newcommand{\approvazione}{}
\newcommand{\distribuzione}{
VT.G \\
& Prof. Vardanega Tullio\\
& Prof. Cardin Riccardo }

% FUNZIONI TIPOGRAFICHE
\newcommand{\co}{\texttt} % courier
\newcommand{\bo}{\textbf} % bold
\newcommand{\pr}{\par\medskip} % paragrafo spaziato
\newcommand{\sca}{\textsc} % small caps

\documentclass[a4paper,12pt]{report}
% 10pt,11pt,12pt
% titlepage, notitlepage -> per dare inizio o no ad una nuova pagina dopo titolo
% twoside -> per dire se fronte-retro
\usepackage[latin1]{inputenc}
% per caratteri accentati
\usepackage[italian]{babel}
% per regole sintattiche italiane
\usepackage[bookmarks=true, pdfborder={0 0 0 0}]{hyperref}
% per collegamenti ipertestuali
\usepackage{graphicx}
% per inserimento immagini

% \usepackage{enumerate}
% per personalizzare elenchi puntati

\usepackage[hmargin=2cm]{geometry} %margine 2 cm
%\geometry{options varie}

% comandi per gestire meglio header e footer
\usepackage{fancyhdr}  % header e footer
\usepackage{totpages}
\pagestyle{fancy}
\renewcommand{\headrulewidth}{0.4pt}
\renewcommand{\footrulewidth}{0.4pt}

\setlength{\headheight}{1.2cm} % NON TOCCARE
\setlength{\voffset}{-1.5cm} % NON TOCCARE
\setlength{\textheight}{700pt} % NON TOCCARE
\setlength{\parindent}{0pt} % INDENTAZIONE

\lhead{\nomedoc\  (ver. \versione)}
\chead{}
\rhead{\includegraphics[height=1cm]{img/netmus.png}}
\lfoot{\includegraphics[height=0.8cm]{img/logo.png}}
\cfoot{}
\rfoot{\thepage}

% \usepackage{listings}   per codice sorgente

\author{VT.G - Valter Texas Group}

\begin{document}

\pagenumbering{Roman} % INIZIO NUMERAZIONE ARABA

\vspace*{1cm}
\begin{center}

\includegraphics[width=9cm]{img/logo.png}\\
\vspace{0.5cm}
\begin{LARGE} \sca{VT.G - Valter Texas Group} \end{LARGE}\\
\vspace{0.5cm}
\begin{Large}
\emph{valtertexasgroup@googlegroups.com} \end{Large}\\
\vspace*{1cm} \includegraphics[width=5cm]{img/netmus.png}\\
\vspace{0.5cm}
\begin{Large} \sca{\nomedoc} \end{Large}\\
\vspace{1cm}
\begin{Large} \emph{Ingegneria del Software A.A. 2010-2011} \end{Large}\\
\end{center}
\vspace{1cm}

% INFORMAZIONI DOCUMENTO
\begin{center}
\begin{tabular}{r|l}
\hline & \\
\bo{Nome} & \nomefile \\
\bo{Versione attuale} & \versione \\
\bo{Data creazione} & \datacreazione \\
\bo{Data ultima modifica} & \datamodifica \\
\bo{Stato} & \stato \\
\bo{Uso} & \uso \\
\bo{Redazione} & \redazione \\
\bo{Verifica} & \verifica \\
\bo{Approvazione} & \approvazione \\
\bo{Distribuzione} & \distribuzione \\
& \\\hline
\end{tabular}
\end{center}
\newpage

% REGISTRO MODIFICHE
\section*{Registro delle modifiche}
\begin{tabular}{lll}
% REVISIONI DEL DOCUMENTO
% DALLA PIU' NUOVA ALLA PIU' VECCHIA

\bo{Data:} 10/12/2010 &
\bo{Versione:} 0.5 &
\bo{Autore:} Palazzin Alberto\\
\hline\\
\multicolumn{3}{p{470px}}{ Sistemazione del Capitolo 3.}\\
\\

\bo{Data:} 10/12/2010 &
\bo{Versione:} 0.4 &
\bo{Autore:} Palazzin Alberto\\
\hline\\
\multicolumn{3}{p{470px}}{ Aggiunta Sommario e correzione del Capitolo 2.}\\
\\

\bo{Data:} 09/12/2010 &
\bo{Versione:} 0.3 &
\bo{Autore:} Baron Federico\\
\hline\\
\multicolumn{3}{p{470px}}{ Aggiunta nella sezione 2.3 la citazione di nomi di file relativi ad altri documenti.}\\
\\

\bo{Data:} 09/12/2010 &
\bo{Versione:} 0.2 &
\bo{Autore:} Baron Federico\\
\hline\\
\multicolumn{3}{p{470px}}{ Modifica alla sezione 4.3 e correzione accenti.}\\
\\

\bo{Data:} 07/12/2010 &
\bo{Versione:} 0.1 &
\bo{Autore:} Baron Federico\\
\hline\\
\multicolumn{3}{p{470px}}{ Prima stesura del documento Norme Di Progetto.}\\
\\

% TEMPLATE PER REVISIONE

\end{tabular}

% INDICE
\tableofcontents
\thispagestyle{fancy} % per lo stile di header e footer


\chapter*{Sommario}
\thispagestyle{fancy}
Questo documento contiene un elenco di norme, approvate da
tutti i membri del gruppo VT.G e applicate su tutto il progetto.
In particolare le norme stabilite riguardano le convenzioni di
comunicazione interne ed esterne, riunioni tra i componenti del team e
documentazioni del progetto. 
Inoltre sono definiti alcuni aspetti dell'amministrazione di progetto
particolarmente importanti per poter lavorare con strumenti e tecnologie
comuni.

\thispagestyle{fancy} % serve perche' nelle pagine di inizio Chapter esca header e footer
\pagenumbering{arabic} % INIZIO NUMERAZIONE NORMALE
\rfoot{\thepage\ di \pageref{TotPages}}
\addcontentsline{toc}{chapter}{Sommario}


\chapter{Introduzione}
\thispagestyle{fancy} % serve perche' nelle pagine di inizio Chapter esca header e footer

\section{Scopo del documento}
Il presente documento viene stilato al fine di descrivere le norme che il gruppo
segue per regolare le proprie attivit\`a di lavoro.


\section{Scopo del prodotto}

\section{Glossario}
Il Glossario \`e definito con un documento a parte (\emph{Glossario.pdf}). Tutti
i termini caratterizzati da \underline{questa sottolineatura} sono ivi
definiti.\\ Verr\`a sottolineata solamente la prima occorrenza di ciascun
termine presente nel Glossario, per non compromettere la leggibilit\`a del documento.

\section{Riferimenti}

\subsection{Normativi} % oppure rif. a Norme di progetto con leggi e tutto
\begin{itemize}
  \item ISO/IEC 12207:1995 - Cicli di vita software
  \item ISO/IEC 9126:2001 - Quality Model
\end{itemize}

\subsection{Informativi}
\begin{itemize}
  \item Capitolato d'appalto CO2-NETMUS del corso di Ingegneria del Software
  A.A. 2010/11 :\\
  \url{http://www.math.unipd.it/~tullio/..}
  \item Slide delle lezioni del corso :\\
  \url{http://www.math.unipd.it/~tullio/.d.}
\end{itemize}


\chapter{Comunicazioni}
\thispagestyle{fancy}
Le comunicazioni si dividono tra interne ed esterne al gruppo, e vengono gestite
nei modi riportati qui sotto.

\section{Interne}
Le comunicazioni interne tra i membri del team si svolgono attraverso gli
strumenti offerti da \underline{Google Groups} nel profilo Valter Texas Group e
le e-mail personali. Nello specifico si utilizzano un forum per le discussioni
(sezione ``Discussioni''), dei documenti condivisi \underline{Google Docs} per
alcune note (sezione ``Documenti'') e degli avvisi tramite e-mail. 
Queste comunicazioni sono protette e visibili soltanto ai membri del gruppo.


\section{Esterne}
Le comunicazioni esterne sono gestite con una e-mail di gruppo
valtertexasgroup@googlegroups.com,  fornita da Google Groups.
I messaggi ricevuti a questo indirizzo di posta elettronica sono inoltrati
automaticamente alle e-mail personali dei componenti del team e pubblicati nella
sezione ``Discussioni'' del forum interno.


\chapter{Norme per la stesura dei documenti}
\thispagestyle{fancy}

\section{Nomenclatura}
Tutti i nomi dei file devono cominciare con la lettera maiuscola ed
indicare in maniera univoca il loro contenuto. Il nome deve essere seguito da un
trattino (-) e due cifre separate da un punto che indicano la versione del
documento.

\section{Contenuto}
Tutti i documenti devono contenere: 
\begin{itemize}
\item {una pagina iniziale con il logo e il nome del gruppo;}
\item {il contatto del gruppo;}
\item {il nome del documento;}
\item {la versione del documento;}
\item {la data di creazione;}
\item {la data di ultima modifica;}
\item {lo stato del documento (formale o informale);}
\item {l'uso del documento (esterno o interno);}
\item {la redazione;}
\item {la verifica;}
\item {l'approvazione;}
\item {la distribuzione;}
\item {un registro delle modifiche;}
\item {un indice del documento;}
\item {un sommario contenente un breve riassunto del documento;}
\item {l'indice delle tabelle (se presenta almeno una tabella nel documento);}
\item {l'indice delle figure (se presente almeno una figura nel documento).}
\end{itemize} 

\section{Norme tipografiche}
Il font da utilizzato \`e Knuth's Computer Modern (di default in
\underline{Latex}), la dimensione del capitolo e del sottocapitolo sono di
default in Latex, la dimensione del testo \`e 12pt e la codifica per i caratteri
\`e ISO-8859-1. In ogni documento si deve utilizzare la \underline{sottolineatura} per marcare i termini presenti nel glossario (solo
alla prima ricorrenza) e l'\emph{italic} per evidenziare i nomi di file che fanno riferimento ad altri documenti.

\section{Norme di stile}
Ogni capitolo deve essere identificato da un numero, mentre ogni sottocapitolo
deve essere identificato da due numeri separati da un punto (il primo indica il
capitolo e il secondo il sottocapitolo). Tra un capitolo/sottocapitolo e
l'altro, si deve utilizzare un'interlinea come separatore.

\section{Grafici ed illustrazioni}
Qualsiasi immagine e grafico devono essere accompagnati da un numero progressivo
all'interno del documento, preceduto da dei numeri che ne identificano il
capitolo e il sottocapitolo in cui si trova (esempio: immagine 1.3.1 \`e
l'immagine 1 all'interno del sottocapitolo 3 del primo capitolo).

\section{Acronimi ed abbreviazioni}
Le diverse revisioni sono indicate con RR, RPP, RPD, RQ, RA. Si utilizza
l'acronimo VT.G (Valter Texas Group) per qualsiasi riferimento al nostro gruppo
di progetto.


\chapter{Strumenti e tecnologie da utilizzare}
\thispagestyle{fancy}
Tutti i membri di VT.G nello svolgere attivit\`a� qui elencate saranno tenuti ad
utilizzare gli stessi strumenti e configurazioni.

\section{IDE}
Eclipse for Java Developers (versione $\geq$ Helios 3.6).
 
\section{Repository}
\underline{Google Code} Project Hosting.
All'interno del progetto saranno utilizzati due \underline{repository}: ``default''
per il progetto e ``documents'' per i documenti.
Nel repository documents verranno aperti tanti \underline{branch} quanti saranno
i documenti da redarre che convergeranno in prossimit\`a delle revisioni.

\section{Sistema di versionamento}
\underline{Mercurial}  (versione $\geq$ 1.7.2).

\section{Plugin Mercurial per Eclipse}
MercurialEclipse (versione $\geq$ 1.7).

% \section{SDK per AppEngine}
% ???
% 
% \section{Plugin per AppEngine}
% ???

\section{Latex}
Sono stati individuati due possibili prodotti:
\begin{itemize}
  \item TeX Live (TexLive 2010)
  \item MiKTeX (versione $\geq$ 2.9)
\end{itemize}

\section{Plugin Latex per Ecplise}
TeXlipse (versione $\geq$ 1.4.0).

\section{UML}
ArgoUML (versione $\geq$ 0.30.2) configurato per \underline{UML} 2.x.

\section{Gantt}
GanttProject (versione $\geq$ 2.0.10).



\chapter{Repository e Versionamento}
\thispagestyle{fancy}
Come servizio di hosting per mantenere online i nostri repository
verr\`a utilizzato Google Code Project Hosting
(http://code.google.com/hosting/), che oltre ad offrire un buon servizio di
\underline{ticketing}, \`e dotato di una comoda e completa interfaccia web per
la visualizzazione e gestione dei dati caricati. Dentro a tale spazio di hosting
verranno creati due repository: uno per i documenti e uno per il codice
sorgente. Come sistema di \underline{versionamento} per tali repository verr\`a
utilizzato Mercurial 1.6 (http://mercurial.selenic.com/), strumento che dovr\`a essere
installato nel sistema da ognuno dei componenti del gruppo. Poich\'e durante
l'intero sviluppo verr\`a principalmente utilizzato l'ambiente Eclipse,
utilizzeremo il comodo plugin Mercurial Eclipse che ci consentir\`a di accedere
graficamente a tutte le funzionalit\`a di Mercurial. Per scaricare localmente un
repository con le ultime versioni dei file si far\`a un PULL dentro il progetto
selezionato. Poi una volta effettuate delle modifiche, aggiunte o rimozioni
viene fatto un relativo COMMIT per descrivere l'avanzamento di versione e poi
viene fatto il PUSH per caricare tale versione nel repository. Mercurial
gestisce in maniera efficiente i vari conflitti che potrebbero esserci in PUSH
simultanei nello stesso branch.
Convenzione per i COMMIT:
\begin{itemize}
\item ADD: \`e stato aggiunto qualcosa di nuovo ad un file che gi\`a esisteva.
\item NEW: \`e stato creato qualcosa di nuovo, quindi un nuovo file. 
\item FIX: \`e stato modificato un contenuto presente in un file che gi\`a
esisteva. 
\item DEL: \`e stato totalmente eliminato qualcosa.
\end{itemize}

\section{Registro delle modifiche}
Per agevolare la comprensione delle modifiche effettuate, all'interno di
ciascun documento sar\`a sempre presente un registro delle modifiche dettagliato (a
partire da pagina II prima dell'indice), che permetter\`a di evidenziare la data
d' introduzione della modifica, il numero di versione raggiunto e l'autore di
tale modifica.




\chapter{Ticketing e time-tracking}
\thispagestyle{fancy}
Il tracciamento delle ore di lavoro per ogni componente verr\`a eseguito tramite
un progetto di GanttProject appositamente condiviso ed accessibile da ognuno
(non concorrentemente). Questo permetter\`a anche un comodo confronto con la
gestione delle risorse pianificata nel PQ.

Il tiketing sar\`a utilizzato tramite gli appositi strumenti forniti da Google Code.



\chapter{Riunioni ed incontri}
\thispagestyle{fancy}
Tutte le riunioni ufficiali vengono svolte nelle seguenti sedi e date :
\begin{itemize}
\item SEDI:
\begin{itemize}
\item Residenza Messori(via Manin, 31 - Padova) - abbreviazione RS,
\item Laboratorio di Informatica(via Paolotti - Padova) - abbreviazione LI.
\end{itemize}
\end{itemize}

\chapter{Consegna}
\thispagestyle{fancy}
Per ciascuna revisione, il gruppo VT.G si impegna a consegnare i documenti
tramite l'eMail del gruppo in un unico file compresso .zip: all'interno i file
saranno suddivisi in base all'uso (interno o esterno).

	

\end{document}