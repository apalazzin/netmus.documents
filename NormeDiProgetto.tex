\newcommand{\nomedoc}{Norme Di Progetto}
\newcommand{\versione}{1.2}
\newcommand{\versioneglossario}{1.0}
\newcommand{\versionenormeprogetto}{1.0}
\newcommand{\nomefile}{NormeDiProgetto-\versione.pdf}
\newcommand{\datacreazione}{7 Dicembre 2010}
\newcommand{\datamodifica}{19 Gennaio 2011}
\newcommand{\stato}{formale}
\newcommand{\uso}{interno}
\newcommand{\redazione}{Baron Federico\\&Palazzin Alberto}
\newcommand{\verifica}{Daminato Simone}
\newcommand{\approvazione}{Lovato Daniele}
\newcommand{\distribuzione}{
VT.G \\
& Prof. Vardanega Tullio\\
& Prof. Cardin Riccardo }

% FUNZIONI TIPOGRAFICHE
\newcommand{\co}{\texttt} % courier
\newcommand{\bo}{\textbf} % bold
\newcommand{\pr}{\par\medskip} % paragrafo spaziato
\newcommand{\sca}{\textsc} % small caps

\documentclass[a4paper,12pt]{report}
% 10pt,11pt,12pt
% titlepage, notitlepage -> per dare inizio o no ad una nuova pagina dopo titolo
% twoside -> per dire se fronte-retro
\usepackage[latin1]{inputenc}
% per caratteri accentati
\usepackage[italian]{babel}
% per regole sintattiche italiane
\usepackage[bookmarks=true, pdfborder={0 0 0 0}]{hyperref}
% per collegamenti ipertestuali
\usepackage{graphicx}
% per inserimento immagini

% \usepackage{enumerate}
% per personalizzare elenchi puntati

\usepackage[hmargin=2cm]{geometry} %margine 2 cm
%\geometry{options varie}

% comandi per gestire meglio header e footer
\usepackage{fancyhdr}  % header e footer
\usepackage{totpages}
\pagestyle{fancy}
\renewcommand{\headrulewidth}{0.4pt}
\renewcommand{\footrulewidth}{0.4pt}

\setlength{\headheight}{1.2cm} % NON TOCCARE
\setlength{\voffset}{-1.5cm} % NON TOCCARE
\setlength{\textheight}{700pt} % NON TOCCARE
\setlength{\parindent}{0pt} % INDENTAZIONE

\lhead{\nomedoc\  (ver. \versione)}
\chead{}
\rhead{\includegraphics[height=1cm]{img/netmus.png}}
\lfoot{\includegraphics[height=0.8cm]{img/logo.png}}
\cfoot{}
\rfoot{\thepage}

% \usepackage{listings}   per codice sorgente

\author{VT.G - Valter Texas Group}

\begin{document}

\pagenumbering{Roman} % INIZIO NUMERAZIONE ARABA

\vspace*{1cm}
\begin{center}

\includegraphics[width=9cm]{img/logo.png}\\
\vspace{0.5cm}
\begin{LARGE} \sca{VT.G - Valter Texas Group} \end{LARGE}\\
\vspace{0.5cm}
\begin{Large}
\emph{valtertexasgroup@googlegroups.com} \end{Large}\\
\vspace*{1cm} \includegraphics[width=5cm]{img/netmus.png}\\
\vspace{0.5cm}
\begin{Large} \sca{\nomedoc} \end{Large}\\
\vspace{1cm}
\begin{Large} \emph{Ingegneria del Software A.A. 2010-2011} \end{Large}\\
\end{center}
\vspace{1cm}

% INFORMAZIONI DOCUMENTO
\begin{center}
\begin{tabular}{r|l}
\hline & \\
\bo{Nome} & \nomefile \\
\bo{Versione attuale} & \versione \\
\bo{Data creazione} & \datacreazione \\
\bo{Data ultima modifica} & \datamodifica \\
\bo{Stato} & \stato \\
\bo{Uso} & \uso \\
\bo{Redazione} & \redazione \\
\bo{Verifica} & \verifica \\
\bo{Approvazione} & \approvazione \\
\bo{Distribuzione} & \distribuzione \\
& \\\hline
\end{tabular}
\end{center}
\newpage

% REGISTRO MODIFICHE
\section*{Registro delle modifiche}

\begin{longtable}{|p{0.13\textwidth}|c|p{0.2\textwidth}|p{0.46\textwidth}|}
\hline
\rowcolor{orange} \bo{Data} & \bo{Versione} & \bo{Autore} & \bo{Descrizione} \\
\hline
\endhead
\hline
\endfoot

19/01/2011 & 1.2 & Palazzin Alberto & Aggiunto e redatto capitolo
``Analisi e Progettazione'' e ``Norme di codifica''.\\
\hline
12/01/2011 & 1.1 & Mandolo Andrea & Modificato layout Registro delle
modifiche.\\
\hline
19/12/2010 & 1.0 & Lovato Daniele & Validazione per consegna RR.\\
\hline
19/12/2010 & 0.6 & Daminato Simone & Verificato l'intero documento.\\
\hline
15/12/2010 & 0.5 & Baron Federico & Correzione ortografica e sintattica
generale.\\
\hline
11/12/2010 & 0.4 & Palazzin Alberto & Sistemazione dei Capitoli 5, 6, 7 e 8.\\
\hline
10/12/2010 & 0.3 & Palazzin Alberto & Sistemazione dei Capitoli 2, 3, 4, 9 e
10.\\
\hline
09/12/2010 & 0.2 & Baron Federico & Aggiunto il capitolo 7.\\
\hline
 07/12/2010 & 0.1 & Baron Federico & Prima stesura del documento.
\end{longtable}


% INDICE
\tableofcontents


\chapter*{Sommario}
\thispagestyle{fancy}
Questo documento contiene un elenco di norme, approvate da
tutti i membri del gruppo VT.G e applicate su l'intero progetto; riguardano in
particolare le convenzioni di comunicazione interne ed esterne, riunioni tra i componenti del team e
documentazioni del progetto. 
Inoltre sono definiti alcuni aspetti dell'amministrazione di progetto
particolarmente importanti per lavorare con strumenti e tecnologie
comuni.

\thispagestyle{fancy} % serve perche' nelle pagine di inizio Chapter esca header e footer
\pagenumbering{arabic} % INIZIO NUMERAZIONE NORMALE
\rfoot{\thepage\ di \pageref{TotPages}}
\addcontentsline{toc}{chapter}{Sommario}


\chapter{Introduzione}
\thispagestyle{fancy} % serve perche' nelle pagine di inizio Chapter esca header e footer

\section{Scopo del documento}
Il presente documento viene stilato al fine di descrivere le norme che il gruppo
deve seguire per regolare le proprie attivit\`a di lavoro.


\section{Scopo del prodotto}

\section{Glossario}
Il Glossario \`e definito con un documento a parte (\emph{Glossario.pdf}). Tutti
i termini caratterizzati da \underline{questa sottolineatura} sono ivi
definiti.\\ Verr\`a sottolineata solamente la prima occorrenza di ciascun
termine presente nel Glossario, per non compromettere la leggibilit\`a del documento.

\section{Riferimenti}

\subsection{Normativi} % oppure rif. a Norme di progetto con leggi e tutto
\begin{itemize}
  \item ISO/IEC 12207:1995 - Cicli di vita software
  \item ISO/IEC 9126:2001 - Quality Model
\end{itemize}

\subsection{Informativi}
\begin{itemize}
  \item Capitolato d'appalto CO2-NETMUS del corso di Ingegneria del Software
  A.A. 2010/11 :\\
  \url{http://www.math.unipd.it/~tullio/..}
  \item Slide delle lezioni del corso :\\
  \url{http://www.math.unipd.it/~tullio/.d.}
\end{itemize}


\chapter{Comunicazioni}
\thispagestyle{fancy}
Le comunicazioni si dividono in interne ed esterne al gruppo, e vengono gestite
nei modi di seguito riportati.

\section{Interne}
Le comunicazioni interne tra i membri del team si svolgono attraverso gli
strumenti offerti da \underline{Google Groups} nel profilo Valter Texas Group e
le e-mail personali. Nello specifico si utilizzano: un forum per le discussioni
(sezione ``Discussioni''), dei documenti condivisi \underline{Google Docs} per
alcune note (sezione ``Documenti'') e degli avvisi tramite e-mail. 
Queste comunicazioni sono protette e visibili soltanto ai membri del gruppo.


\section{Esterne}
Le comunicazioni esterne sono gestite con un'e-mail di gruppo
valtertexasgroup@googlegroups.com, fornita da Google Groups. I messaggi ricevuti
a questo indirizzo di posta elettronica sono inoltrati automaticamente alle e-mail personali dei componenti del team e pubblicati nella
sezione ``Discussioni'' del forum interno.


\chapter{Norme per la stesura dei documenti}
\thispagestyle{fancy}

\section{Nomenclatura}
Tutti i nomi dei file devono cominciare con la lettera maiuscola e
indicare in maniera univoca il loro contenuto. Il nome deve essere seguito da un
trattino (-) e due cifre separate da un punto che indicano la versione del
documento (esempio: NormeDiProgetto-1.0.pdf).

\section{Contenuto}
Tutti i documenti devono contenere: 
\begin{itemize}
\item {una pagina iniziale con il logo e il nome del gruppo;}
\item {il contatto del gruppo;}
\item {il nome del documento;}
\item {la versione del documento;}
\item {la data di creazione;}
\item {la data di ultima modifica;}
\item {lo stato del documento (formale o informale);}
\item {l'uso del documento (esterno o interno);}
\item {la redazione;}
\item {la verifica;}
\item {l'approvazione;}
\item {la distribuzione;}
\item {un registro delle modifiche;}
\item {un indice del documento;}
\item {un sommario contenente un breve riassunto del documento;}
\item {l'indice delle tabelle (se presenta almeno una tabella nel documento);}
\item {l'indice delle figure (se presente almeno una figura nel documento).}
\end{itemize} 

\section{Norme tipografiche}
Il font da utilizzare \`e Knuth's Computer Modern (di default in
\underline{Latex}), la dimensione del capitolo e del sottocapitolo \`e quella di
default in Latex, la dimensione del testo \`e 12pt e la codifica per i caratteri
\`e ISO-8859-1. In ogni documento si deve utilizzare la \underline{sottolineatura} per marcare i termini presenti nel glossario (solo
alla prima occorrenza) e l'\emph{italic} per evidenziare i nomi di file che
fanno riferimento ad altri documenti.

\section{Norme di stile}
Ogni capitolo deve essere identificato da un numero mentre ogni sottocapitolo
deve essere identificato da due numeri separati da un punto (il primo indica il
capitolo e il secondo il sottocapitolo). Tra un capitolo/sottocapitolo e
l'altro, si deve utilizzare un'interlinea come separatore.

\section{Figure e tabelle}
Qualsiasi figura di contenuto (grafici, schemi) e tabella deve essere
accompagnata da un numero progressivo all'interno del documento, preceduto da
numeri che ne identificano il capitolo e il numero progressivo relativo al capitolo (esempio: immagine 1.3
\`e l'immagine 3 all'interno del capitolo 1) e da una
didascalia che ne dia una breve spiegazione.

\section{Acronimi e abbreviazioni}
Le diverse revisioni sono indicate con:
\begin{itemize}
\item {RR: Revisione dei Requisiti;}
\item {RP: Revisione di Progettazione; }
\item {RQ: Revisione di Qualifica;}
\item {RA: Revisione di Accettazione.}
\end{itemize}
Si utilizza l'acronimo VT.G (Valter Texas Group) per qualsiasi riferimento al
nostro gruppo di progetto.



\chapter{Strumenti e tecnologie da utilizzare}
\thispagestyle{fancy}
Si impone a tutti i membri di VT.G di utilizzare gli stessi strumenti e
configurazioni per qualsiasi attivit\`a di lavoro. Qui di seguito vengono
riportati i dettagli relativi ad ogni strumento.

\section{IDE}
Eclipse for \underline{Java} Developers (versione $\geq$ Helios 3.6).
\\
Scarica e installa dal link\\
\url{http://www.eclipse.org/downloads/packages/eclipse-ide-java-developers/heliossr1}.
 
\section{Repository}
\underline{Google Code} Project Hosting.
\\
All'interno del progetto (\url{http://code.google.com/p/netmus/}) vengono
utilizzati due \underline{repository}: ``default'' per il progetto
(\url{https://netmus.googlecode.com/hg/}) e ``documents'' per i documenti
(\url{https://documents.netmus.googlecode.com/hg/}).\\
Per facilitare il lavoro simultaneo tra i membri, nel repository documents \`e
stato aperto un \underline{branch} per ogni documento da redarre. In prossimit\`a di una
revisione vengono fatti convergere in un nuovo branch relativo ad essa.

\section{Sistema di versionamento}
\underline{Mercurial}  (versione $\geq$ 1.6). 
\\
Scarica e installa dal link
\url{http://mercurial.selenic.com/}.

\section{Plugin Mercurial per Eclipse}
MercurialEclipse (versione $\geq$ 1.7). 
\\
Per installare seguire queste
operazioni: 
\begin{itemize}
\item {aprire il programma Eclipse;} 
\item {nel men\`u in alto selezionare ``Help'';}
\item {selezionare ``Install new Software'';}
\item {in ``Work with'' inserire \url{http://cbes.javaforge.com/update};}
\item {selezionare ``Mercurial Eclipse'' dalla lista comparsa e deselezionare
``Windows Binaries for Mercurial'' se non si usa un ambiente Windows (se
richiede una login e una password, andare al link
\url{http://www.javaforge.com/createUser.spr}, registrarsi al sito e poi
utilizzare queste credenziali per continuare l'istallazione del plugin);}
\item {installare.}
\end{itemize}

\section{Plugin e SDKs di GAE e GWT per Eclipse}
Google App Engine (versione $\geq$ 1.4.0) e Google Web Toolkit (versione $\geq$
2.1.1).
\\
Per installare seguire queste
operazioni: 
\begin{itemize}
\item {aprire il programma Eclipse;} 
\item {nel men\`u in alto selezionare ``Help'';}
\item {selezionare ``Install new Software'';}
\item {in ``Work with'' inserire \url{http://dl.google.com/eclipse/plugin/3.6};}
\item {selezionare Plugin (Google Plugin 1.4.2 for Eclipse 3.6) e SDKs (Google
App Engine Java SDK 1.4.0 e Google Web Toolkit SDK 2.1.1) dalla lista comparsa;}
\item {installare.}
\end{itemize} 

\section{JavaFX}
JavaFX SDK (versione 1.2.1).
\\
Scarica e installa la versione 1.2.1 dal link \\
\url{http://www.oracle.com/technetwork/java/javafx/downloads/previous-jsp-137062.html}.

\section{Plugin JavaFX per Eclipse}
JavaFX Feature (versione 1.2.1).
\\
Per installare seguire queste
operazioni: 
\begin{itemize}
\item {aprire il programma Eclipse;} 
\item {nel men\`u in alto selezionare ``Help'';}
\item {selezionare ``Install new Software'';}
\item {in ``Work with'' inserire \url{http://javafx.com/downloads/eclipse-plugin};}
\item {selezionare l'unico file dalla lista comparsa;}
\item {installare.}
\end{itemize} 

\section{Latex}
\begin{itemize}
\item {Per Linux: TeX Live (versione 2010). 
\\
Scarica e installa dal link
\url{http://www.tug.org/texlive/}.}
\item {Per Mac OS X: MacTex (versione 2010). 
\\
Scarica e installa dal link
\url{http://www.tug.org/mactex/}.}
\item {Per Windows: MiKTeX (versione 2.9). 
\\
Scarica e installa dal link
\url{http://miktex.org/}.}
\end{itemize}

\section{Plugin Latex per Ecplise}
TeXlipse (versione $\geq$ 1.4.0). 
\\
Per installare seguire queste
operazioni: 
\begin{itemize}
\item {aprire il programma Eclipse;} 
\item {nel men\`u in alto selezionare ``Help'';}
\item {selezionare ``Install new Software'';}
\item {in ``Work with'' inserire \url{http://texlipse.sourceforge.net};}
\item {selezionare l'unico file dalla lista comparsa;}
\item {installare.}
\end{itemize} 

\section{Plugin UML per Eclipse}
Papyrus (versione $\geq$ 0.7.2). 
\\
Per installare seguire queste
operazioni: 
\begin{itemize}
\item {aprire il programma Eclipse;} 
\item {nel men\`u in alto selezionare ``Help'';}
\item {selezionare ``Install new Software'';}
\item {in ``Work with'' inserire} \\
{\url{http://download.eclipse.org/modeling/mdt/papyrus/updates/releases/helios};}
\item {selezionare ``Papyrus UML MDT'' dalla lista comparsa;}
\item {installare.}
\end{itemize} 

\section{Gantt}
GanttProject (versione $\geq$ 2.0.10). 
\\
Scarica e installa dal link
\url{http://www.ganttproject.biz/download}.


\chapter{Pianificazione, ticketing e \\time-tracking}
\thispagestyle{fancy}
Per fare pianificazione verranno utilizzati i diagrammi di Gantt e nello
specifico lo strumento GanttProject (per maggiori dettagli vedi sezione 4.11). 
\\Il tracciamento delle ore di lavoro per ogni componente verr\`a eseguito
tramite un progetto di GanttProject appositamente condiviso e accessibile da ognuno (non concorrentemente). Questo permetter\`a anche un comodo confronto con la
gestione delle risorse pianificata nel \emph{PianoDiQualifica-1.0.pdf}.
Il ticketing sar\`a utilizzato tramite gli appositi strumenti forniti da Google
Code.


\chapter{Analisi e Progettazione}
\thispagestyle{fancy} 

Per l'analisi dei requisiti e per la progettazione si utilizzer\`a il linguaggio
UML (versione 2.0) perch\`e adottato come standard nella modellazione e
scrittura dei diagrammi: Use case, di flusso, dei package, delle classi, degli
oggetti, di attivit\`a e di sequenza. 
\\
La tecnologia e gli strumenti utilizzati per la scrittura in UML
sono specificati chiaramente nella sezione 4.10.

\section{Individuazione e tracciamento dei requisiti}
L'individuazione e il tracciamento dei requisiti si effettueranno analizzando a
fondo i requisiti esposti nel capitolato d'appalto e quelli emersi nell'incontro effettuato con il proponente (vedi documento \emph{Verbale-1.0.pdf}).
\\
Per identificare in modo univoco e significativo ognuno dei requisiti software
estratti dall'analisi, gli viene assegnato un codice.
La lista dei requisiti sar\`a presente solamente nel documento
\emph{AnalisiDeiRequisiti-1.0.pdf} mentre in tutti gli altri verranno riferiti tramite il codice suddetto.
\\
\\
(C1/C2)(F/Q/I/V)(N/D/O)-n1.n2.n3...nx\\
C1 = componente 1 (componente di persistenza e visualizzazione)\\
C2 = componente 2 (componente di recupero delle informazioni dei brani
musicali contenuti nei sistemi di riproduzione personale dell'utente)\\
F = funzionale\\
Q = qualit\`a\\
I = interfacciamento\\
V = vincolo\\
N = obbligatorio (necessario)\\
D = desiderabile\\
O = opzionale\\
nx = livello di innestamento nella gerarchia dei requisiti. I ``figli'' di un
requisito avranno una cifra in pi\`u del genitore.


\chapter{Norme di Codifica}
\thispagestyle{fancy} 
Per la fase di codifica del progetto \`e stato scelto innanzitutto l'ambiente
integrato di sviluppo Eclipse perch\`e supporta sia la programmazione Java
sia la possibilit\`a  di installare eventuali plugin per interfacciarsi con
altre tecnologie (per maggiori dettagli vedi sezione 4.1).\\
Per la componente di visualizzazione del Catalogo
Multimediale sono stati scelti i plugin di GAE e di GWT oltre i relativi SDKs
(per maggiori dettagli vedi sezione 4.5).\\ 
Per la codifica della componente di estrazione delle informazioni \`e stata
scelta la tecnologia JavaFX. Questa tecnologia \`e stata sia suggerita dal
proponente nel capitolato d'appalto sia risultata una delle tecnologie migliori
per lo sviluppo delle RIA (per maggiori dettagli vedi sezione 4.6 e 4.7).
\\ \\
Per cercare di semplificare la verificabilit\`a, migliorare la
manutenibilit\`a e massimizzare la portabilit\`a si deve rendere il codice
scritto pi\`u leggibile possibile.\\
Per fare questo nel codice si dovr\`a utilizzare un tool automatico gi\`a
installato dentro Eclipse che genera documentazione API in HTML a partire da codice sorgente Java
e pi\`u precisamente usando il comando ``Generate Javadoc''; mentre nel progetto
si dovr\`a strutturare nella maniera pi\`u semplice e coerente i moduli, i file,
le directory, ecc. .\\
Inoltre si dovr\`a cercare di scrivere meno linee di codice possibile e non
usare l'identazione con il TAB (poco portabile) ma utilizzare gli spazi.
\\ \\
Ogni classe Java dovr\`a avere la sua intestazione contenente:
\begin{itemize}
  \item {Dati dell'unit\`a: tipo, contenuto, posizione;}
  \item {Responsabilit\`a: autore, reparto, organizzazione;}
  \item {Copyright / copyleft: licenze, visibilit\`a;}
  \item {Avvertenze: limiti di uso e di garanzia;}
  \item {Registro modifiche: storia, spiegazione, versione.} 
\end{itemize}
Ogni programmatore dovr\`a:
\begin{itemize}
  \item{consegnare codice che compila senza errori fatali o
potenziali (warning);} 
  \item{aver utilizzato con chiarezza e coerenza i costrutti del
linguaggio;}
  \item{aver utilizzato un sottoinsieme appropriato del linguaggio
(preferire i costrutti di maggiore robustezza, verificabilit\`a e leggibilit\`a
rispetto quelli di maggiore potenza espressa e velocit\`a).}
\end{itemize}

\chapter{Repository e Versionamento}
\thispagestyle{fancy}
Come servizio di hosting per mantenere online i repository
viene utilizzato Google Code Project Hosting
(\url{http://code.google.com/hosting/}). Questo offre un buon
servizio di \underline{ticketing} e una comoda e completa interfaccia web per
la visualizzazione e gestione dei dati caricati. In tale spazio sono
stati creati due repository: uno per i documenti e uno per il codice sorgente
(per maggiori dettagli vedi sezione 4.2). 
\\
Come sistema di
\underline{versionamento} per tali repository viene utilizzato Mercurial,
strumento che deve essere installato nel sistema di ogni componente del gruppo
(per maggiori dettagli vedi sezione 4.3).
\\
Poich\'e durante l'intero sviluppo verr\`a usato principalmente l'ambiente
Eclipse, si utilizzer\`a il comodo plugin Mercurial Eclipse che consente di accedere
graficamente a tutte le funzionalit\`a di Mercurial (per maggiori dettagli vedi
sezione 4.4). 
\\
Per scaricare localmente un repository con le ultime versioni dei
file si deve eseguire un ``\underline{Pull}'' all'interno del progetto
selezionato. Una volta effettuate delle modifiche, aggiunte o rimozioni viene
utilizzato il relativo ``\underline{Commit}'' per descrivere l'avanzamento di
versione e poi viene ultimato il ``\underline{Push}'' per caricare tale versione
nel repository. Mercurial con i ``\underline{Merge}'' riesce a gestire in maniera efficiente i vari conflitti
che potrebbero esserci in ``Push'' simultanei nello stesso branch.\\ 
\\
In ogni ``Commit'' viene adottata tale convenzione: 
``Operazione NomeDelFile Descrizione''
\begin{itemize}
\item {Operazione: \`e l'operazione fatta su un determinato file (ADD, NEW, FIX,
DEL) e nello specifico}
\begin {itemize}
\item {ADD: \`e stato aggiunto qualcosa di nuovo ad un file gi\`a esistente;}
\item {NEW: \`e stato creato qualcosa di nuovo, quindi un nuovo file;} 
\item {FIX: \`e stato modificato un contenuto presente in un file gi\`a
esistente;} 
\item {DEL: \`e stato totalmente eliminato qualcosa;}
\end {itemize}
\item {NomeDelFile: \`e il nome del file in questione;}
\item {Descrizione: \`e una breve descrizione dell'operazione eseguita.}
\end{itemize}
Riguardo al versionamento di un documento, ogni modifica apportata a questo
determina un incremento dell'indice minore (esempio: da 0.1 a 0.2) mentre ogni
rilascio pubblico determina un incremento dell'indice maggiore (esempio in
previsione della RR: da 0.4 a 1.0).


\chapter{Registro delle modifiche}
\thispagestyle{fancy}
Per agevolare la comprensione e tracciare uno storico delle modifiche
effettuate, all'interno di ciascun documento deve essere sempre
presente un registro dettagliato delle modifiche (a partire da
pagina II prima dell'indice). Questo permette di evidenziare la data
d'introduzione della modifica, il numero di versione raggiunto e l'autore di tale modifica.


\chapter{Riunioni e incontri}
\thispagestyle{fancy}
Tutte le riunioni e gli incontri ufficiali vengono svolti nelle seguenti sedi:

\begin{itemize}
\item Residenza Messori, in via Manin 31 a Padova (sede principale);
\item Laboratorio di Informatica, in via Paolotti a Padova.
\end{itemize}


\chapter{Consegna}
\thispagestyle{fancy}
Per ciascuna revisione, il gruppo VT.G si impegna a consegnare i documenti
tramite l'e-mail del gruppo in un unico file compresso ``.zip''. Al suo interno i
file saranno suddivisi in base all'uso, interno o esterno.


\end{document}