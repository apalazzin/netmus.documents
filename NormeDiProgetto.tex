\newcommand{\nomedoc}{Norme di progetto}
\newcommand{\versione}{0.1}
\newcommand{\nomefile}{NormeDiProgetto\versione.pdf}
\newcommand{\datacreazione}{7 Dicembre 2010}
\newcommand{\datamodifica}{8 Dicembre 2010}
\newcommand{\stato}{formale}
\newcommand{\uso}{interno}
\newcommand{\redazione}{Baron Federico}
\newcommand{\verifica}{Caputo Cosimo}
\newcommand{\approvazione}{}
\newcommand{\distribuzione}{
VT.G \\
& Prof. Vardanega Tullio }

% FUNZIONI TIPOGRAFICHE
\newcommand{\co}{\texttt} % courier
\newcommand{\bo}{\textbf} % bold
\newcommand{\pr}{\par\medskip} % paragrafo spaziato
\newcommand{\sca}{\textsc} % small caps

\documentclass[a4paper,12pt]{report}
% 10pt,11pt,12pt
% titlepage, notitlepage -> per dare inizio o no ad una nuova pagina dopo titolo
% twoside -> per dire se fronte-retro
\usepackage[latin1]{inputenc}
% per caratteri accentati
\usepackage[italian]{babel}
% per regole sintattiche italiane
\usepackage[bookmarks=true, pdfborder={0 0 0 0}]{hyperref}
% per collegamenti ipertestuali
\usepackage{graphicx}
% per inserimento immagini

% \usepackage{enumerate}
% per personalizzare elenchi puntati

\usepackage[hmargin=2cm]{geometry} %margine 2 cm
%\geometry{options varie}

% comandi per gestire meglio header e footer
\usepackage{fancyhdr}  % header e footer
\usepackage{totpages}
\pagestyle{fancy}
\renewcommand{\headrulewidth}{0.4pt}
\renewcommand{\footrulewidth}{0.4pt}

\setlength{\headheight}{1.2cm} % NON TOCCARE
\setlength{\voffset}{-1.5cm} % NON TOCCARE
\setlength{\textheight}{700pt} % NON TOCCARE
\setlength{\parindent}{0pt} % INDENTAZIONE

\lhead{\nomedoc\  (ver. \versione)}
\chead{}
\rhead{\includegraphics[height=1cm]{img/netmus.png}}
\lfoot{\includegraphics[height=0.8cm]{img/logo.png}}
\cfoot{}
\rfoot{\thepage}

% \usepackage{listings}   per codice sorgente

\author{VT.G - Valter Texas Group}

\begin{document}

\pagenumbering{Roman} % INIZIO NUMERAZIONE ARABA

\vspace*{1cm}
\begin{center}

\includegraphics[width=9cm]{img/logo.png}\\
\vspace{0.5cm}
\begin{LARGE} \sca{VT.G - Valter Texas Group} \end{LARGE}\\
\vspace{0.5cm}
\begin{Large}
\emph{valtertexasgroup@googlegroups.com} \end{Large}\\
\vspace*{1cm} \includegraphics[width=5cm]{img/netmus.png}\\
\vspace{0.5cm}
\begin{Large} \sca{\nomedoc} \end{Large}\\
\vspace{1cm}
\begin{Large} \emph{Ingegneria del Software A.A. 2010-2011} \end{Large}\\
\end{center}
\vspace{1cm}

% INFORMAZIONI DOCUMENTO
\begin{center}
\begin{tabular}{r|l}
\hline & \\
\bo{Nome} & \nomefile \\
\bo{Versione attuale} & \versione \\
\bo{Data creazione} & \datacreazione \\
\bo{Data ultima modifica} & \datamodifica \\
\bo{Stato} & \stato \\
\bo{Uso} & \uso \\
\bo{Redazione} & \redazione \\
\bo{Verifica} & \verifica \\
\bo{Approvazione} & \approvazione \\
\bo{Distribuzione} & \distribuzione \\
& \\\hline
\end{tabular}
\end{center}
\newpage

% REGISTRO MODIFICHE
\section*{Registro delle modifiche}
\begin{tabular}{lll}
% REVISIONI DEL DOCUMENTO
% DALLA PIU' NUOVA ALLA PIU' VECCHIA

% TEMPLATE PER REVISIONE

\end{tabular}

% INDICE
\tableofcontents
\thispagestyle{fancy} % per lo stile di header e footer


\chapter*{Sommario}
Questo documento contiene un elenco di norme le quali sono state approvate da
tutti i membri del gruppo VT.G e di conseguenza saranno applicate in tutti i
loro punto durante l'intero progetto in questione.
Le norme stabilite riguardano in particolare le convenzioni di comunicazione tra
i membri del gruppo e tra questi e l'esterno, dalle riunioni alla
documentazione.
Vengono definiti inoltre alcuni aspetti dell'amministrazione di progetto
particolarmente importanti che necessitano fin da subito della conoscenza e
dell'approvazione di tutti i membri di VT.G, come le tecnologie utilizzate
riguardo ad attività fondamentali per il progetto.

\thispagestyle{fancy} % serve perche' nelle pagine di inizio Chapter esca header e footer
\pagenumbering{arabic} % INIZIO NUMERAZIONE NORMALE
\rfoot{\thepage\ di \pageref{TotPages}}
\addcontentsline{toc}{chapter}{Sommario}


\chapter{Introduzione}
\thispagestyle{fancy} % serve perche' nelle pagine di inizio Chapter esca header e footer

\section{Scopo del documento}
Il presente documento viene redatto al fine di descrivere le norme che il gruppo
si propone per regolare le attivit\`a, di vario genere, che da esso
verranno svolte.
Tutti i documenti redatti da VT.G. seguiranno le norme qui riportate.

\section{Scopo del prodotto}

\section{Glossario}
Il Glossario \`e definito con un documento a parte (\emph{Glossario.pdf}). Tutti
i termini caratterizzati da \underline{questa sottolineatura} sono ivi
definiti.\\ Verr\`a sottolineata solamente la prima occorrenza di ciascun
termine presente nel Glossario, per non compromettere la leggibilit\`a del documento.

\section{Riferimenti}

\subsection{Normativi} % oppure rif. a Norme di progetto con leggi e tutto
\begin{itemize}
  \item ISO/IEC 12207:1995 - Cicli di vita software
  \item ISO/IEC 9126:2001 - Quality Model
\end{itemize}

\subsection{Informativi}
\begin{itemize}
  \item Capitolato d'appalto CO2-NETMUS del corso di Ingegneria del Software
  A.A. 2010/11 :\\
  \url{http://www.math.unipd.it/~tullio/..}
  \item Slide delle lezioni del corso :\\
  \url{http://www.math.unipd.it/~tullio/.d.}
\end{itemize}


\chapter{Comunicazioni}
\thispagestyle{fancy}
Le comunicazioni tra i componenti del gruppo, e tra questi ed i committenti e
proponenti del capitolato, verranno gestite nei modi seguenti.

\section{Interne}
Le comunicazioni interne al gruppo si svolgeranno attraverso gli strumenti
offerti da \underline{Google Groups} (documenti condivisi \underline{Google
Docs}, discussioni su forum, discussioni tramite eMail personali) e saranno
visibili soltanto ai membri del gruppo. Tutti i messaggi ed i documenti
condivisi saranno salvati in modo da essere facilmente reperibili all'interno
delle sezioni ``Discussioni'' e ``Pagine'' del profilo Walter Texas Group in
Google Groups.

\section{Esterne}
Le comunicazioni esterne saranno gestite con una mail di gruppo
valtertexasgroup@googlegroups.com,  fornita da Google Groups i cui messaggi
ricevuti saranno inoltrati alle email personali dei componenti del gruppo.
Queste comunicazioni saranno sempre documentate tramite la redazione di un
verbale da sottoporre ai committenti/proponenti in modo da formalizzare gli
accordi presi.



\chapter{Norme per la stesura dei documenti}
\thispagestyle{fancy}

\section{Nomenclatura}
Tutti i nomi dei file cominceranno con la lettera maiuscola, ed indicheranno in
maniera univoca il loro contenuto. Il nome sar\`a seguito da due cifre separate da
un punto, che indicheranno la versione del documento (guardare il capitolo in proposito).

\section{Contenuto}
Tutti i documenti conterranno, nell'ordine: una pagina iniziale contenente logo,
nome del documento, versione e contatti; un diario delle modifiche; un indice
del documento; un sommario contenente un breve riassunto del documento. Al
termine di ciascun documento saranno presenti inoltre, se necessari, un indice
delle tabelle ed uno delle illustrazioni.

\section{Norme tipografiche}
Il font utilizzato sar\`a Knuth's Computer Modern (default \underline{Latex}) e
dimensione del testo 12pt. La codifica utilizzata per i caratteri sar\`a UTF-8.

\section{Norme di stile}
Ogni capitolo e sottocapitolo sar\`a identificato da due numeri separati da un
punto: il primo indicher\`a il capitolo, il secondo il sottocapitolo. Tra un
capitolo, o un sottocapitolo, e l'altro, verr\`a utilizzato un interlinea come
separatore.

\section{Grafici ed illustrazioni}
Ciascuna immagine e grafico saranno accompagnati da un numero progressivo
all'interno del documento, preceduto da dei numeri che ne identificheranno il
capitolo e il sottocapitolo in cui si trova (per esempio, immagine 1.3.1:
immagine 1 all'interno del sottocapitolo 3 del primo capitolo).

\section{Acronimi ed abbreviazioni}
Le diverse revisioni verranno indicate con RR, RPP, RPD, RQ, RA. Verr\`a
utilizzato l'acronimo VT.G (Valter Texas Group) per riferimenti al nostro gruppo
di progetto.

\section{Descrizione delle modifiche}
Le seguenti parole chiave verranno utilizzate per definire ci\`o di cui si occupa
la modifica apportata.
\begin{itemize}
\item add: \`e stato aggiunto qualcosa di nuovo ad un file che gi\`a esisteva.
\item new: \`e stato creato qualcosa di nuovo, quindi un nuovo file. 
\item fix: \`e stato modificato un contenuto presente in un file che gi\`a
esisteva. 
\item del: \`e stato totalmente eliminato qualcosa.
\end{itemize}


\chapter{Strumenti e tecnologie da utilizzare}
\thispagestyle{fancy}
Tutti i membri di VT.G nello svolgere attività qui elencate saranno tenuti ad
utilizzare gli stessi strumenti e configurazioni.

\section{IDE}
Eclipse for Java Developers (versione $\geq$ Helios 3.6).
 
\section{Repository}
\underline{Google Code} Project Hosting.
All'interno del progetto saranno utilizzati due \underline{repository}: ``default''
per il progetto e ``documents'' per i documenti.

\section{Sistema di versionamento}
\underline{Mercurial}  (versione $\geq$ 1.7.2).

\section{Plugin Mercurial per Eclipse}
MercurialEclipse (versione $\geq$ 1.7).

% \section{SDK per AppEngine}
% ???
% 
% \section{Plugin per AppEngine}
% ???

\section{Latex}
Sono stati individuati due possibili prodotti:
\begin{itemize}
  \item TeX Live (TexLive 2010)
  \item MiKTeX (versione $\geq$ 2.9)
\end{itemize}

\section{Plugin Latex per Ecplise}
TeXlipse (versione $\geq$ 1.4.0).

\section{UML}
ArgoUML (versione $\geq$ 0.30.2) configurato per \underline{UML} 2.x.

\section{Gantt}
GanttProject (versione $\geq$ 2.0.10).



\chapter{Repository e Versionamento}
\thispagestyle{fancy}
Come servizio di hosting per mantenere online i nostri repository
verr\`a utilizzato Google Code Project Hosting
(http://code.google.com/hosting/), che oltre ad offrire un buon servizio di
\underline{ticketing}, \`e dotato di una comoda e completa interfaccia web per
la visualizzazione e gestione dei dati caricati. Dentro a tale spazio di hosting
verranno creati due repository: uno per i documenti e uno per il codice
sorgente. Come sistema di \underline{versionamento} per tali repository verr\`a
utilizzato Mercurial 1.6 (http://mercurial.selenic.com/), strumento che dovr\`a essere
installato nel sistema da ognuno dei componenti del gruppo. Poich\'e durante
l'intero sviluppo verr\`a principalmente utilizzato l'ambiente Eclipse,
utilizzeremo il comodo plugin Mercurial Eclipse che ci consentir\`a di accedere
graficamente a tutte le funzionalit\`a di Mercurial. Per scaricare localmente un
repository con le ultime versioni dei file si far\`a un PULL dentro il progetto
selezionato. Poi una volta effettuate delle modifiche, aggiunte o rimozioni
viene fatto un relativo COMMIT per descrivere l'avanzamento di versione e poi
viene fatto il PUSH per caricare tale versione nel repository. Mercurial
gestisce in maniera efficiente i vari conflitti che potrebbero esserci in PUSH
simultanei nello stesso branch.
Convenzione per i COMMIT:
\begin{itemize}
\item add: \`e stato aggiunto qualcosa di nuovo ad un file che gi\`a esisteva.
\item new: \`e stato creato qualcosa di nuovo, quindi un nuovo file. 
\item fix: \`e stato modificato un contenuto presente in un file che gi\`a
esisteva. 
\item del: \`e stato totalmente eliminato qualcosa.
\end{itemize}

\section{Registro delle modifiche}
Per agevolare la comprensione delle modifiche effettuate, all'interno di
ciascun documento sar\`a sempre presente un registro delle modifiche dettagliato (a
partire da pagina II prima dell'indice), che permetter\`a di evidenziare la data
d' introduzione della modifica, il numero di versione raggiunto e l'autore di
tale modifica.




\chapter{Ticketing e time-tracking}
\thispagestyle{fancy}
Il tracciamento delle ore di lavoro per ogni componente verr\`a eseguito tramite
un progetto di GanttProject appositamente condiviso ed accessibile da ognuno
(non concorrentemente). Questo permetter\`a anche un comodo confronto con la
gestione delle risorse pianificata nel PQ.

Il tiketing sar\`a utilizzato tramite gli appositi strumenti forniti da Google Code.



\chapter{Riunioni ed incontri}
\thispagestyle{fancy}
Tutte le riunioni ufficiali vengono svolte nelle seguenti sedi e date :
\begin{itemize}
\item SEDI:
\begin{itemize}
\item Residenza Messori(via Manin, 31 - Padova) - abbreviazione RS,
\item Laboratorio di Informatica(via Paolotti - Padova) - abbreviazione LI.
\end{itemize}
\end{itemize}

\chapter{Consegna}
\thispagestyle{fancy}
Per ciascuna revisione, il gruppo VT.G si impegna a consegnare i documenti
tramite l'eMail del gruppo in un unico file compresso .zip: all'interno i file
saranno suddivisi in base all'uso (interno o esterno).

	

\end{document}